\documentclass{article}
\usepackage{amsmath}
\usepackage{amsfonts}

\begin{document}

\section*{Quiz 4 – Solutions}

\subsection*{Question 1: Discrete Random Variable}

The die has faces: \{1, 2, 2, 3, 3, 4\}, so the probability distribution is:
\[
P(1) = \frac{1}{6}, \quad P(2) = \frac{2}{6}, \quad P(3) = \frac{2}{6}, \quad P(4) = \frac{1}{6}
\]

We define \( Z = X_1 + X_2 \). The maximum value of \( Z \) is \( 4 + 4 = 8 \).

\[
P(Z = 8) = P(X_1 = 4, X_2 = 4) = \frac{1}{6} \cdot \frac{1}{6} = \frac{1}{36}
\]
\[
P(Z > 7) = P(Z = 8) =  \boxed{\frac{1}{36}}
\]

\subsection*{Question 2: Bayes' Rule}

Given:
\[
P(G) = 0.002, \quad P(E \mid G) = 0.75, \quad P(E \mid G^c) = 0.05
\]

Using Bayes' Rule:
\[
P(G \mid E) = \frac{P(E \mid G) P(G)}{P(E \mid G)P(G) + P(E \mid G^c)P(G^c)}
\]
\[
= \frac{0.75 \cdot 0.002}{0.75 \cdot 0.002 + 0.05 \cdot 0.998} = \frac{0.0015}{0.0514} \approx \boxed{0.0292}
\]

\subsection*{Question 3: Binomial and Complement Rule}

Let \( Y \sim \text{Binomial}(n = 5, p = 0.6) \), where \( Y \) is the number of votes in favor.

We are asked:
\[
P(\text{At least 2 vote against}) = P(Y \leq 3)
\]

This can be computed directly:
\[
P(Y \leq 3) = \sum_{k=0}^{3} \binom{5}{k} (0.6)^k (0.4)^{5-k}
\]
\[
= 0.01024 + 0.0768 + 0.2304 + 0.3456 = \boxed{0.6630}
\]

\paragraph*{Note: Complement Rule}

Alternatively, we can apply the complement rule:
\[
P(Y \leq 3) = 1 - P(Y > 3) = 1 - P(Y \in \{4, 5\})   = 1 - P(Y = 4) - P(Y = 5)
\]
\[
P(Y = 4) = \binom{5}{4} (0.6)^4 (0.4)^1 = 5 \cdot 0.1296 \cdot 0.4 = 0.2592
\]
\[
P(Y = 5) = \binom{5}{5} (0.6)^5 (0.4)^0 = 1 \cdot 0.07776 = 0.07776
\]
\[
P(Y \leq 3) = 1 - (0.2592 + 0.07776) = 1 - 0.33696 = \boxed{0.6630}
\]

Both methods yield the same result.

\end{document}
