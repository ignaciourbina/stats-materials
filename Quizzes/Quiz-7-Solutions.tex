\documentclass[12pt]{article}
\usepackage[utf8]{inputenc} % Allows direct UTF-8 input
\usepackage{amsmath}        % For advanced math typesetting
\usepackage{amssymb}        % For math symbols like \mathbb
\usepackage{amsfonts}       % For math fonts
\usepackage{geometry}       % For page layout customization
\usepackage{enumitem}       % For list customization
% \usepackage{array}         % For more control over arrays and tables (not strictly needed here)
\usepackage{wasysym}    % https://aneescraftsmanship.com/circle-symbols%E2%97%8B%E2%97%8F%E2%97%8D%E2%97%97%E2%97%94%E2%97%99%E2%A6%BF-in-latex/
\usepackage{stix}  % adds symbols and makes New Times Roman defaul

% Page geometry: A4 paper with 1-inch margins
\geometry{a4paper, margin=1in}

% Define a command for consistent spacing after question/justification blocks
\newcommand{\questionsep}{\vspace{1em}}

\begin{document}

% Title block
\begin{center}
    \Large\textbf{POL 201.30 – Introduction to Statistical Methods in Political Science} \\
    \vspace{0.5em} % Vertical space between title lines
    \large\textbf{Online Quiz \#7 – Solution Key}
\end{center}
\vspace{1em} % Vertical space after title block

% Total Quiz Score section
\noindent\textbf{Total Quiz Score:}
\begin{itemize}[label=\textbullet, leftmargin=*, itemsep=0.2em, topsep=0.2em]
    \item 13 points.
\end{itemize}
\vspace{0.5em}

% Notes section
\noindent\textbf{Notes:}
\begin{itemize}[label=\textbullet, leftmargin=*, itemsep=0.2em, topsep=0.2em]
    \item For “Multiple-Choice” questions with just one correct answer:
    \begin{itemize}[label=-, leftmargin=*, itemsep=0.2em, topsep=0.2em] % Nested list for the note detail
        \item The “$\mdlgblkcircle$” symbol next to a bolded option represents the correct answer.
    \end{itemize}
\end{itemize}
\questionsep

% --- Section 1 ---
\section*{Section 1} 

% Question 1
\subsection*{Question 1 \normalfont{(1 point)}}
Which of the following best describes the role of the null hypothesis in hypothesis testing for a proportion?

\medskip\noindent\textbf{Question 1. Options:}
\begin{itemize}[leftmargin=2em, labelsep=0.5em, itemsep=0.3em, topsep=0.3em]
    \item[$\bigcirc$] It assumes the sample proportion is the true population proportion (0 point)
    \item[$\bigcirc$] It claims that the sample was biased and not randomly selected (0 point)
    \item[$\mdlgblkcircle$] \textbf{It specifies a fixed value of the population proportion to test against (+1 point)}
    \item[$\bigcirc$] It adjusts the population proportion based on the sample data (0 point)
\end{itemize}

\medskip\noindent\textbf{Justification:}

The answer is “It specifies a fixed value of the population proportion to test against”. Hypothesis testing is used when we want to test whether the observed proportion statistically differs from the null value. It allows us to determine if there is enough evidence to support a claim about the population proportion. The formulating hypotheses are the Null Hypothesis and Alternative Hypothesis. The null hypothesis assumes the population proportion equals the specified null value. This is done by setting a fixed point to test against. An example would be setting $H_0: p = 0.5$. Here, 0.5 is the set value, and we test for alternative outcomes such as $p > 0.5$ or $p < 0.5$. In simple terms, it is a fixed value that we test our data against to decide whether the true population proportion statistically differs from that fixed value.
\questionsep

% Question 2
\subsection*{Question 2 \normalfont{(1 point)}}
What does it mean if your calculated $z$-statistic falls inside the rejection region?

\medskip\noindent\textbf{Question 2. Options:}
\begin{itemize}[leftmargin=2em, labelsep=0.5em, itemsep=0.3em, topsep=0.3em]
    \item[$\mdlgblkcircle$] \textbf{The sample result is significantly different from the null, so you reject the null hypothesis (1 point)}
    \item[$\bigcirc$] The null hypothesis is definitely true (0 point)
    \item[$\bigcirc$] You need to collect more data before making a conclusion (0 point)
    \item[$\bigcirc$] The alternative hypothesis is definitely true (0 point)
\end{itemize}

\medskip\noindent\textbf{Justification:}

The answer is: “The sample result is significantly different from the null, so you reject the null hypothesis.”

When your calculated $z$-statistic lands in the rejection region—which is set based on your chosen significance level, $\alpha$—it means the result you observed in your sample is pretty far off from the range of outcomes we’d usually expect if the null hypothesis were actually true. In plain terms, it’s saying: “Hey, this result is just too unlikely to happen by random chance under the null.” That small probability tells us that the observed result isn’t just a fluke of sampling. Statistically, we’d call it significant—it provides strong enough evidence to cast doubt on the null hypothesis. So, based on that, we reject the null and lean toward the alternative hypothesis. It’s not a guarantee that the alternative is true, but the data makes a compelling case that the null probably isn’t.

% --- Section 2 ---
\section*{Section 2}

Suppose you are part of a research team that is investigating how students who have declared a political science major feel about the president’s handling of the economy. You ran a survey that measures whether each student either; approves (1) or disapproves (0) of the president’s economic policies.

Suppose you have been given a random sample of 100 political science majors, drawn from the student population at Stony Brook. Further, suppose that the calculated sample proportion of students who approve is 58\%. The goal is to use this sample to test whether the population proportion of students who approve is statistically different from 0.50.

% Question 3
\subsection*{Question 3 \normalfont{(1 point)}}
Which of the following correctly states the null and alternative hypotheses?

\medskip\noindent\textbf{Question 3 Options:}
\begin{itemize}[leftmargin=2em, labelsep=0.5em, itemsep=0.3em, topsep=0.3em]
    \item[$\bigcirc$] $H_0: p = 0.50 \quad H_A: p > 0.50$ (0 point)
    \item[$\mdlgblkcircle$] \textbf{$H_0: p = 0.50 \quad H_A: p \neq 0.50$ (1 point)}
    \item[$\bigcirc$] $H_0: p = 0.50 \quad H_A: p < 0.50$ (0 point)
    \item[$\bigcirc$] $H_0: p = 0.50 \quad H_A: \hat{p} \neq 0.50$ (0 point)
\end{itemize}

\medskip\noindent\textbf{Justification:}

The answer is $H_0\!: p = 0.50 \quad H_A\!: p \neq 0.50$.  
This is the answer because when we are testing to see if our observed sample proportion is statistically significant in differing from a hypothesized population proportion, we set $H_0$ to a specific value, in this case 0.50, and $H_A$ represents the claim that the true population proportion is any value other than 0.50, so $p \neq 0.50$.  
This setup corresponds to a \textbf{two-tailed} test for a single population proportion.  
\emph{Remember, though, that the tail structure must match the substantive research question:} if the study is concerned only with detecting an increase (or only a decrease), the appropriate alternative would be one-tailed—$H_A\!: p > 0.50$ or $H_A\!: p < 0.50$—and the rejection region would lie entirely in that single tail.

\questionsep

% Question 4
\subsection*{Question 4 \normalfont{(1 point)}}
Which of the following correctly states whether the normal approximation is appropriate as the sampling distribution of the test statistic? (Justify your answer with the appropriate calculation in your hand-written work).

\medskip\noindent\textbf{Question 4 Options:}
\begin{itemize}[leftmargin=2em, labelsep=0.5em, itemsep=0.3em, topsep=0.3em]
    \item[$\bigcirc$] Yes it is appropriate because $\hat{p}$ is close to 0.50 (0 point)
    \item[$\mdlgblkcircle$] \textbf{Yes it is appropriate because both $np_0$ and $n(1-p_0)$ are at least 15 (1 point)}
    \item[$\bigcirc$] No it is not appropriate because we don’t know the standard deviation (0 point)
    \item[$\bigcirc$] No it is not appropriate because the sample size is not greater than 30 (0 point)
\end{itemize}

\medskip\noindent\textbf{Justification:}

The answer is “Yes it is appropriate because both $np_0$ and $n(1-p_0)$ are at least 15”. This refers to the "success-failure condition" for using a normal approximation for the sampling distribution of a sample proportion. The conditions that need to be met are $np_0 \geq 15$ (at least 15 expected successes) and $n(1-p_0) \geq 15$ (at least 15 expected failures). If $n = 100$ and the hypothesized population proportion $p_0$ is 0.50:
The first part of the check, $np_0$, is $100 \times 0.50 = 50$. Since $50 \geq 15$, this condition is met.
The second part of the check, $n(1-p_0)$, is $100 \times (1-0.50) = 100 \times 0.50 = 50$. Since $50 \geq 15$, this condition is also met.
Because both conditions of the success-failure rule are satisfied, the normal approximation is appropriate as the sampling distribution of the test statistic. Being close to 0.50 (for $\hat{p}$ or $p_0$) is not the formal condition, and the sample size being greater than 30 is a general guideline for the Central Limit Theorem concerning means, not specifically for proportions (where the success-failure condition is key).
\questionsep

% Question 5
\subsection*{Question 5 \normalfont{(1 point)}}
Using the provided app, compute the test statistic. Which of the following is closest to the correct value of the test statistic? (Do not forget to upload a snapshot of the output given by the app at the end of the quiz).

\medskip\noindent\textbf{Question 5. Options:}
\begin{itemize}[leftmargin=2em, labelsep=0.5em, itemsep=0.3em, topsep=0.3em]
    \item[$\mdlgblkcircle$] \textbf{$z \approx 1.60$ (1 point)}
    \item[$\bigcirc$] $z \approx 10$ (0 point)
    \item[$\bigcirc$] $z \approx 1.96$ (0 point)
    \item[$\bigcirc$] $z \approx 11.75$ (0 point)
\end{itemize}

\medskip\noindent\textbf{Justification:}

The answer is $z \approx 1.60$. To replicate this, the inputs for the app would be based on the scenario (which typically involves a sample size $n$, an observed number of successes $x$ or sample proportion $\hat{p} = x/n$, and the null hypothesis value $p_0$). Given the context provides $n=100$, $\hat{p}=0.58$ (so 58 successes), and $p_0=0.50$.
The formula for the test statistic $z$ is:
$z = \frac{\hat{p} - p_0}{\sqrt{\frac{p_0(1-p_0)}{n}}}$
$z = \frac{0.58 - 0.50}{\sqrt{\frac{0.50(1-0.50)}{100}}} = \frac{0.08}{\sqrt{\frac{0.25}{100}}} = \frac{0.08}{\sqrt{0.0025}} = \frac{0.08}{0.05} = 1.60$.
The output from the app, using these inputs, would show $z \approx 1.60$.
\questionsep

% Question 6
\subsection*{Question 6 \normalfont{(1 point)}}
Suppose the significance level is $\alpha = 0.05$. Which of the following best describes the rejection region for this two-sided test?

\medskip\noindent\textbf{Question 6. Options:}
\begin{itemize}[leftmargin=2em, labelsep=0.5em, itemsep=0.3em, topsep=0.3em]
    \item[$\mdlgblkcircle$] \textbf{Reject $H_0$ if $|z| > 1.96$ (1 point)}
    \item[$\bigcirc$] Reject $H_0$ if $z > 1.645$ (0 point)
    \item[$\bigcirc$] Reject $H_0$ if $z < -1.645$ (0 point)
    \item[$\bigcirc$] Reject $H_0$ if $|z| > 2.58$ (0 point)
\end{itemize}

\medskip\noindent\textbf{Justification:}

The answer is Reject $H_0$ if $|z| > 1.96$. For a two-sided test with a significance level of $\alpha = 0.05$, we divide $\alpha$ by 2 to find the area in each tail of the standard normal distribution, which is $0.05 / 2 = 0.025$. The $z$-critical values corresponding to an area of 0.025 in each tail are $z = \pm 1.96$. Therefore, we reject the null hypothesis $H_0$ if the absolute value of the calculated test statistic $|z|$ is greater than 1.96. This means we reject $H_0$ if $z > 1.96$ or if $z < -1.96$.
\questionsep

% Question 7
\subsection*{Question 7 \normalfont{(1 point)}}
Based on your test statistic and the rejection region, what is the correct decision?

\medskip\noindent\textbf{Question 7. Options:}
\begin{itemize}[leftmargin=2em, labelsep=0.5em, itemsep=0.3em, topsep=0.3em]
    \item[$\mdlgblkcircle$] \textbf{Fail to reject $H_0$; there is not sufficient evidence that the proportion is different from 0.50 (1 point)}
    \item[$\bigcirc$] Reject $H_0$; there is sufficient evidence that the proportion is different from 0.50 (0 point)
    \item[$\bigcirc$] Reject $H_0$; the sample proportion is numerically different than 0.5 (0 point)
    \item[$\bigcirc$] Fail to reject $H_0$; the population is not normal (0 point)
\end{itemize}

\medskip\noindent\textbf{Justification:}

The answer is “Fail to reject $H_0$; there is not sufficient evidence that the proportion is different from 0.50”. We fail to reject the null hypothesis $H_0$ because our calculated $z$ test statistic from Question 5 was approximately 1.60. The rejection region for a two-sided test at $\alpha = 0.05$ (from Question 6) is $|z| > 1.96$. Since $|1.60| = 1.60$, which is not greater than 1.96 (i.e., $1.60 \not> 1.96$), our test statistic does not fall into the rejection region. Therefore, there is not enough evidence at the 0.05 significance level to conclude that the true population proportion is different from 0.50.
\questionsep

% Question 8
\subsection*{Question 8 \normalfont{(1 point)}}
Which of the following is the most appropriate conclusion?

\medskip\noindent\textbf{Question 8. Options:}
\begin{itemize}[leftmargin=2em, labelsep=0.5em, itemsep=0.3em, topsep=0.3em]
    \item[$\mdlgblkcircle$] \textbf{There is not enough evidence to conclude that the approval rate among political science majors is different from 50\%. (1 point)}
    \item[$\bigcirc$] The sample shows that 50\% of political science majors approve, so no difference exists in the population (0 point)
    \item[$\bigcirc$] There is significant evidence that the approval rate among political science majors is greater than 50\%. (0 point)
    \item[$\bigcirc$] Political science majors strongly disapprove of the president's handling of the economy. (0 point)
\end{itemize}

\medskip\noindent\textbf{Justification:}

The answer is “There is not enough evidence to conclude that the approval rate among political science majors is different from 50\%”. This answer choice is the most appropriate conclusion because, as determined in Question 7, we failed to reject the null hypothesis ($H_0: p = 0.50$). Our calculated $z$-statistic of approximately 1.60 did not exceed the critical value of $1.96$ (for a two-sided test at $\alpha=0.05$). This means there is not enough statistical evidence from our sample to be confident that the true population proportion of political science majors who approve is different from 50\%. Option 2 is incorrect because our sample proportion was 58\%, not 50\%, and failing to reject $H_0$ doesn't mean $H_0$ is true. For option 3, we did not find statistically significant evidence of a difference, let alone that it is greater than 50\%. The last claim is not supported by the data; in our sample, 58\% approved, which is a majority approval, not strong disapproval.
\questionsep

\end{document}
