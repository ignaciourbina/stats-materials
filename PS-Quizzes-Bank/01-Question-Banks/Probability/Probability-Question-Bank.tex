\documentclass{article}
\usepackage{amsmath}
\usepackage{amssymb}
\usepackage{geometry}
\geometry{margin=1in}
\usepackage{booktabs}

\title{Quiz 2}
\author{POL 201 - POL 501}
\date{}

\begin{document}

\maketitle

\section*{Part 1: Instructions}

For the next problems, you have to reason and justify whether the given statement is correct or incorrect. First, state whether the statement is correct and then proceed to provide a  justification for your argument. Most of the score will be assigned to the justification.

In most problems, you will need to justify your answer by showing probability calculations and being transparent about the intermediate steps. You will have points deducted if you don't justify your answers appropriately.

\subsection*{Question 1a (2 points)}

A survey was conducted to understand the voting preferences of urban and rural residents in a county. The survey included 1,000 randomly selected participants. Out of these, 400 were from urban areas, and 600 were from rural areas. Among urban residents, 280 expressed support for the local mayor, while 360 rural residents expressed support for the mayor.

\textbf{Statement:} ``Given this data, an urban resident is more likely to support the mayor compared to a rural resident.''

\subsection*{Question 1b (2 points)}

A local school board conducted a study of 900 students to examine support for a new extracurricular program. Of these students, 360 attended private schools and 540 attended public schools. In the private school group, 252 students favored the new program. In the public school group, 324 students favored it.

\textbf{Statement:} “According to this study, a student from a private school is more likely to favor the new program than a student from a public school.”

\subsection*{Question 2 (2 points)}

A survey conducted among 1,200 adults showed the following results: 400 adults exercise regularly (at least three times a week). Among those who exercise regularly, 320 report being in excellent health. Among those who do not exercise regularly, 200 report being in excellent health.

\textbf{Statement:} ``There is a higher probability of being in excellent health for people who exercise regularly versus those who don't''

\subsection*{Question 3 (2 points)}

A survey of 1,000 adults found the following: 500 adults use social media regularly. 200 adults shop online regularly. 400 adults do both.

\textbf{Statement:} ``The probability that a person in the survey uses social media regularly or shops online is greater than 50\%.''

\subsection*{Question 4 (2 points)}

A survey was conducted to understand the attitudes about a new healthcare policy and the voting behavior of adults in a city. The following probabilities were reported:
\begin{itemize}
    \item Probability that an adult supports the new healthcare policy: \( P(A) = 0.40 \)
    \item Probability that an adult voted in the last election: \( P(B) = 0.60 \)
    \item Probability that an adult supports the new healthcare policy given that they voted in the last election: \( P(A|B) = 0.50 \)
\end{itemize}

\textbf{Statement:} ``The probability that an adult both supports the new healthcare policy and voted in the last election is 0.24.''

\subsection*{Question 5 (2 points)}

A survey asked people to pick one and only one of three mutually exclusive policy proposals: Policy A, Policy B, and Policy C. The probabilities are as follows:
\begin{itemize}
    \item Probability that a person supports Policy B: \( P(B) = 0.30 \)
    \item Probability that a person supports Policy C: \( P(C) = 0.50 \)
\end{itemize}

\textbf{Statement:} ``The probability of a person choosing policy A is 0.20.''

\section*{Part 2}

\subsection*{Question 6: Regime Type and Economic Inequality}

Consider the following joint frequency distribution of regime type and economic inequality for a set of countries. The table shows the number of countries associated with each type of regime and level of economic inequality. There are a total of 200 countries.

\begin{table}[h!]
\centering
\begin{tabular}{|c|c|c|c|}
\hline
Regime & Low Inequality & Medium Inequality & High Inequality \\
\hline
Full Democracy & 30 & 20 & 10 \\ %60
\hline
Flawed Democracy & 20 & 30 & 20 \\ % 70
\hline
Electoral Autocracy  & 10 & 20 & 40 \\ % 70 == 200
\hline
\end{tabular}
\end{table}

Based on this table, calculate the following probabilities. Important: justify your calculations with the appropriate formulas.

\begin{enumerate}
    \item[a)] The probability that a country is a Full Democracy . (1 point)
    \item[b)] The probability that a country does not have Medium Inequality . (1 point)
    \item[c)] The probability that a country is a Full Democracy and does not have Medium Inequality. (1 point)
    \item[d)] The probability that a country is a Full Democracy given that it has Medium Inequality. (1 point)
    \item[e)] The probability that a country is either a Flawed Democracy or has High Inequality . (1 point)
\end{enumerate}

\section*{Other questions}

Political‐Debate Viewing Habits

A nationwide poll reports the following about how adults keep up with political debates:

\begin{center}
\begin{tabular}{lcc}
\toprule
\textbf{Behavior} & \textbf{Symbol} & \textbf{Probability} \\ \midrule
Watches debates on television & $T$ & $0.30$ \\
Follows debates online        & $O$ & $0.25$ \\
Does \emph{both} ($T\cap O$)   &  & $0.10$ \\ \bottomrule
\end{tabular}
\end{center}

Assume these percentages represent probabilities for a randomly selected adult.  Answer the following:

\begin{enumerate}
  \item[\textbf{1.}] (\emph{Union})\; Find the probability that a randomly chosen person \emph{either} watches the debates on TV \emph{or} follows them online, i.e.\ $P(T\cup O)$.
  \item[\textbf{2.}] (\emph{Conditional probability})\; Given that a person follows debates online, what is the probability that they also watch them on TV?  Compute $P(T \mid O)$.
  \item[\textbf{3.}] (\emph{Independence check})\; Are the events $T$ and $O$ independent?  Show your calculation and give a brief justification.
\end{enumerate}

\vspace{1em}
\hrule
\vspace{1em}

\textbf{Answers}

\begin{enumerate}
  \item[\textbf{1.}] $P(T\cup O)=P(T)+P(O)-P(T\cap O)=0.30+0.25-0.10=0.45$
  \item[\textbf{2.}] $P(T\mid O)=\dfrac{P(T\cap O)}{P(O)}=\dfrac{0.10}{0.25}=0.40$
  \item[\textbf{3.}] Independence would require
    \[
      P(T\cap O)=P(T)P(O)=0.30\times 0.25=0.075.
    \]
    Because the observed $P(T\cap O)=0.10\neq 0.075$, the events are \textbf{not independent}.
\end{enumerate}




\end{document}
