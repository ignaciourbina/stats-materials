\documentclass{article}
\usepackage{enumitem}

\begin{document}

\section*{Multiple Choice Questions: Please clearly circle the best answer. Each question is worth 1 point.}

\begin{enumerate}[label=\textbf{Q\arabic*:}]
    \item A researcher has measured the height of a sample of US adults. What measures of central tendency is it possible to compute on this data? Hint: First, ask yourself what level of measurement is height.
    \begin{enumerate}[label=\alph*)]
        \item Mode only
        \item Mean, Median, and Mode
        \item Median and Mode
        \item Mean only
    \end{enumerate}

    \item A researcher asks each person in a sample what political party they belong to. What measures of central tendency is it possible to compute on this data? Hint: First, ask yourself what level of measurement is political party.
    \begin{enumerate}[label=\alph*)]
        \item Mean, Median, and Mode
        \item Median and Mode
        \item Mode only
        \item Median only
    \end{enumerate}

    \item Descriptive statistics are \_\_\_\_\_\_; inferential statistics are \_\_\_\_\_\_\_\_. Fill in the blanks in the correct order.
    \begin{enumerate}[label=\alph*)]
        \item Related to statistics and parameters; Related to samples and populations
        \item Ways of summarizing or condensing data; Ways of making decisions or drawing conclusions from data
        \item Ways of making decisions or drawing conclusions from data; Ways of summarizing or condensing data
        \item means and standard deviations; modes and ranges
    \end{enumerate}

    \item When computing confidence intervals, researchers have to choose the level of confidence they want. What happens to the size of a confidence interval when a researcher increases the level of confidence they want?
    \begin{enumerate}[label=\alph*)]
        \item It depends on the standard error of the mean
        \item The confidence interval gets smaller
        \item The confidence interval gets larger
        \item The size of confidence interval is not affected by the confidence level
    \end{enumerate}

    \item A statistic is \_\_\_\_\_\_; a parameter is \_\_\_\_\_\_. Fill in the blanks in the correct order.
    \begin{enumerate}[label=\alph*)]
        \item Calculated using interval/ratio variables; Calculated using ordinal data
        \item Calculated using ordinal data; Calculated using interval/ratio variables
        \item Calculated from a sample; Calculated from a population
        \item Calculated from a population; Calculated from a sample
    \end{enumerate}

    \item Which is \textbf{NOT} a property of the Normal Distribution?
    \begin{enumerate}[label=\alph*)]
        \item Symmetry
        \item Bell-shape
        \item Mode, Median, Mean are identical
        \item All of these are properties of the Normal Distribution
    \end{enumerate}

    \item A researcher observes children and finds that bigger children are more aggressive. He concludes that size causes aggression. What is problematic with his conclusion?
    \begin{enumerate}[label=\alph*)]
        \item Correlation does not necessarily mean one variable causes another
        \item He used a sample; all research must always study the population
        \item Other observational studies have found similar results
        \item There is nothing wrong with his conclusion
    \end{enumerate}

    \item A sampling distribution is:
    \begin{enumerate}[label=\alph*)]
        \item A way of determining who should be in your sample
        \item The probability distribution of means for all possible samples of a fixed size
        \item The distribution of raw scores you observed in your sample
        \item One type of frequency distribution
    \end{enumerate}

    \item Research that analyzes a portion of an entire population is carried out on a:
    \begin{enumerate}[label=\alph*)]
        \item sample
        \item population
        \item census
        \item survey
    \end{enumerate}

    \item Researchers flip a coin to determine who receives a mild electrical shock and who does not in their sample. They then measure cortisol, a hormone that indicates a person is feeling stressed. This type of research is an example of:
    \begin{enumerate}[label=\alph*)]
        \item an observational study
        \item an experiment
        \item participant observation
        \item a sampling distribution
    \end{enumerate}

\end{enumerate}


\section*{Multiple Choice Questions: Please clearly circle the best answer. Each question is worth 2 points.}

\begin{enumerate}[label=\textbf{Q\arabic*:}]
    \item The end result of a hypothesis test is:
    \begin{enumerate}[label=\alph*)]
        \item A decision to reject or retain the null hypothesis
        \item A description of how large or substantive the difference is between means
        \item Infallible knowledge of which hypothesis is correct
        \item None of the above
    \end{enumerate}

    \item Which one of these is a type of null hypothesis:
    \begin{enumerate}[label=\alph*)]
        \item There is no correlation between two variables
        \item There is no difference between two means
        \item There is no difference between three means 
        \item All of the above are null hypotheses
    \end{enumerate}

    \item In hypothesis testing, what is $\alpha$?
    \begin{enumerate}[label=\alph*)]
        \item The Type I Error rate
        \item A value conventionally set by scientists, usually as .05
        \item A number you pick ahead of time that determines the critical value(s)
        \item All of the above are definitions of $\alpha$
    \end{enumerate}

    \item A p-value is:
    \begin{enumerate}[label=\alph*)]
        \item The probability the null hypothesis is true
        \item The probability the null hypothesis is false
        \item The proportion of a null hypothesis distribution at or more extreme than your test statistic
        \item A measure of whether effect of one factor depends on the levels of another factor
    \end{enumerate}

    \item The power of a test is:
    \begin{enumerate}[label=\alph*)]
        \item The same thing as the standard error of a test
        \item A description of how large or substantive the difference is between means
        \item The probability you successfully reject the null when the null is actually false
        \item None of the above
    \end{enumerate}

    \item A researcher conducts a multiple regression. She uses infants’ weight as the outcome variable and has two predictor variables: their socioeconomic status (SES) and whether their parents live together. The $b$ for SES is .30 and its significance value is .24. The $b$ for parents cohabitating is .7 and its significance value is .021. Which of the following is true?
    \begin{enumerate}[label=\alph*)]
        \item There is a significant relationship between parents cohabitating and weight
        \item There is a significant relationship between SES and weight
        \item The intercept is significant
        \item None of the above
    \end{enumerate}

    \item Go back to multiple choice problem 6. Assume that the variable for cohabitation is coded as follows: 0 = not living together; 1 = living together. Also assume that weight is measured in pounds. Which of the following is true?
    \begin{enumerate}[label=\alph*)]
        \item Cohabitation and SES are too strongly correlated to infer anything.
        \item Infants without cohabitating parents weigh .7 pounds more
        \item Infants with cohabitating parents weigh .7 pounds more
        \item None of the above
    \end{enumerate}

    \item A researcher finds that rural people are more likely to vote than urban people. However, a critic contends that this relationship does not really exist. Instead, rural people tend to be older and it is well-known that older people vote more. What method that we learned in class would be most suitable for testing the critic’s idea?
    \begin{enumerate}[label=\alph*)]
        \item Two-sample t-test
        \item Correlation
        \item Confidence intervals
        \item Partial correlation
    \end{enumerate}

    \item A researcher is interested in sex differences in spatial abilities, like map reading. He collects mean map reading scores from a sample of men and women. He then computes a hypothesis test and an effect size, Cohen’s $d$, on these means. The hypothesis test will tell the researcher \_\_\_\_\_\_\_\_; the effect size will tell the researcher \_\_\_\_\_\_\_\_.
    \begin{enumerate}[label=\alph*)]
        \item Whether there is a null hypothesis; whether the population is skewed
        \item Whether the population is skewed; whether there is a null hypothesis
        \item Whether the means differ; whether the difference between the means is substantive or meaningful
        \item Whether the difference between the means is substantive or meaningful; whether the means differ
    \end{enumerate}

    \item The rate of Type II Errors is symbolized by $\beta$. We learned in class that the power of statistical test is $1 - \beta$. We also learned in class that one of the easiest ways for a researcher to increase the power of statistical test is to:
    \begin{enumerate}[label=\alph*)]
        \item Run a correlation
        \item Increase the standard error
        \item Make $\alpha$ very small, such as .00001
        \item Increase the sample size
    \end{enumerate}

    \item A researcher wants to test a new math education technique. She measures math test scores in a group of students taught using traditional methods and in a different group of students taught using the new method. The math quiz she uses to assess their knowledge has never been used before in previous research. What kind of hypothesis test will she use?
    \begin{enumerate}[label=\alph*)]
        \item Two-sample t-test
        \item One-sample t-test
        \item Two-sample z-test
        \item ANOVA
    \end{enumerate}

    \item A polling firm wants to know whether people with larger incomes are more likely to support reforming social security. They collect a sample of 1,000 Americans and ask them to report their household’s total incomes and to rate, from 0 to 100, how in favor they are of reforming social security. What kind of hypothesis test should the polling firm use?
    \begin{enumerate}[label=\alph*)]
        \item Two-sample t-test
        \item One-sample t-test
        \item Correlation test
        \item Partial correlation test
    \end{enumerate}

    \item The personnel director at a large corporation wants to predict the number of projects new hires will complete in their first five years at the company, based on two factors: the new hires’ years of previous employment experience in the industry and whether or not they have an MBA. What kind of technique should the personnel director use?
    \begin{enumerate}[label=\alph*)]
        \item Correlation
        \item Multiple Regression
        \item Partial Correlation
        \item ANOVA
    \end{enumerate}

    \item A researcher wants to study the IQ of people seeking political office. She wonders whether they differ from IQ in the general population, which based on decades of research is known to have $\mu = 100$ and $\sigma = 10$. What would be the MOST APPRORIATE test for her to use?
    \begin{enumerate}[label=\alph*)]
        \item Dependent samples t-test (also called paired or repeated samples t-test)
        \item One-sample t-test
        \item Two-sample z-test
        \item One-sample z-test
    \end{enumerate}

    \item The campaign for a presidential candidate wants to know if a commercial they have created will increase support for their candidate. They collect data from a sample of 100 people. First, they measure how likely the people in the sample are to vote for the candidate. Next, they show everyone in the sample the commercial. Finally, they measure again how likely the sample is to vote for the candidate.
    \begin{enumerate}[label=\alph*)]
        \item Dependent samples t-test (also called paired or repeated samples t-test)
        \item One-sample t-test
        \item Two-sample t-test
        \item Two-sample z-test
    \end{enumerate}

    \item The National Bureau of Labor Statistics wants to know which white-collar jobs make more money. In particular, they are interested in lawyers, doctors, professors, and finance workers. What is the MOST APPROPRIATE test for the Bureau to use?
    \begin{enumerate}[label=\alph*)]
        \item Two-sample t-test
        \item One-sample t-test
        \item Two-sample z-test
        \item ANOVA
    \end{enumerate}

    \item A professor wants to know whether students’ hours spent doing statistics homework is related to their final grade, after controlling for the effect of the amount of previous math education they have had. What test should the professor use?
    \begin{enumerate}[label=\alph*)]
        \item Two-sample t-test
        \item One-sample t-test
        \item Correlation test
        \item Partial correlation test
    \end{enumerate}

\end{enumerate}


\end{document}
