\documentclass[12pt]{article}
\usepackage{amsmath, amssymb, amsthm}
\usepackage{geometry}
\geometry{letterpaper, margin=1in}
\setlength{\parskip}{1em}
\setlength{\parindent}{0pt}

\begin{document}

\section*{Six Multiple-Choice Questions Emphasizing Estimators}

Below are six questions that highlight the role of \emph{estimators} in making inferences about populations based on sample data. Each question is followed by four answer options, with only one correct choice. These are purely conceptual and do not require numerical calculations.

\bigskip

\textbf{1. Which statement captures the primary goal of using an \emph{estimator} in statistics?}
\begin{enumerate}
\item[(A)] To remove randomness from the observed data.
\item[(B)] To generate a single numerical value that matches the population parameter.
\item[(C)] To forecast the probability for a range of values for the population using only observed frequencies.
\item[(D)] To use sample data to approximate a population quantity and acknowledge possible sampling variation.
\end{enumerate}
\textbf{Answer: (D)}

\bigskip

\textbf{2. Consider the following statement:}

\emph{“Since any estimate is one possible result from a single sample, we shouldn’t trust it as there is no way to know how representative it is.”}

Which of the following best evaluates the truth and reasoning of this statement?

\begin{enumerate}
\item[(A)] The statement is \textbf{True}, because estimates are inherently unreliable and can’t be evaluated.
\item[(B)] The statement is \textbf{True}, because without knowing the population, we have no grounds to assess an estimate’s accuracy.
\item[(C)] The statement is \textbf{False}, because although estimates come from samples, we can quantify their uncertainty using sampling distributions.
\item[(D)] The statement is \textbf{False}, because most sample estimates exactly match the population value in practice.
\end{enumerate}
\textbf{Answer: (C)}

\bigskip

\textbf{2. Suppose a survey estimates that 52\% of participants favor a new policy. From an inference standpoint, what is the key consideration for this estimated proportion?}
\begin{enumerate}
\item[(A)] Whether it is guaranteed to match the exact population proportion of 52\%.
\item[(B)] How well it accounts for the fact that the sample proportion can vary due to random sampling.
\item[(C)] Whether 52\% is larger than every other statistic we might collect.
\item[(D)] How to reduce the sample size so the estimate becomes simpler to calculate.
\end{enumerate}
\textbf{Answer: (B)}

\bigskip

\textbf{3. When we call $X$ a \emph{random variable} that underlies our observed data points $x_i$, what do we mean in the context of making estimates?}
\begin{enumerate}
\item[(A)] We treat each $x_i$ as a random draw from a distribution, helping us formalize the behavior of an estimator using $x_i$.
\item[(B)] $X$ is a fixed quantity that never changes.
\item[(C)] $X$ is always continuous, and cannot represent discrete outcomes.
\item[(D)] Each observed $x_i$ is unrelated to any population measure.
\end{enumerate}
\textbf{Answer: (A)}

\bigskip

\textbf{4. The \emph{expected value} of a random variable $X$ (often denoted $E[X]$) is conceptually important because:}
\begin{enumerate}
\item[(A)] It identifies the most likely single outcome that $X$ can take.
\item[(B)] It is a theoretical value describing the long-run average, which we often attempt to estimate using sample means.
\item[(C)] It is calculated by identifying the midpoint between the highest and lowest values of $X$.
\item[(D)] It remains purely symbolic and has no connection to real-world data or estimators.
\end{enumerate}
\textbf{Answer: (B)}

\bigskip

\textbf{5. In describing an \emph{estimator} for a population quantity, why do we talk about the \emph{distribution} of sample outcomes?}
\begin{enumerate}
\item[(A)] Because the distribution allows us to determine the most common value of the estimator and treat it as the population parameter.
\item[(B)] Because knowing the full distribution ensures that a single sample estimate will be unbiased.
\item[(C)] Because sample distributions always approximate the population distribution, regardless of sample size.
\item[(D)] Because it helps us understand how an estimator (like a sample mean or proportion) might vary across repeated samples.
\end{enumerate}
\textbf{Answer: (D)}

\bigskip

\textbf{6. Which statement best summarizes how \emph{modeling assumptions} aid in interpreting an estimator's result?}
\begin{enumerate}
\item[(A)] They ensure that the estimator will converge exactly to the population parameter, regardless of sample size.
\item[(B)] They justify treating the sample statistic as equal to the population value if the data fit the model well.
\item[(C)] They allow us to use theoretical tools (like probability distributions) to quantify sampling uncertainty, even when we only have one observed sample.
\item[(D)] They make it possible to generalize the estimator’s result to any population, regardless of how the data were collected.
\end{enumerate}
\textbf{Answer: (C)}

\bigskip

\textbf{End of Questions.}

\end{document}
