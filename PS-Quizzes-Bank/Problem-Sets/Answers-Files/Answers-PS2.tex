% Options for packages loaded elsewhere
\PassOptionsToPackage{unicode}{hyperref}
\PassOptionsToPackage{hyphens}{url}
%
\documentclass[
  11pt,
]{article}
\usepackage{amsmath,amssymb}
\usepackage{iftex}
\ifPDFTeX
  \usepackage[T1]{fontenc}
  \usepackage[utf8]{inputenc}
  \usepackage{textcomp} % provide euro and other symbols
\else % if luatex or xetex
  \usepackage{unicode-math} % this also loads fontspec
  \defaultfontfeatures{Scale=MatchLowercase}
  \defaultfontfeatures[\rmfamily]{Ligatures=TeX,Scale=1}
\fi
\usepackage{lmodern}
\ifPDFTeX\else
  % xetex/luatex font selection
\fi
% Use upquote if available, for straight quotes in verbatim environments
\IfFileExists{upquote.sty}{\usepackage{upquote}}{}
\IfFileExists{microtype.sty}{% use microtype if available
  \usepackage[]{microtype}
  \UseMicrotypeSet[protrusion]{basicmath} % disable protrusion for tt fonts
}{}
\makeatletter
\@ifundefined{KOMAClassName}{% if non-KOMA class
  \IfFileExists{parskip.sty}{%
    \usepackage{parskip}
  }{% else
    \setlength{\parindent}{0pt}
    \setlength{\parskip}{6pt plus 2pt minus 1pt}}
}{% if KOMA class
  \KOMAoptions{parskip=half}}
\makeatother
\usepackage{xcolor}
\usepackage[left=1in,right=1in,top=1in,bottom=1in]{geometry}
\usepackage{color}
\usepackage{fancyvrb}
\newcommand{\VerbBar}{|}
\newcommand{\VERB}{\Verb[commandchars=\\\{\}]}
\DefineVerbatimEnvironment{Highlighting}{Verbatim}{commandchars=\\\{\}}
% Add ',fontsize=\small' for more characters per line
\usepackage{framed}
\definecolor{shadecolor}{RGB}{248,248,248}
\newenvironment{Shaded}{\begin{snugshade}}{\end{snugshade}}
\newcommand{\AlertTok}[1]{\textcolor[rgb]{0.94,0.16,0.16}{#1}}
\newcommand{\AnnotationTok}[1]{\textcolor[rgb]{0.56,0.35,0.01}{\textbf{\textit{#1}}}}
\newcommand{\AttributeTok}[1]{\textcolor[rgb]{0.13,0.29,0.53}{#1}}
\newcommand{\BaseNTok}[1]{\textcolor[rgb]{0.00,0.00,0.81}{#1}}
\newcommand{\BuiltInTok}[1]{#1}
\newcommand{\CharTok}[1]{\textcolor[rgb]{0.31,0.60,0.02}{#1}}
\newcommand{\CommentTok}[1]{\textcolor[rgb]{0.56,0.35,0.01}{\textit{#1}}}
\newcommand{\CommentVarTok}[1]{\textcolor[rgb]{0.56,0.35,0.01}{\textbf{\textit{#1}}}}
\newcommand{\ConstantTok}[1]{\textcolor[rgb]{0.56,0.35,0.01}{#1}}
\newcommand{\ControlFlowTok}[1]{\textcolor[rgb]{0.13,0.29,0.53}{\textbf{#1}}}
\newcommand{\DataTypeTok}[1]{\textcolor[rgb]{0.13,0.29,0.53}{#1}}
\newcommand{\DecValTok}[1]{\textcolor[rgb]{0.00,0.00,0.81}{#1}}
\newcommand{\DocumentationTok}[1]{\textcolor[rgb]{0.56,0.35,0.01}{\textbf{\textit{#1}}}}
\newcommand{\ErrorTok}[1]{\textcolor[rgb]{0.64,0.00,0.00}{\textbf{#1}}}
\newcommand{\ExtensionTok}[1]{#1}
\newcommand{\FloatTok}[1]{\textcolor[rgb]{0.00,0.00,0.81}{#1}}
\newcommand{\FunctionTok}[1]{\textcolor[rgb]{0.13,0.29,0.53}{\textbf{#1}}}
\newcommand{\ImportTok}[1]{#1}
\newcommand{\InformationTok}[1]{\textcolor[rgb]{0.56,0.35,0.01}{\textbf{\textit{#1}}}}
\newcommand{\KeywordTok}[1]{\textcolor[rgb]{0.13,0.29,0.53}{\textbf{#1}}}
\newcommand{\NormalTok}[1]{#1}
\newcommand{\OperatorTok}[1]{\textcolor[rgb]{0.81,0.36,0.00}{\textbf{#1}}}
\newcommand{\OtherTok}[1]{\textcolor[rgb]{0.56,0.35,0.01}{#1}}
\newcommand{\PreprocessorTok}[1]{\textcolor[rgb]{0.56,0.35,0.01}{\textit{#1}}}
\newcommand{\RegionMarkerTok}[1]{#1}
\newcommand{\SpecialCharTok}[1]{\textcolor[rgb]{0.81,0.36,0.00}{\textbf{#1}}}
\newcommand{\SpecialStringTok}[1]{\textcolor[rgb]{0.31,0.60,0.02}{#1}}
\newcommand{\StringTok}[1]{\textcolor[rgb]{0.31,0.60,0.02}{#1}}
\newcommand{\VariableTok}[1]{\textcolor[rgb]{0.00,0.00,0.00}{#1}}
\newcommand{\VerbatimStringTok}[1]{\textcolor[rgb]{0.31,0.60,0.02}{#1}}
\newcommand{\WarningTok}[1]{\textcolor[rgb]{0.56,0.35,0.01}{\textbf{\textit{#1}}}}
\usepackage{graphicx}
\makeatletter
\def\maxwidth{\ifdim\Gin@nat@width>\linewidth\linewidth\else\Gin@nat@width\fi}
\def\maxheight{\ifdim\Gin@nat@height>\textheight\textheight\else\Gin@nat@height\fi}
\makeatother
% Scale images if necessary, so that they will not overflow the page
% margins by default, and it is still possible to overwrite the defaults
% using explicit options in \includegraphics[width, height, ...]{}
\setkeys{Gin}{width=\maxwidth,height=\maxheight,keepaspectratio}
% Set default figure placement to htbp
\makeatletter
\def\fps@figure{htbp}
\makeatother
\setlength{\emergencystretch}{3em} % prevent overfull lines
\providecommand{\tightlist}{%
  \setlength{\itemsep}{0pt}\setlength{\parskip}{0pt}}
\setcounter{secnumdepth}{-\maxdimen} % remove section numbering
\usepackage{fvextra}
\DefineVerbatimEnvironment{Highlighting}{Verbatim}{breaklines, breakanywhere, commandchars=\\\{\}}
\ifLuaTeX
  \usepackage{selnolig}  % disable illegal ligatures
\fi
\usepackage{bookmark}
\IfFileExists{xurl.sty}{\usepackage{xurl}}{} % add URL line breaks if available
\urlstyle{same}
\hypersetup{
  pdftitle={POL501 - Problem Set 2},
  pdfauthor={Answers to Questions},
  hidelinks,
  pdfcreator={LaTeX via pandoc}}

\title{POL501 - Problem Set 2}
\author{Answers to Questions}
\date{2024-11-05}

\begin{document}
\maketitle

{
\setcounter{tocdepth}{2}
\tableofcontents
}
\section{Grading Criteria}\label{grading-criteria}

\begin{itemize}
\tightlist
\item
  There will be \textbf{four problem sets} throughout the semester,
  which together account for \textbf{25\% of the final course grade}.
\item
  The total possible score for these problem sets is \textbf{25 out of
  100 points} (with 100 points being the maximum course score).
\item
  This particular problem set has a maximum score of \textbf{8 points}.
\item
  \textbf{Scoring Breakdown}:

  \begin{itemize}
  \tightlist
  \item
    Each sub-letter in Question 1 is worth \textbf{1 point} (Total of 5
    points)
  \item
    Each sub-letter in Question 2 is worth \textbf{0.5 points} (Total of
    3 points)
  \end{itemize}
\item
  \textbf{Grading Guidelines}:

  \begin{itemize}
  \tightlist
  \item
    Full credit will be awarded for answers that closely match the
    provided solutions.
  \item
    Partial credit will be given for incomplete or partially incorrect
    answers/justifications.
  \item
    \textbf{0 points} will be awarded for missing answers, answers with
    no justification, or entirely incorrect responses.
  \item
    Submissions without the RMD file will have point deductions.
  \end{itemize}
\end{itemize}

\newpage

\section{Question 1 - Solutions}\label{question-1---solutions}

Let's start by describing the data using summary statistics.

\begin{Shaded}
\begin{Highlighting}[]
\NormalTok{knitr}\SpecialCharTok{::}\FunctionTok{kable}\NormalTok{(}\FunctionTok{describe}\NormalTok{(df\_clean)[}\FunctionTok{c}\NormalTok{(}\StringTok{\textquotesingle{}vars\textquotesingle{}}\NormalTok{, }\StringTok{\textquotesingle{}n\textquotesingle{}}\NormalTok{, }\StringTok{\textquotesingle{}mean\textquotesingle{}}\NormalTok{, }\StringTok{\textquotesingle{}sd\textquotesingle{}}\NormalTok{, }\StringTok{\textquotesingle{}median\textquotesingle{}}\NormalTok{, }\StringTok{\textquotesingle{}min\textquotesingle{}}\NormalTok{, }\StringTok{\textquotesingle{}max\textquotesingle{}}\NormalTok{)], }\AttributeTok{format=}\StringTok{\textquotesingle{}latex\textquotesingle{}}\NormalTok{)}
\end{Highlighting}
\end{Shaded}

\begin{tabular}{l|r|r|r|r|r|r|r}
\hline
  & vars & n & mean & sd & median & min & max\\
\hline
RESPID & 1 & 5199 & 9457.661281 & 5483.2245412 & 9464 & 2 & 18838\\
\hline
PARTY & 2 & 5199 & 2.177150 & 0.9685161 & 2 & 1 & 4\\
\hline
INTFREQ & 3 & 5199 & 1.810156 & 0.8919744 & 2 & 1 & 5\\
\hline
RADIO & 4 & 5199 & 1.219465 & 0.4139242 & 1 & 1 & 2\\
\hline
ECON1MOD & 5 & 5199 & 2.730140 & 0.8352547 & 3 & 1 & 4\\
\hline
INFRASPEND & 6 & 5199 & 2.151568 & 0.9420048 & 2 & 1 & 5\\
\hline
MOREGUNIMPACT & 7 & 5199 & 1.878631 & 0.8552866 & 2 & 1 & 3\\
\hline
CRIMESAFE & 8 & 5199 & 2.575303 & 0.8459974 & 3 & 1 & 5\\
\hline
\end{tabular}

\subsection{Answer - Question (1.a)}\label{answer---question-1.a}

\textbf{\emph{Based on the survey data, calculate the probability that a
randomly selected respondent identifies as either a Democrat or a
Republican.}}

\begin{Shaded}
\begin{Highlighting}[]
\CommentTok{\# Step 1: Calculate the total number of valid respondents}
\NormalTok{n\_total\_q1 }\OtherTok{\textless{}{-}} \FunctionTok{nrow}\NormalTok{(df\_clean)}

\CommentTok{\# Step 2: Filter for respondents who identify as either Democrat or Republican}
\NormalTok{n\_rep\_OR\_dem }\OtherTok{\textless{}{-}}\NormalTok{ df\_clean }\SpecialCharTok{\%\textgreater{}\%}
  \FunctionTok{filter}\NormalTok{(PARTY }\SpecialCharTok{==} \DecValTok{1} \SpecialCharTok{|}\NormalTok{ PARTY }\SpecialCharTok{==} \DecValTok{2}\NormalTok{) }\SpecialCharTok{\%\textgreater{}\%}  \CommentTok{\# Logical OR condition to select Democrats (coded as 2) or Republicans (coded as 1)}
  \FunctionTok{nrow}\NormalTok{()}

\CommentTok{\# Step 2 (Equivalent Alternative): Filter for respondents who identify as either Democrat or Republican}
\NormalTok{n\_rep\_OR\_dem }\OtherTok{\textless{}{-}}\NormalTok{ df\_clean }\SpecialCharTok{\%\textgreater{}\%}
  \FunctionTok{filter}\NormalTok{(PARTY }\SpecialCharTok{\%in\%} \FunctionTok{c}\NormalTok{(}\DecValTok{1}\NormalTok{,}\DecValTok{2}\NormalTok{) ) }\SpecialCharTok{\%\textgreater{}\%}  \CommentTok{\# Logical OR condition to select Democrats (coded as 2) or Republicans (coded as 1)}
  \FunctionTok{nrow}\NormalTok{()}

\CommentTok{\# Step 3: Calculate the probability of identifying as Democrat or Republican}
\NormalTok{prob\_rep\_OR\_dem }\OtherTok{\textless{}{-}}\NormalTok{ n\_rep\_OR\_dem }\SpecialCharTok{/}\NormalTok{ n\_total\_q1}

\CommentTok{\# Step 4: Display the result}
\FunctionTok{cat}\NormalTok{(}\FunctionTok{paste}\NormalTok{(}\StringTok{"Probability of identifying as Democrat or Republican:"}\NormalTok{, }
            \FunctionTok{round}\NormalTok{(prob\_rep\_OR\_dem, }\DecValTok{2}\NormalTok{)))}
\end{Highlighting}
\end{Shaded}

\begin{verbatim}
## Probability of identifying as Democrat or Republican: 0.64
\end{verbatim}

\textbf{Explanation}: The probability of identifying as Democrat or
Republican is equal to 64 percent. This was computed by counting the
respondents who classified as Democrat \textbf{OR} Republican. Since
Democrat and Republican are mutually exclusive categories, it is valid
to add their counts. In other words, \textbf{Probability of Democrat or
Republican = (\#Democrats + \#Republicans)/(Total Respondents)}. This
ensures that we are not double-counting any respondents, as they can
belong to only one of these categories.

\subsection{Answer - Question (1.b)}\label{answer---question-1.b}

\textbf{\emph{What is the probability that a respondent listens to the
radio or uses the internet almost constantly?}}

\begin{Shaded}
\begin{Highlighting}[]
\CommentTok{\# Step 1: Calculate the total number of respondents}
\NormalTok{n\_total\_q1b }\OtherTok{\textless{}{-}} \FunctionTok{nrow}\NormalTok{(df\_clean)  }\CommentTok{\# This is the total number of respondents in the dataset}

\CommentTok{\# Step 2: Count respondents who listen to the radio}
\NormalTok{n\_radio\_yes }\OtherTok{\textless{}{-}}\NormalTok{ df\_clean }\SpecialCharTok{\%\textgreater{}\%}
  \FunctionTok{filter}\NormalTok{(RADIO }\SpecialCharTok{==} \DecValTok{1}\NormalTok{) }\SpecialCharTok{\%\textgreater{}\%}  \CommentTok{\# \textquotesingle{}RADIO == 1\textquotesingle{} filters for respondents who answered \textquotesingle{}Yes\textquotesingle{} to listening to the radio}
  \FunctionTok{nrow}\NormalTok{()  }\CommentTok{\# Counts the number of rows that meet the condition (i.e., number of people who listen to the radio)}

\CommentTok{\# Step 3: Count respondents who use the internet almost constantly}
\NormalTok{n\_inter\_constantly }\OtherTok{\textless{}{-}}\NormalTok{ df\_clean }\SpecialCharTok{\%\textgreater{}\%}
  \FunctionTok{filter}\NormalTok{(INTFREQ }\SpecialCharTok{==} \DecValTok{1}\NormalTok{) }\SpecialCharTok{\%\textgreater{}\%}  \CommentTok{\# \textquotesingle{}INTFREQ == 1\textquotesingle{} filters for respondents who use the internet \textquotesingle{}almost constantly\textquotesingle{}}
  \FunctionTok{nrow}\NormalTok{()  }\CommentTok{\# Counts the number of rows that meet the condition (i.e., number of people using internet almost constantly)}

\CommentTok{\# Step 4: Count respondents who both listen to the radio and use the internet almost constantly}
\NormalTok{n\_radio\_yes\_AND\_inter\_constantly }\OtherTok{\textless{}{-}}\NormalTok{ df\_clean }\SpecialCharTok{\%\textgreater{}\%}
  \FunctionTok{filter}\NormalTok{(RADIO }\SpecialCharTok{==} \DecValTok{1} \SpecialCharTok{\&}\NormalTok{ INTFREQ }\SpecialCharTok{==} \DecValTok{1}\NormalTok{) }\SpecialCharTok{\%\textgreater{}\%}  \CommentTok{\# \textquotesingle{}\&\textquotesingle{} is used for the logical AND to filter for people who listen to the radio AND use the internet almost constantly}
  \FunctionTok{nrow}\NormalTok{()  }\CommentTok{\# Counts the number of rows that meet both conditions}

\CommentTok{\# Step 5: Calculate probabilities for individual events and their intersection}
\NormalTok{prob\_event\_A }\OtherTok{\textless{}{-}}\NormalTok{ n\_radio\_yes }\SpecialCharTok{/}\NormalTok{ n\_total\_q1b  }\CommentTok{\# Probability of listening to the radio (Event A)}
\NormalTok{prob\_event\_B }\OtherTok{\textless{}{-}}\NormalTok{ n\_inter\_constantly }\SpecialCharTok{/}\NormalTok{ n\_total\_q1b  }\CommentTok{\# Probability of using the internet almost constantly (Event B)}
\NormalTok{prob\_A\_and\_B }\OtherTok{\textless{}{-}}\NormalTok{ n\_radio\_yes\_AND\_inter\_constantly }\SpecialCharTok{/}\NormalTok{ n\_total\_q1b  }\CommentTok{\# Probability of both listening to the radio AND using the internet almost constantly (A and B)}

\CommentTok{\# Step 6: Calculate the probability of either listening to the radio OR using the internet almost constantly}
\CommentTok{\# Using the formula P(A or B) = P(A) + P(B) {-} P(A and B) to avoid double{-}counting overlap}
\NormalTok{prob\_question\_1b }\OtherTok{\textless{}{-}}\NormalTok{ prob\_event\_A }\SpecialCharTok{+}\NormalTok{ prob\_event\_B }\SpecialCharTok{{-}}\NormalTok{ prob\_A\_and\_B}

\CommentTok{\# Step 7: Verify calculation using OR logic (alternative approach)}
\CommentTok{\# This calculates the number of respondents who meet either condition (radio OR internet almost constantly)}
\NormalTok{n\_A\_and\_B }\OtherTok{\textless{}{-}}\NormalTok{ df\_clean }\SpecialCharTok{\%\textgreater{}\%}
  \FunctionTok{filter}\NormalTok{(RADIO }\SpecialCharTok{==} \DecValTok{1} \SpecialCharTok{|}\NormalTok{ INTFREQ }\SpecialCharTok{==} \DecValTok{1}\NormalTok{) }\SpecialCharTok{\%\textgreater{}\%}  \CommentTok{\# Logical OR condition selects respondents who meet either condition}
  \FunctionTok{nrow}\NormalTok{()}
\NormalTok{prob\_A\_and\_B }\OtherTok{\textless{}{-}}\NormalTok{ n\_A\_and\_B }\SpecialCharTok{/}\NormalTok{ n\_total\_q1b  }\CommentTok{\# Calculate probability based on OR condition}

\CommentTok{\# Step 8: Print results to verify consistency between calculated and alternative method}
\FunctionTok{print}\NormalTok{(prob\_A\_and\_B)  }\CommentTok{\# Print probability from OR calculation}
\end{Highlighting}
\end{Shaded}

\begin{verbatim}
## [1] 0.8907482
\end{verbatim}

\begin{Shaded}
\begin{Highlighting}[]
\FunctionTok{print}\NormalTok{(prob\_question\_1b)  }\CommentTok{\# Print probability from formula calculation}
\end{Highlighting}
\end{Shaded}

\begin{verbatim}
## [1] 0.8907482
\end{verbatim}

\textbf{Explanation}: The probability that a respondent listens to the
radio or uses the internet almost constantly is equal to 89 percent. We
used the general OR formula:
\[\Pr(A\text{ or }B) = \Pr(A) + \Pr(B) - \Pr(A\text{ and }B)\] This
formula accounts for the overlap between the two groups to avoid
double-counting individuals who both listen to the radio and use the
internet almost constantly. Additionally, we verified this calculation
using an alternative method that directly counts the respondents meeting
either condition, which yielded the same result.

\subsection{Answer - Question (1.c)}\label{answer---question-1.c}

\textbf{\emph{What is the probability that a respondent believes there
would be more crime if more Americans owned guns and describes their
community as somewhat safe or safer?}}

\begin{Shaded}
\begin{Highlighting}[]
\CommentTok{\# Step 1: Calculate the total number of respondents}
\NormalTok{n\_total\_q1c }\OtherTok{\textless{}{-}} \FunctionTok{nrow}\NormalTok{(df\_clean)  }\CommentTok{\# This is the total number of respondents in the dataset}


\CommentTok{\# Construct binary variable for CRIMESAFE and MOREGUNIMPACT given the question prompt}
\NormalTok{df\_clean }\OtherTok{\textless{}{-}}\NormalTok{ df\_clean }\SpecialCharTok{\%\textgreater{}\%}
  \FunctionTok{mutate}\NormalTok{(}\AttributeTok{somwhsafe\_or\_safer =} \FunctionTok{if\_else}\NormalTok{(CRIMESAFE }\SpecialCharTok{\%in\%} \FunctionTok{c}\NormalTok{(}\DecValTok{1}\NormalTok{,}\DecValTok{2}\NormalTok{,}\DecValTok{3}\NormalTok{), }\DecValTok{1}\NormalTok{, }\DecValTok{0}\NormalTok{),}
         \AttributeTok{guns\_more\_crime =} \FunctionTok{if\_else}\NormalTok{(MOREGUNIMPACT }\SpecialCharTok{==} \DecValTok{1}\NormalTok{, }\DecValTok{1}\NormalTok{, }\DecValTok{0}\NormalTok{) }
\NormalTok{         )}

\CommentTok{\# Step 2: Count respondents with both somwhsafe\_or\_safer==1 (event A) AND guns\_more\_crime==1 (event B)}
\NormalTok{n\_A\_and\_B }\OtherTok{\textless{}{-}}\NormalTok{ df\_clean }\SpecialCharTok{\%\textgreater{}\%}
  \FunctionTok{filter}\NormalTok{(somwhsafe\_or\_safer}\SpecialCharTok{==}\DecValTok{1} \SpecialCharTok{\&}\NormalTok{ guns\_more\_crime}\SpecialCharTok{==}\DecValTok{1}\NormalTok{) }\SpecialCharTok{\%\textgreater{}\%}
  \FunctionTok{nrow}\NormalTok{()}

\CommentTok{\# Step 3: Compute the probability.}
\NormalTok{prob\_q1c }\OtherTok{\textless{}{-}}\NormalTok{ n\_A\_and\_B }\SpecialCharTok{/}\NormalTok{ n\_total\_q1c}

\FunctionTok{print}\NormalTok{(prob\_q1c)}
\end{Highlighting}
\end{Shaded}

\begin{verbatim}
## [1] 0.3869975
\end{verbatim}

\textbf{Explanation}:

Recall the coding of \texttt{CRIMESAFE}. How would you describe the area
where you live, in terms of crime?

-- 1 = Extremely safe; -- 2 = Very safe; -- 3 = Somewhat safe; -- 4 =
Not too safe; -- 5 = Not at all safe

Also, recall the coding for \texttt{MOREGUNIMPACT.} If more Americans
owned guns, do you think there would be\ldots{}

-- 1 = More crime; -- 2 = Less crime; -- 3 = No difference

A key step then is defining the binary variables per question prompt
requirements.

\textbf{Binary Variable Definitions}:

\begin{enumerate}
\def\labelenumi{\arabic{enumi}.}
\item
  \textbf{Binary Variable for Safety (\texttt{somwhsafe\_or\_safer})}:
  This variable is 1 if the respondent describes their community as
  ``somewhat safe'' or safer, otherwise 0.

  \[ 
  \texttt{somwhsafe\_or\_safer} = 
  \begin{cases} 
     1 & \text{if } CRIMESAFE \in \{1, 2, 3\} \\
     0 & \text{if } CRIMESAFE \in \{4, 5\}
  \end{cases}
  \]
\item
  \textbf{Binary Variable for Gun Impact (\texttt{guns\_more\_crime})}:
  This variable is 1 if the respondent believes that more Americans
  owning guns would lead to more crime, otherwise 0.

  \[ 
  \texttt{guns\_more\_crime} = 
  \begin{cases} 
     1 & \text{if } MOREGUNIMPACT = 1 \\
     0 & \text{if } MOREGUNIMPACT \in \{2, 3\}
  \end{cases}
  \]
\end{enumerate}

These definitions reflect how we classified each respondent into binary
categories based on their responses to the survey questions.

The probability that a respondent believes there would be more crime if
more Americans owned guns and describes their community as somewhat safe
or safer is equal to 39 percent. We used the classical probability
definition:
\[\Pr(A\text{ and }B) = \frac{\text{Cases with both A and B}}{\text{Total Cases in Sample Space}}\]
This calculation shows how we determine the likelihood that both events
occur simultaneously.

\subsection{Answer - Question (1.d)}\label{answer---question-1.d}

\textbf{\emph{Given that a respondent describes their community as
somewhat safe or safer, what is the probability that they believe there
would be more crime if more Americans owned guns?}}

\begin{Shaded}
\begin{Highlighting}[]
\NormalTok{n\_total\_sample }\OtherTok{\textless{}{-}} \FunctionTok{nrow}\NormalTok{(df\_clean)}

\NormalTok{n\_total\_event\_B }\OtherTok{\textless{}{-}}\NormalTok{ df\_clean }\SpecialCharTok{\%\textgreater{}\%}
  \FunctionTok{filter}\NormalTok{(somwhsafe\_or\_safer}\SpecialCharTok{==}\DecValTok{1}\NormalTok{) }\SpecialCharTok{\%\textgreater{}\%}
  \FunctionTok{nrow}\NormalTok{()}

\NormalTok{n\_A\_and\_B }\OtherTok{\textless{}{-}}\NormalTok{ df\_clean }\SpecialCharTok{\%\textgreater{}\%}
  \FunctionTok{filter}\NormalTok{(somwhsafe\_or\_safer}\SpecialCharTok{==}\DecValTok{1} \SpecialCharTok{\&}\NormalTok{ guns\_more\_crime}\SpecialCharTok{==}\DecValTok{1}\NormalTok{) }\SpecialCharTok{\%\textgreater{}\%}
  \FunctionTok{nrow}\NormalTok{()}

\NormalTok{prob\_A\_and\_B }\OtherTok{\textless{}{-}}\NormalTok{ n\_A\_and\_B }\SpecialCharTok{/}\NormalTok{ n\_total\_sample}
\NormalTok{prob\_B }\OtherTok{\textless{}{-}}\NormalTok{ n\_total\_event\_B }\SpecialCharTok{/}\NormalTok{ n\_total\_sample}

\CommentTok{\# Conditional Probability of A given B:}
\NormalTok{prob\_A\_given\_B }\OtherTok{\textless{}{-}}\NormalTok{ prob\_A\_and\_B }\SpecialCharTok{/}\NormalTok{ prob\_B}
\end{Highlighting}
\end{Shaded}

Define \textbf{Event B} as ``respondent describes their community as
somewhat safe or safer.'' Additionally, define \textbf{Event A} as
``they believe there would be more crime if more Americans owned guns.''
Then, \[\Pr(A | B) = \frac{P(A\text{ and }B)}{P(B)}\] Thus, given that a
respondent describes their community as somewhat safe or safer, the
probability that they believe there would be more crime if more
Americans owned guns is equal to 43 percent. This conditional
probability helps us understand the likelihood of one event occurring
given the occurrence of another event.

\subsection{Answer - Question (1.e)}\label{answer---question-1.e}

Create a 2x2 table examining the relationship between:

-- Perceptions of economic conditions in the community (grouped as
``Excellent/Good'' vs.~``Only fair/Poor'')

-- Support for increasing government spending on roads and bridges
(grouped as ``Increase a lot/little'' vs.~``Stay the same/Decrease'')

Using this table, calculate the conditional probability that a
respondent supports increased spending, given that they perceive
economic conditions as Excellent or Good.

\begin{Shaded}
\begin{Highlighting}[]
\CommentTok{\# Step 1: Create new binary variables for economic perceptions and spending support}
\NormalTok{df\_clean }\OtherTok{\textless{}{-}}\NormalTok{ df\_clean }\SpecialCharTok{\%\textgreater{}\%}
  \FunctionTok{mutate}\NormalTok{(}
    \AttributeTok{econ\_excell\_good =} \FunctionTok{if\_else}\NormalTok{(ECON1MOD }\SpecialCharTok{\%in\%} \FunctionTok{c}\NormalTok{(}\DecValTok{1}\NormalTok{, }\DecValTok{2}\NormalTok{), }\DecValTok{1}\NormalTok{, }\DecValTok{0}\NormalTok{),  }\CommentTok{\# 1 for "Excellent/Good", 0 for "Only fair/Poor"}
    \AttributeTok{incrspend\_alot\_alittle =} \FunctionTok{if\_else}\NormalTok{(INFRASPEND }\SpecialCharTok{\%in\%} \FunctionTok{c}\NormalTok{(}\DecValTok{1}\NormalTok{, }\DecValTok{2}\NormalTok{), }\DecValTok{1}\NormalTok{, }\DecValTok{0}\NormalTok{)  }\CommentTok{\# 1 for "Increase a lot/little", 0 for "Stay the same/Decrease"}
\NormalTok{  )}

\CommentTok{\# Step 2: Use dplyr to create the 2x2 contingency table}
\NormalTok{contingency\_table }\OtherTok{\textless{}{-}}\NormalTok{ df\_clean }\SpecialCharTok{\%\textgreater{}\%}
  \FunctionTok{group\_by}\NormalTok{(econ\_excell\_good, incrspend\_alot\_alittle) }\SpecialCharTok{\%\textgreater{}\%}
  \FunctionTok{summarise}\NormalTok{(}\AttributeTok{count =} \FunctionTok{n}\NormalTok{()) }\SpecialCharTok{\%\textgreater{}\%}
  \FunctionTok{ungroup}\NormalTok{() }
\end{Highlighting}
\end{Shaded}

\begin{verbatim}
## `summarise()` has grouped output by 'econ_excell_good'. You can override using
## the `.groups` argument.
\end{verbatim}

\begin{Shaded}
\begin{Highlighting}[]
\FunctionTok{print}\NormalTok{(contingency\_table)}
\end{Highlighting}
\end{Shaded}

\begin{verbatim}
## # A tibble: 4 x 3
##   econ_excell_good incrspend_alot_alittle count
##              <dbl>                  <dbl> <int>
## 1                0                      0  1125
## 2                0                      1  1892
## 3                1                      0   757
## 4                1                      1  1425
\end{verbatim}

\begin{Shaded}
\begin{Highlighting}[]
\CommentTok{\# We need some changes to get it to the correct 2x2 format}
\NormalTok{contingency\_table }\OtherTok{\textless{}{-}}\NormalTok{ contingency\_table }\SpecialCharTok{\%\textgreater{}\%}
  \FunctionTok{mutate}\NormalTok{(}\AttributeTok{econ\_excell\_good =} \FunctionTok{if\_else}\NormalTok{(econ\_excell\_good}\SpecialCharTok{==}\DecValTok{1}\NormalTok{, }
                                    \StringTok{\textquotesingle{}Econ: Excellent/Good\textquotesingle{}}\NormalTok{, }\StringTok{\textquotesingle{}Econ: Only fair/Poor\textquotesingle{}}\NormalTok{),}
         \AttributeTok{incrspend\_alot\_alittle =} \FunctionTok{if\_else}\NormalTok{(incrspend\_alot\_alittle}\SpecialCharTok{==}\DecValTok{1}\NormalTok{, }
                                          \StringTok{\textquotesingle{}Spend: Increase a lot/little\textquotesingle{}}\NormalTok{, }\StringTok{\textquotesingle{}Spend: Stay the same/Decrease\textquotesingle{}}\NormalTok{)) }\SpecialCharTok{\%\textgreater{}\%}
\NormalTok{  tidyr}\SpecialCharTok{::}\FunctionTok{spread}\NormalTok{(econ\_excell\_good, count, }\AttributeTok{fill =} \DecValTok{0}\NormalTok{)  }\CommentTok{\# Spread into a wide format table}

\CommentTok{\# Step 3: Print the contingency table}
\FunctionTok{print}\NormalTok{(contingency\_table)}
\end{Highlighting}
\end{Shaded}

\begin{verbatim}
## # A tibble: 2 x 3
##   incrspend_alot_alittle        `Econ: Excellent/Good` `Econ: Only fair/Poor`
##   <chr>                                          <dbl>                  <dbl>
## 1 Spend: Increase a lot/little                    1425                   1892
## 2 Spend: Stay the same/Decrease                    757                   1125
\end{verbatim}

\begin{Shaded}
\begin{Highlighting}[]
\NormalTok{knitr}\SpecialCharTok{::}\FunctionTok{kable}\NormalTok{(contingency\_table, }\AttributeTok{format=}\StringTok{\textquotesingle{}latex\textquotesingle{}}\NormalTok{)}
\end{Highlighting}
\end{Shaded}

\begin{center}
\begin{tabular}{l|r|r}
\hline
incrspend\_alot\_alittle & Econ: Excellent/Good & Econ: Only fair/Poor\\
\hline
Spend: Increase a lot/little & 1425 & 1892\\
\hline
Spend: Stay the same/Decrease & 757 & 1125\\
\hline
\end{tabular}
\end{center}

\begin{Shaded}
\begin{Highlighting}[]
\CommentTok{\# Step 4: Calculate the conditional probability of supporting increased spending given Excellent/Good economic conditions}
\NormalTok{n\_tot\_econ\_good }\OtherTok{\textless{}{-}} \FunctionTok{sum}\NormalTok{(contingency\_table}\SpecialCharTok{$}\StringTok{\textasciigrave{}}\AttributeTok{Econ: Excellent/Good}\StringTok{\textasciigrave{}}\NormalTok{)}
\NormalTok{n\_tot\_incr\_spend\_AND\_econ\_good }\OtherTok{\textless{}{-}}\NormalTok{ contingency\_table }\SpecialCharTok{\%\textgreater{}\%}
  \FunctionTok{filter}\NormalTok{(incrspend\_alot\_alittle}\SpecialCharTok{==}\StringTok{\textquotesingle{}Spend: Increase a lot/little\textquotesingle{}}\NormalTok{) }\SpecialCharTok{\%\textgreater{}\%}
  \FunctionTok{pull}\NormalTok{(}\StringTok{\textasciigrave{}}\AttributeTok{Econ: Excellent/Good}\StringTok{\textasciigrave{}}\NormalTok{)}

\CommentTok{\# Calculate the conditional probability}
\NormalTok{conditional\_prob\_incrspend\_given\_econ\_good }\OtherTok{\textless{}{-}}\NormalTok{ n\_tot\_incr\_spend\_AND\_econ\_good }\SpecialCharTok{/}\NormalTok{ n\_tot\_econ\_good}

\CommentTok{\# Print the conditional probability as a percentage}
\FunctionTok{cat}\NormalTok{(}\FunctionTok{paste0}\NormalTok{(}\StringTok{"Conditional Probability of supporting increased spending }\SpecialCharTok{\textbackslash{}n}\StringTok{given Excellent/Good economic conditions: "}\NormalTok{, }
            \FunctionTok{round}\NormalTok{(conditional\_prob\_incrspend\_given\_econ\_good }\SpecialCharTok{*} \DecValTok{100}\NormalTok{, }\DecValTok{2}\NormalTok{), }\StringTok{"\%"}\NormalTok{))}
\end{Highlighting}
\end{Shaded}

\begin{verbatim}
## Conditional Probability of supporting increased spending 
## given Excellent/Good economic conditions: 65.31%
\end{verbatim}

Recall:

\texttt{INFRASPEND.} Thinking about government spending on roads and
bridges in the area where you live, do you think this spending
should\ldots{} -- 1 = Increase a lot; -- 2 = Increase a little; -- 3 =
Stay about the same; -- 4 = Decrease a little; -- 5 = Decrease a lot

\texttt{ECON1MOD.} How would you rate economic conditions in your
community today? -- 1 = Excellent; -- 2 = Good; -- 3 = Only fair; -- 4 =
Poor

Then, the probability that a respondent supports increased spending,
given that they perceive economic conditions as Excellent or Good is
equal to 65.31\% percent. By grouping economic perceptions and support
for spending, we can better understand how positive economic outlooks
influence support for public investment.

\section{Question 2 - Solutions}\label{question-2---solutions}

\subsection{Answer - Question (2.a)}\label{answer---question-2.a}

\textbf{\emph{Calculate the Expected Value of the Number of
Participants}}

\begin{Shaded}
\begin{Highlighting}[]
\CommentTok{\# Define the possible values of the number of participants (in thousands)}
\NormalTok{P\_values }\OtherTok{\textless{}{-}} \FunctionTok{c}\NormalTok{(}\DecValTok{50000}\NormalTok{, }\DecValTok{75000}\NormalTok{, }\DecValTok{100000}\NormalTok{, }\DecValTok{125000}\NormalTok{, }\DecValTok{150000}\NormalTok{)}

\CommentTok{\# Define the corresponding probabilities of each value of P}
\NormalTok{P\_probs }\OtherTok{\textless{}{-}} \FunctionTok{c}\NormalTok{(}\FloatTok{0.08}\NormalTok{, }\FloatTok{0.22}\NormalTok{, }\FloatTok{0.31}\NormalTok{, }\FloatTok{0.24}\NormalTok{, }\FloatTok{0.15}\NormalTok{)}

\CommentTok{\# Calculate the expected value of the number of participants}
\NormalTok{E\_P }\OtherTok{\textless{}{-}} \FunctionTok{sum}\NormalTok{(P\_values }\SpecialCharTok{*}\NormalTok{ P\_probs)}

\CommentTok{\# Print the expected value}
\FunctionTok{print}\NormalTok{(}\FunctionTok{paste0}\NormalTok{(}\StringTok{"Expected value of the number of participants: "}\NormalTok{, E\_P))}
\end{Highlighting}
\end{Shaded}

\begin{verbatim}
## [1] "Expected value of the number of participants: 104000"
\end{verbatim}

\textbf{Explanation}: The expected value of the number of participants
is calculated using the formula for the expected value of a discrete
random variable: \[E[P] = \sum p_i x_i\] Where \(p_i\) represents the
probability of each possible outcome, and \(x_i\) represents each
possible outcome.

\subsection{Answer - Question (2.b)}\label{answer---question-2.b}

\textbf{\emph{Define the Total Monthly Cost as a Linear Combination}}

\begin{Shaded}
\begin{Highlighting}[]
\CommentTok{\# Define fixed administrative cost and monthly benefit per participant}
\NormalTok{fixed\_cost }\OtherTok{\textless{}{-}} \DecValTok{6000000}
\NormalTok{monthly\_benefit\_per\_participant }\OtherTok{\textless{}{-}} \DecValTok{200}

\CommentTok{\# Total monthly cost as a linear combination of P (number of participants)}
\CommentTok{\# C = fixed\_cost + (monthly\_benefit\_per\_participant * P)}
\FunctionTok{cat}\NormalTok{(}\StringTok{"Total Monthly Cost (C):"}\NormalTok{, }\StringTok{"C ="}\NormalTok{, fixed\_cost, }\StringTok{"+"}\NormalTok{, monthly\_benefit\_per\_participant, }\StringTok{"* P"}\NormalTok{)}
\end{Highlighting}
\end{Shaded}

\begin{verbatim}
## Total Monthly Cost (C): C = 6e+06 + 200 * P
\end{verbatim}

\textbf{Explanation}: The total monthly cost, C, is expressed as a
linear combination of the number of participants, P, the monthly benefit
per participant, and the fixed administrative cost. The fixed cost
remains constant, while the cost associated with participants is
proportional to the number of participants.

\[ C = 6000000 + 200 P\]

\subsection{Answer - Question (2.c)}\label{answer---question-2.c}

\textbf{\emph{Compute the Probability Mass Function of the Total Monthly
Cost}}

\begin{Shaded}
\begin{Highlighting}[]
\CommentTok{\# Calculate the possible values of the total monthly cost}
\NormalTok{C\_values }\OtherTok{\textless{}{-}}\NormalTok{ fixed\_cost }\SpecialCharTok{+}\NormalTok{ (monthly\_benefit\_per\_participant }\SpecialCharTok{*}\NormalTok{ P\_values)}

\CommentTok{\# Probability Mass Function of the total monthly cost}
\NormalTok{C\_probs }\OtherTok{\textless{}{-}}\NormalTok{ P\_probs}

\CommentTok{\# Create a data frame to display the probability mass function}
\NormalTok{pmf\_C }\OtherTok{\textless{}{-}} \FunctionTok{data.frame}\NormalTok{(}\AttributeTok{Total\_Monthly\_Cost =}\NormalTok{ C\_values, }\AttributeTok{Probability =}\NormalTok{ C\_probs)}

\CommentTok{\# Print the probability mass function}
\FunctionTok{print}\NormalTok{(pmf\_C)}
\end{Highlighting}
\end{Shaded}

\begin{verbatim}
##   Total_Monthly_Cost Probability
## 1            1.6e+07        0.08
## 2            2.1e+07        0.22
## 3            2.6e+07        0.31
## 4            3.1e+07        0.24
## 5            3.6e+07        0.15
\end{verbatim}

\textbf{Explanation}: The probability mass function (PMF) of the total
monthly cost is calculated by using the possible values of P to
determine the corresponding costs. Since the number of participants has
a discrete distribution, the total cost also has a corresponding
probability for each value.

\begin{center}
\begin{tabular}{r|r}
\hline
Total\_Monthly\_Cost & Probability\\
\hline
1.6e+07 & 0.08\\
\hline
2.1e+07 & 0.22\\
\hline
2.6e+07 & 0.31\\
\hline
3.1e+07 & 0.24\\
\hline
3.6e+07 & 0.15\\
\hline
\end{tabular}
\end{center}

\subsection{Answer - Question (2.d)}\label{answer---question-2.d}

\textbf{\emph{Compute the Expected Value of the Total Monthly Cost}}

\begin{Shaded}
\begin{Highlighting}[]
\CommentTok{\# Calculate the expected value of the total monthly cost}
\NormalTok{E\_C }\OtherTok{\textless{}{-}} \FunctionTok{sum}\NormalTok{(C\_values }\SpecialCharTok{*}\NormalTok{ C\_probs)}

\CommentTok{\# Print the expected value of the total monthly cost}
\FunctionTok{print}\NormalTok{(}\FunctionTok{paste0}\NormalTok{(}\StringTok{"Expected value of the total monthly cost: "}\NormalTok{, E\_C))}
\end{Highlighting}
\end{Shaded}

\begin{verbatim}
## [1] "Expected value of the total monthly cost: 26800000"
\end{verbatim}

\textbf{Explanation}: The expected value of the total monthly cost is
calculated by taking the sum of the product of each possible value of
the total cost and its corresponding probability. This gives us the
long-run average cost of the program.

\subsection{Answer - Question (2.e)}\label{answer---question-2.e}

\textbf{\emph{Compute the Cumulative Distribution Function of the Total
Monthly Cost}}

\begin{Shaded}
\begin{Highlighting}[]
\CommentTok{\# Calculate the cumulative distribution function (CDF) of the total monthly cost}
\NormalTok{CDF\_C }\OtherTok{\textless{}{-}} \FunctionTok{cumsum}\NormalTok{(C\_probs)}

\CommentTok{\# Create a data frame to display the CDF}
\NormalTok{cdf\_C }\OtherTok{\textless{}{-}} \FunctionTok{data.frame}\NormalTok{(}\AttributeTok{Total\_Monthly\_Cost =}\NormalTok{ C\_values, }\AttributeTok{Cumulative\_Probability =}\NormalTok{ CDF\_C)}

\CommentTok{\# Print the cumulative distribution function}
\FunctionTok{print}\NormalTok{(cdf\_C)}
\end{Highlighting}
\end{Shaded}

\begin{verbatim}
##   Total_Monthly_Cost Cumulative_Probability
## 1            1.6e+07                   0.08
## 2            2.1e+07                   0.30
## 3            2.6e+07                   0.61
## 4            3.1e+07                   0.85
## 5            3.6e+07                   1.00
\end{verbatim}

\textbf{Explanation}: The cumulative distribution function (CDF) of the
total monthly cost is calculated by taking the cumulative sum of the
probabilities. This helps us determine the probability that the total
monthly cost will be less than or equal to a certain value.

\begin{center}
\begin{tabular}{r|r}
\hline
Total\_Monthly\_Cost & Cumulative\_Probability\\
\hline
1.6e+07 & 0.08\\
\hline
2.1e+07 & 0.30\\
\hline
2.6e+07 & 0.61\\
\hline
3.1e+07 & 0.85\\
\hline
3.6e+07 & 1.00\\
\hline
\end{tabular}
\end{center}

\subsection{Answer - Question (2.f)}\label{answer---question-2.f}

\textbf{\emph{Assess Budget Constraints}}

\begin{Shaded}
\begin{Highlighting}[]
\CommentTok{\# Define the maximum monthly budget}
\NormalTok{max\_budget }\OtherTok{\textless{}{-}} \DecValTok{30000000}

\CommentTok{\# Calculate the probability that the cost exceeds the budget}
\NormalTok{prob\_exceeds\_budget }\OtherTok{\textless{}{-}} \DecValTok{1} \SpecialCharTok{{-}}\NormalTok{ CDF\_C[}\FunctionTok{which}\NormalTok{(C\_values }\SpecialCharTok{\textless{}=}\NormalTok{ max\_budget)][}\FunctionTok{length}\NormalTok{(}\FunctionTok{which}\NormalTok{(C\_values }\SpecialCharTok{\textless{}=}\NormalTok{ max\_budget))]}

\CommentTok{\# Print the probability that the cost exceeds the budget}
\FunctionTok{print}\NormalTok{(}\FunctionTok{paste0}\NormalTok{(}\StringTok{"Probability that the cost exceeds the budget: "}\NormalTok{, }\FunctionTok{round}\NormalTok{(prob\_exceeds\_budget, }\DecValTok{2}\NormalTok{)))}
\end{Highlighting}
\end{Shaded}

\begin{verbatim}
## [1] "Probability that the cost exceeds the budget: 0.39"
\end{verbatim}

\textbf{Explanation}: The probability that the cost exceeds the
allocated monthly budget of \$30,000,000 is calculated by finding the
complement of the CDF up to that value. This tells us the likelihood
that the costs will surpass the budget constraints. We find that the
probability that the maximum budget is exceeded is 39\%.

\end{document}
