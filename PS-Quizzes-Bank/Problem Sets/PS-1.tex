\documentclass{article}
\usepackage{amsmath}
\usepackage{enumitem}
\usepackage{array}

\title{POL 201 / POL 501 - Practice Problem Set Unit I}
\author{}
\date{}

\begin{document}

\maketitle

\section*{Problem Set}

\subsection*{Learning Objectives}

\begin{itemize}
    \item Identify variables as numerical and categorical.
    \item Define associated variables as variables that show some relationship with one another. Categorize the relationship as a positive or negative association. Define variables that are not associated as independent.
    \item Identify the explanatory variable in a pair of variables.
    \item Classify a study as observational or experimental.
    \item Question confounding variables and sources of bias in a given study.
    \item Distinguish between simple random, stratified, cluster, and multistage sampling.
    \item Identify the four principles of experimental design.
    \item Identify if single or double blinding has been used in a study
\end{itemize}

\subsection*{Questions}

\subsubsection*{1. Variable Classification (Multiple Choice)}
Identify the type of variable in the following examples. \emph{Be sure to justify your answer by referencing the specific definitions}.\footnote{In the quiz, you will required to justify your answers in multiple choice questions.}

\begin{enumerate}
    \item The number of votes received by a candidate.
    \begin{itemize}
        \item[(a)] Regular Categorical (Nominal)
        \item[(b)] Ordinal Categorical (Ordinal)
        \item[(c)] Numerical (Discrete)
        \item[(d)] Numerical (Continuous)
    \end{itemize}
    
    \item The political party affiliation of survey respondents.
    \begin{itemize}
        \item[(a)] Regular Categorical (Nominal)
        \item[(b)] Ordinal Categorical (Ordinal)
        \item[(c)] Numerical (Discrete)
        \item[(d)] Numerical (Continuous)
    \end{itemize}

    \item The age of a registered voter
    \begin{itemize}
        \item[(a)] Regular Categorical (Nominal)
        \item[(b)] Ordinal Categorical (Ordinal)
        \item[(c)] Numerical (Discrete)
        \item[(d)] Numerical (Continuous)
    \end{itemize}

    \item The type of electoral system used in different countries (e.g., proportional representation, majoritarian).
    \begin{itemize}
        \item[(a)] Regular Categorical (Nominal)
        \item[(b)] Ordinal Categorical (Ordinal)
        \item[(c)] Numerical (Discrete)
        \item[(d)] Numerical (Continuous)
    \end{itemize}  

    \item The level of agreement with the statement "Democracy is the best form of government" measured on a Likert scale (Strongly Disagree, Disagree, Neutral, Agree, Strongly Agree).
    \begin{itemize}
        \item[(a)] Regular Categorical (Nominal)
        \item[(b)] Ordinal Categorical (Ordinal)
        \item[(c)] Numerical (Discrete)
        \item[(d)] Numerical (Continuous)
    \end{itemize}

    \item The amount of money spent by a political campaign in dollars.
    \begin{itemize}
        \item[(a)] Regular Categorical (Nominal)
        \item[(b)] Ordinal Categorical (Ordinal)
        \item[(c)] Numerical (Discrete)
        \item[(d)] Numerical (Continuous)
    \end{itemize}
    
\end{enumerate}

\subsubsection*{2. Association of Variables (Multiple Choice)}
Read the following hypothetical findings from studies and determine the type of relationship described. \emph{Be sure to justify your answer by referencing the specific definitions}.\footnote{In the quiz, you will required to justify your answers in multiple choice questions.}

\begin{enumerate}
    \item A study found that people with lower educational achievement are less likely to vote in presidential elections. 
    \begin{itemize}
        \item[(a)] Positive relationship
        \item[(b)] Negative relationship
        \item[(c)] No relationship
    \end{itemize}

    \item An analysis finds that Democrats and Republicans are equally likely to engage in volunteer work. 
    \begin{itemize}
        \item[(a)] Positive relationship
        \item[(b)] Negative relationship
        \item[(c)] No relationship
    \end{itemize}
    
    \item Research indicates that candidates who spend more money on their campaigns have a higher chance of winning the election.
    \begin{itemize}
        \item[(a)] Positive relationship
        \item[(b)] Negative relationship
        \item[(c)] No relationship
    \end{itemize}
    
    \item A survey shows that older individuals are less likely to support progressive policies compared to younger individuals.
    \begin{itemize}
        \item[(a)] Positive relationship
        \item[(b)] Negative relationship
        \item[(c)] No relationship
    \end{itemize}
    
    \item Data suggests that supporters of different political parties (e.g., Democrat, Republican) exhibit similar preferences for a particular type of media outlet (e.g., TV, online, newspaper).
    \begin{itemize}
        \item[(a)] Positive relationship
        \item[(b)] Negative relationship
        \item[(c)] No relationship
    \end{itemize}
    
\end{enumerate}


\subsubsection*{3. Identifying Variables (Multiple Choice)}
For each of the following findings, identify the most likely explanatory or causal relationship between the variables. \emph{Be sure to justify your answer by referencing the specific definitions}.\footnote{In the quiz, you will required to justify your answers in multiple choice questions.}

\begin{enumerate}
    \item A study found that increased access to education leads to higher employment rates.
    \begin{itemize}
        \item[(a)] Access to education is the response, and employment rates are the explanatory variable.
        \item[(b)] Access to education is the explanatory variable, and employment rates are the response.
    \end{itemize}

    \item Research shows that longer daily commute times are associated with lower job satisfaction.
    \begin{itemize}
        \item[(a)] Commute time is the response, and job satisfaction is the explanatory variable.
        \item[(b)] Commute time is the explanatory variable, and job satisfaction is the response.
    \end{itemize}

    \item A survey shows that older individuals are less likely to support progressive policies compared to younger individuals.
    \begin{itemize}
        \item[(a)] Age is the response, and policy support is the explanatory variable.
        \item[(b)] Age is the explanatory variable, and policy support is the response.
    \end{itemize}

    \item Data suggests that there is no clear pattern between political party affiliation (e.g., Democrat, Republican) and the preference for a particular type of media outlet (e.g., TV, online, newspaper).
    \begin{itemize}
        \item[(a)] Political party affiliation is the response, and media preference is the explanatory variable.
        \item[(b)] Political party affiliation is the explanatory variable, and media preference is the response.
    \end{itemize}

    \item An analysis finds that individuals who identify with a particular political ideology are not more or less likely to engage in volunteer work than those who do not.
    \begin{itemize}
        \item[(a)] Political ideology is the response, and volunteer work engagement is the explanatory variable.
        \item[(b)] Political ideology is the explanatory variable, and volunteer work engagement is the response.
    \end{itemize}
\end{enumerate}

\subsubsection*{4. Study Classification (Multiple Choice)}
Classify each study as observational or experimental. \emph{Be sure to justify your answer by referencing the specific definitions}. \footnote{In the quiz, you will required to justify your answers in multiple-choice questions.}

\begin{enumerate}
    \item A survey examining the correlation between voting behavior and social media usage.
    \begin{itemize}
    \item [(a)] Observational
    \item [(b)] Experimental
\end{itemize}
    \item An study where participants are randomly assigned to watch either a political debate or a control program, and their attitudes toward the candidates are measured afterward.
    \begin{itemize}
    \item [(a)] Observational
    \item [(b)] Experimental
\end{itemize}
\end{enumerate}


\subsubsection*{5. Sampling Methods (Multiple Choice)}
Match the sampling method to the description. \emph{Be sure to justify your answer by referencing the specific definitions}. \footnote{In the quiz, you will required to justify your answers in multiple-choice questions.}

\begin{enumerate}
    \item Dividing a city into neighborhoods and randomly selecting neighborhoods to survey all residents within them.
    \begin{itemize}
        \item[(a)] Simple Random Sampling
        \item[(b)] Stratified Sampling
        \item[(c)] Cluster Sampling
    \end{itemize}
    
    \item Randomly selecting individuals from each age group to ensure all age groups are represented.
    \begin{itemize}
        \item[(a)] Simple Random Sampling
        \item[(b)] Stratified Sampling
        \item[(c)] Multistage Sampling
    \end{itemize}
\end{enumerate}

\subsubsection*{6. Experimental Design (Open-ended)}
\begin{enumerate}
    \item Describe the four principles of experimental design and explain how each principle helps to ensure the validity of an experiment.
    \par \emph{Provide a detailed justification for your answer below:}
        \begin{center}
        \fbox{\parbox{0.9\textwidth}{\vspace{3cm} \hspace{9cm}}}
        \end{center}
    \item After reading the following study description, use the four principles of experimental design to analyze how strong the study is in satisfying each principle, and determine if no information is mentioned regarding one of the principles: 
    \par Consider a hypothetical political science study that investigates the effect of campaign email strategies on voter turnout. The study selects a small city known for its close election results and politically engaged population. Researchers randomly assign half of the registered voters to receive a weekly email encouraging them to vote, detailing polling station information, and providing transportation options to reach the polling stations. The other half, serving as a comparison group, receives no additional communication. The demographics of the city are diverse, ensuring a broad range of socio-economic backgrounds in the participant pool. The city’s previous voting data is meticulously analyzed to ensure that voters in both groups have historically shown similar voting patterns to avoid bias from past behaviors. 
    \par \emph{Provide a detailed justification for your answer below:}
    \begin{center}
    \fbox{\parbox{0.9\textwidth}{\vspace{3cm} \hspace{9cm}}}
    \end{center}
\end{enumerate}


\subsubsection*{7. Confounding Variables (Short Answer)}

You are reviewing the results of an observational study that surveys a representative random sample of American citizens. The study investigates the relationship between educational achievement and political participation. It reports a positive correlation, showing that higher levels of education are associated with increased political participation. You know that previous research has established that higher education levels correlate positively with greater political knowledge, and that such knowledge predicts increased political participation. However, the study in question does not control for or measure political knowledge.
\par Given the scenario described, explain why the absence of controls for political knowledge could be problematic, using the concept of ``confounding variable." \newline
    \par \emph{Provide a detailed justification for your answer below:}
    \begin{center}
    \fbox{\parbox{0.9\textwidth}{\vspace{3cm} \hspace{9cm}}}
    \end{center}

\subsubsection*{8. Data Interpretation (Data Table and Open-ended)}

Consider the following dataset from a recent political campaign study, which investigates the relationship between the number of visits a candidate makes to various regions during her tenure and the resulting changes in voter turnout in those regions in the next election. The table below presents the data collected:

\begin{center}
\begin{tabular}{|c|c|c|}
\hline
Region & Number of Visits & Increase in Voter Turnout (\%) \\
\hline 
North & 5 & 2 \\
South & 3 & 1 \\
East & 8 & 5 \\
West & 2 & 0.5 \\
\hline
\end{tabular}
\end{center}

\begin{enumerate}
\item Identify and define the explanatory and response variables in this study.
\item Identify the variable type for each of the columns of the data matrix. 
\item Rearrange the data in the table to sort the regions in ascending order based on the \emph{number of visits} by the candidate. Recreate the sorted table below.
\item Analyze the sorted data to determine if there is a positive, negative, or no apparent relationship between the number of candidate visits and the percentage increase in voter turnout. Provide a justification for your assessment using the sorted data.
\end{enumerate}

\par \emph{Provide a detailed justification for your analysis in the space provided below:}
\begin{center}
\fbox{\parbox{0.9\textwidth}{\vspace{3cm} \hspace{9cm}}}
\end{center}

\subsubsection*{9. Blinding (Multiple Choice)}
Identify the type of blinding used in the following scenario:

Participants in a study are randomly assigned to receive either a real or a placebo political advertisement. Neither the participants nor the researchers who interact with them know which type of advertisement they received.

\begin{itemize}
    \item[(a)] Single Blinding
    \item[(b)] Double Blinding
    \item[(c)] No Blinding
\end{itemize}

\section*{Concepts and Terms Covered in Unit I}

\begin{itemize}
    \item Observation (Data point): A single measurement or data point collected from a study or experiment. Each observation represents an individual instance of the variables being measured.
    
    \item Research Question: A question that the research aims to answer, typically framed to investigate the relationship between variables or the effectiveness of different treatments.
    
    \item Treatment Group: A group of subjects in an experiment that receives the treatment whose effect the researchers wish to study.
    
    \item Control Group: A group of subjects in an experiment that does not receive the treatment being tested, used as a baseline to measure the effects of the treatment group.
    
    \item Summary Statistic: A single measure that summarizes or represents the characteristics of a dataset, such as mean, median, standard deviation, etc.
    
    \item Data Matrix: A rectangular array of data where rows represent individual observations and columns represent variables.
    
    \item Variables (columns of the data matrix): Features or characteristics that are measured in a study. Each variable represents a different type of data collected from each observation.
    
    \item Units (rows of the data matrix): The individual entities or subjects from which data are collected in a study, represented as rows in a data matrix.
    
\item Type of Variables
    \begin{itemize}
        \item Categorical
        \begin{itemize}
            \item Regular Categorical (or Nominal): Variables that categorize or label attributes without any order or ranking (e.g., gender, eye color).
            \item Ordered Categorical: Variables that categorize attributes where the categories have a logical order or ranking but the intervals between categories are not necessarily equal (e.g., ratings such as poor, fair, good, excellent).
        \end{itemize}
        \item Numerical
        \begin{itemize}
            \item Numerical Discrete: Numerical values that can only take integer values within a range (e.g., number of countries visited).
            \item Numerical Continuous: Numerical values that can take any value within a range (any real number), allowing for infinitely fine subdivisions (e.g., sleep measured in hours and minutes).
        \end{itemize}
    \end{itemize}
\item Statistical Association or Relationship:
    \begin{itemize}
        \item Independent variables (No association): Variables that do not show any predictive or explanatory relationship between them.
        \item Associated variables:
        \begin{itemize}
            \item Positive association: A relationship where increases (decreases) in one variable are associated with increases (decreases) in another variable (i.e., variables move in the same direction).
            \item Negative association: A relationship where increases (decreases) in one variable are associated with decreases (increases) in another variable (i.e., variables move in the opposite direction).
            \item Non-linear association: A relationship between variables that does not follow a straight line, indicating that the effect of one variable on another changes at different levels of those variables.
        \end{itemize}
    \end{itemize}

\item Causal Relationship: A relationship where one variable directly affects another. That is, keeping every other factor constant, a change in one variable ($X$) \emph{causes} a change in another variable ($Y$). In that case, we say that $X$ \emph{causes} $Y$. This can be established conclusively only through properly controlled experiments.
    \begin{itemize}
        \item Response variable (outcome variable or dependent variable): The variable that we are interested in understanding as an outcome of a process. 
        \item Explanatory variable: The variable that is believed to be a factor directly influencing another variable (the response or outcome variable). In experiments, the explanatory variable is manipulated or changed to determine if it causes a change in the response variable. 
        \item ``Association does not imply causation": A principle stating that just because two variables are associated (correlated), it does not mean that one causes the other to occur.
        \item Confounding variable: An external variable that influences both the explanatory and response variables, potentially misleading the results of a study if not controlled.
    \end{itemize}
\item Types of Studies
    \begin{itemize}
        \item Observational study: A study where the researcher observes and measures characteristics of interest, but does not intervene or manipulate the study environment.
        \item Experimental study: A study where the researcher actively manipulates one or more variables to investigate the effect on other variables, typically to establish causal relationships.
    \end{itemize}
\item Population: The entire set of individuals or objects of interest from which researchers may potentially collect data.
\item Sample:
    \begin{itemize}
        \item Representative Sample: A subset of the population that accurately reflects the members of the entire population.
        \item Convenience Sample: A sample selected from the population based on ease of access and not randomly, which may lead to biased results.
        \item Non-response rate and bias: The bias that occurs when the responses received are not representative of the intended sample, often because a significant portion of sampled individuals does not respond.
    \end{itemize}

\item Sampling Strategies
    \begin{itemize}
        \item Simple Random Sample: A sampling method where each set of n elements has an equal chance of being chosen from the population, ensuring each sample is equally representative.
        \item Stratified Sampling: A sampling method that divides the population into homogeneous subgroups known as strata and then takes a simple random sample from each stratum.
        \item Cluster Sampling: A sampling approach where the population is divided into separate groups, known as clusters, usually along geographical boundaries. A random sample of these clusters is chosen, and all observations within selected clusters are included in the sample.
        \item Multi-Stage Sampling: A complex form of cluster sampling where multiple levels of units are embedded one in another. This method involves performing cluster sampling in stages, selecting smaller and smaller groups at each stage.
    \end{itemize}
\item Randomized experiment:
    \begin{itemize}
        \item Controlling: The method of holding constant external variables that might influence the experiment, to ensure that any observed effects on the dependent variable are due to the independent variable.
        \item Randomization: The process of randomly assigning subjects to different treatment groups in an experiment to ensure that each subject has an equal chance of receiving any treatment, thus eliminating bias.
        \item Replication: The practice of repeating the experiment on many subjects to ensure that the results are valid and not due to random variation or peculiarities of a single sample.
        \item Blocking: The technique of arranging experimental units into groups (blocks) that are similar to one another. This can reduce the variation in response due to extraneous factors, making it easier to detect differences caused by treatments.
    \end{itemize}

\end{itemize}

\end{document}
