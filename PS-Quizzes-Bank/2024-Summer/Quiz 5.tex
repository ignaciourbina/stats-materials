\documentclass{article}
\usepackage{amsmath}
\usepackage{amssymb}
\usepackage{geometry}
\geometry{margin=1in}

\usepackage{xcolor}  % This line includes the xcolor package
\definecolor{dblue}{rgb}{0.252, 0.378, 0.462}  % Reduced each component by about 20%

% ------------------------------------------- %
\title{Quiz 5}
\author{POL 201 - POL 501}
\date{}

\begin{document}
\maketitle

\subsection*{Question 1 (5 points)}
City council members are interested in understanding whether a new public safety campaign has differently affected perceptions of safety in two urban neighborhoods. To investigate this, researchers conducted a study using random samples from both neighborhoods. In Neighborhood A, 25 randomly sampled residents rated their sense of safety on a scale from 1 to 10 after the campaign. In Neighborhood B, 28 randomly sampled residents provided similar ratings. The sample means and standard deviations for each neighborhood are provided below:

\begin{center}
\renewcommand{\arraystretch}{1.5} % Adjusts the row height, 1.5 is 50% more than the original.
\begin{tabular}{c|c|c}
\hline
     & \textbf{Neighborhood A} & \textbf{Neighborhood B} \\ \hline
  $\bar{X}$ (Mean) &  5.2 & 7.6 \\ \hline
  $s$ (Standard Deviation) &  4.3 &  4.5 \\ \hline 
\end{tabular}
\end{center}


The city council wants to determine whether there is a statistically significant difference in the average safety ratings between the two neighborhoods after the campaign.

Based on the provided information, answer the following questions:

\begin{enumerate}
\item[a)] Clearly state the null and alternative hypotheses, both in words and as formal mathematical statements. \emph{Justify your answer.} (1 point)
\begin{center}
\fbox{\parbox{0.93\textwidth}{\vspace{4.0cm} \hspace{1cm}}}
\end{center}

\item[b)] Assuming equal population variances for both samples, identify the appropriate test statistic to use for this hypothesis test and write down its formula. \emph{Justify your answer.} (1 point)
\begin{center}
\fbox{\parbox{0.93\textwidth}{\vspace{4.0cm} \hspace{1cm}}}
\end{center}

\item[c)] Calculate the test statistic and the corresponding P-value for the test. Be very clear on how you are doing the computations and your assumptions. \emph{Justify your answer.} (1 point)
\begin{center}
\fbox{\parbox{0.93\textwidth}{\vspace{4.0cm} \hspace{1cm}}}
\end{center}

\item[d)] Assess whether we can reject the null hypothesis at both the 5\% and 1\% significance levels. \emph{Justify your answer.} (1 point)
\begin{center}
\fbox{\parbox{0.93\textwidth}{\vspace{4.0cm} \hspace{1cm}}}
\end{center}

\item[e)] In simple terms, explain how we should interpret the results of the hypothesis test. \emph{Justify your answer.} (1 point)
\begin{center}
\fbox{\parbox{0.93\textwidth}{\vspace{4.0cm} \hspace{1cm}}}
\end{center}

\end{enumerate}

\newpage
\subsection*{Question 2 (5 points)}
In a study to determine the effectiveness of a new online therapy program designed to reduce anxiety, researchers collected data on participants' anxiety scores before and after completing the program. Participants were evaluated using a standardized anxiety scale where higher scores indicate greater anxiety. The scores for 10 participants before and after the program are presented below:

\begin{center}
\begin{tabular}{c|c|c}
\hline
\textbf{Participant} & \textbf{Before (Score)} & \textbf{After (Score)} \\ \hline
1 & 24 & 18 \\ \hline
2 & 30 & 22 \\ \hline
3 & 28 & 25 \\ \hline
4 & 32 & 30 \\ \hline
5 & 20 & 15 \\ \hline
6 & 22 & 17 \\ \hline
7 & 26 & 21 \\ \hline
8 & 25 & 20 \\ \hline
9 & 27 & 19 \\ \hline
10 & 29 & 23 \\ \hline
\end{tabular}
\end{center}

Researchers want to determine whether there is a statistically significant reduction in anxiety scores after participants completed the therapy program.

Based on the provided information, answer the following questions:

\begin{enumerate}
\item[a)] Clearly state the null and alternative hypotheses, both in words and as formal mathematical statements. \emph{Justify your answer.} (1 point)
\begin{center}
\fbox{\parbox{0.93\textwidth}{\vspace{4.0cm} \hspace{1cm}}}
\end{center}

\item[b)] Identify the appropriate test statistic to use for this hypothesis test and write down its formula. \emph{Justify your answer.} (1 point)
\begin{center}
\fbox{\parbox{0.93\textwidth}{\vspace{4.0cm} \hspace{1cm}}}
\end{center}

\item[c)] Calculate the test statistic and the corresponding P-value for the test. Be very clear on how you are doing the computations and your assumptions. \emph{Justify your answer.} (1 point)
\begin{center}
\fbox{\parbox{0.93\textwidth}{\vspace{4.0cm} \hspace{1cm}}}
\end{center}

\item[d)] Assess whether we can reject the null hypothesis at both the 5\% and 1\% significance levels. \emph{Justify your answer.} (1 point)
\begin{center}
\fbox{\parbox{0.93\textwidth}{\vspace{4.0cm} \hspace{1cm}}}
\end{center}

\item[e)] In simple terms, explain how we should interpret the results of the hypothesis test. \emph{Justify your answer.} (1 point)
\begin{center}
\fbox{\parbox{0.93\textwidth}{\vspace{4.0cm} \hspace{1cm}}}
\end{center}
\end{enumerate}

\newpage
\subsection*{Question 3 (5 points)}
Researchers are studying the effects of two types of advertisements on raising awareness of environmental issues. They conducted a nationwide campaign, randomly assigning participants to one of two groups exposed to different advertisement styles: one using visual imagery (Group A) and the other using statistical facts (Group B). One week after exposure, participants were asked if their awareness of environmental issues had increased. The results from these large samples are as follows:

\begin{center}
\renewcommand{\arraystretch}{1.5} % Adjusts the row height, 1.5 is 50% more than the original.
\begin{tabular}{c|c|c}
\hline
     & \textbf{Group A (Visual)} & \textbf{Group B (Facts)} \\ \hline
  $n_i$ (Total participants) &  1200 & 1150 \\ \hline
  $X_i$ (Number reporting increased awareness) &  860 & 780 \\ \hline
\end{tabular}
\end{center}

The researchers want to determine whether there is a statistically significant difference in the proportion of participants reporting increased awareness between the two advertisement styles.

Based on the provided information, answer the following questions:

\begin{enumerate}
\item[a)] Clearly state the null and alternative hypotheses, both in words and as formal mathematical statements. \emph{Justify your answer.} (1 point)
\begin{center}
\fbox{\parbox{0.93\textwidth}{\vspace{4.0cm} \hspace{1cm}}}
\end{center}

\item[b)] Identify the appropriate test statistic to use for this hypothesis test and write down its formula. \emph{Justify your answer.} (1 point)
\begin{center}
\fbox{\parbox{0.93\textwidth}{\vspace{4.0cm} \hspace{1cm}}}
\end{center}

\item[c)] Calculate the test statistic and the corresponding P-value for the test. Be very clear on how you are doing the computations and what are your assumptions. \emph{Justify your answer.} (1 point)
\begin{center}
\fbox{\parbox{0.93\textwidth}{\vspace{4.0cm} \hspace{1cm}}}
\end{center}

\item[d)] Assess whether we can reject the null hypothesis at both the 5\% and 1\% significance levels. \emph{Justify your answer.} (1 point)
\begin{center}
\fbox{\parbox{0.93\textwidth}{\vspace{4.0cm} \hspace{1cm}}}
\end{center}

\item[e)] In simple terms, explain how we should interpret the results of the hypothesis test. \emph{Justify your answer.} (1 point)
\begin{center}
\fbox{\parbox{0.93\textwidth}{\vspace{4.0cm} \hspace{1cm}}}
\end{center}
\end{enumerate}


\end{document}
