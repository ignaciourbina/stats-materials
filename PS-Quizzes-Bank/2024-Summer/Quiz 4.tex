\documentclass{article}
\usepackage{amsmath}
\usepackage{amssymb}
\usepackage{geometry}
\geometry{margin=1in}

\usepackage{xcolor}  % This line includes the xcolor package
\definecolor{dblue}{rgb}{0.252, 0.378, 0.462}  % Reduced each component by about 20%

% ------------------------------------------- %
\title{Quiz 4}
\author{POL 201 - POL 501}
\date{}

\begin{document}
\maketitle

\subsection*{Question 1 (3 points)}
In a certain city, the response time of ambulances to emergency calls is normally distributed. The mean response time is 12 minutes, with a standard deviation of 3 minutes. Let \(X\) represent the random variable measuring the response time of an ambulance. \newline 
Based on this information, answer the following questions:

\begin{enumerate}
\item[a)] What is the probability that an ambulance will respond within 10 minutes? \emph{Justify your answer.} (1 point)
\begin{center}
\fbox{\parbox{0.93\textwidth}{\vspace{4.0cm} \hspace{1cm}
}}
\end{center}

\item[b)] What is the probability that an ambulance will respond within 15 minutes? \emph{Justify your answer.} (1 point)
\begin{center}
\fbox{\parbox{0.93\textwidth}{\vspace{4.0cm} \hspace{1cm}
}}
\end{center}

\item[c)] What is the probability that an ambulance will respond within 10 to 15 minutes? \emph{Justify your answer.} (1 point)

\begin{center}
\fbox{\parbox{0.93\textwidth}{\vspace{4.0cm} \hspace{1cm}
}}
\end{center}

\end{enumerate}

\subsection*{Question 2 (6 points)}
A study of 400 social media users found that 280 of them believe that social media has a negative impact on their mental health. Suppose we are interested in making inferences about the population proportion of social media users who believe social media has a negative impact on their mental health. 

Based on the provided information, answer the following questions:

\begin{enumerate}
\item[a)] Calculate the sample proportion of social media users who believe that social media has a negative impact on their mental health. \emph{Justify your answer.} (1 point)
\begin{center}
\fbox{\parbox{0.93\textwidth}{\vspace{4.0cm} \hspace{1cm}
}}
\end{center}
    
\item[b)] Calculate the standard error of the sample proportion. \emph{Justify your answer.} (1 point)
\begin{center}
\fbox{\parbox{0.93\textwidth}{\vspace{4.0cm} \hspace{1cm}
}}
\end{center}
    
\item[c)] If we want to construct a confidence interval with 95\% confidence, which critical value of the standard normal distribution should we use? \emph{Justify your answer.} (1 point)
    \begin{center}
    \fbox{\parbox{0.93\textwidth}{\vspace{4.0cm} \hspace{1cm}
    }}
    \end{center}
    
\item[d)] If we want to construct a confidence interval with 99\% confidence, which critical value of the standard normal distribution should we use? \emph{Justify your answer.} (1 point)
\begin{center}
\fbox{\parbox{0.93\textwidth}{\vspace{4.0cm} \hspace{1cm}
}}
\end{center}
    
\item[e)] Using the sample data, construct a confidence interval for the population proportion at the 95\% confidence level. Interpret the confidence interval. \emph{Justify your answer.} (1 point)
\begin{center}
\fbox{\parbox{0.93\textwidth}{\vspace{4.0cm} \hspace{1cm}
}}
\end{center}
    
\item[f)] Using the sample data, construct a confidence interval for the population proportion at the 99\% confidence level. Interpret the confidence interval. \emph{Justify your answer.} (1 point)
\begin{center}
\fbox{\parbox{0.93\textwidth}{\vspace{4.0cm} \hspace{1cm}
}}
\end{center}
    
\end{enumerate}

\newpage
\subsection*{Question 3 (6 points)}

A recent survey suggests that 45\% of residents in a small town support implementing a participatory budgeting process for local government spending. A political scientist believes that the true support for participatory budgeting is actually higher than 45\%. To test this claim, a random sample of 700 residents is surveyed, and 360 of them express support for participatory budgeting.\footnote{Participatory budgeting is a democratic process in which community members directly decide how to allocate a portion of a public budget.} We are asked to carry out a hypothesis test to assess whether there is evidence in support of the political scientist's claim. The test is conducted at the 1\% significance level ($\alpha=0.01$).

Based on the provided information, answer the following questions:

\begin{enumerate}
\item[a)] State the null and alternative hypotheses. \emph{Justify your answer.} (1 point)
\begin{center}
\fbox{\parbox{0.93\textwidth}{\vspace{4.0cm} \hspace{1cm}
}}
\end{center}

\item[b)] Calculate the sample proportion. \emph{Justify your answer.} (1 point)
\begin{center}
\fbox{\parbox{0.93\textwidth}{\vspace{4.0cm} \hspace{1cm}
}}
\end{center}

\item[c)] Compute the standard error of the proportion assuming the null hypothesis. \emph{Justify your answer.} (1 point)
\begin{center}
\fbox{\parbox{0.93\textwidth}{\vspace{4.0cm} \hspace{1cm}
}}
\end{center}
    
\newpage
\item[d)] Calculate the test statistic (Z). \emph{Justify your answer.} (1 point)
\begin{center}
\fbox{\parbox{0.93\textwidth}{\vspace{4.0cm} \hspace{1cm}
}}
\end{center}

\item[e)] Determine the critical value for a one-sided test at the 1\% significance level ($\alpha=0.01$). \emph{Justify your answer.} (1 point)
\begin{center}
\fbox{\parbox{0.93\textwidth}{\vspace{4.0cm} \hspace{1cm}
}}
\end{center}

\item[f)] Compare the value of the test statistic to the critical value and make a conclusion regarding whether the new data supports the political scientist's claim. \emph{Justify your answer.} (1 point)
\begin{center}
\fbox{\parbox{0.93\textwidth}{\vspace{4.0cm} \hspace{1cm}
}}
\end{center}

\end{enumerate}



\end{document}
