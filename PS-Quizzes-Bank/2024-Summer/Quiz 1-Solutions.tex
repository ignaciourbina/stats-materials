\documentclass{article}
\usepackage{amsmath}
\usepackage{geometry}
\geometry{margin=1in}

\usepackage{graphicx}
\usepackage{float}

\begin{document}

Note: From questions 1 to 5, each question is scored with 1 point. Question 6 is scored with 10 points (2.5 points each part/letter).

\section*{Quiz 1 - Units I and II}

\subsection*{Question 1}
A study exploring the effects of political corruption on tax compliance found that in countries with less political corruption, private firms were less likely to engage in tax fraud.

\textbf{What type of relationship is described in the study?}
\begin{enumerate}
    \item[(a)] Positive relationship
    \item[(b)] Negative relationship
    \item[(c)] Non-linear relationship.
    \item[(c)] No relationship
\end{enumerate}

\subsection*{Question 2}
A study found that as the number of community events increases from a few to a moderate number, community cohesion improves. However, when the number of events increases from moderate to high, community cohesion decreases.

\textbf{What kind of association is this?}
\begin{enumerate}
    \item[(a)] Positive association
    \item[(b)] Negative association
    \item[(c)] Non-linear association
    \item[(d)] The variables are independent of each other
\end{enumerate}

\subsection*{Question 3}
A study of one recent primary for the Republican party revealed the following data. Researchers were surprised to observe the results.

\begin{table}[h!]
\centering
\begin{tabular}{|c|c|c|}
    \hline
    \textbf{Candidate Names} & \textbf{Total Votes Won} & \textbf{Campaign Spending (\$)} \\
    \hline
    Candidate A & 10,000 & 500,000 \\
    Candidate B & 15,000 & 300,000 \\
    Candidate C & 8,000 & 700,000 \\
    Candidate D & 12,000 & 400,000 \\
    \hline
\end{tabular}
\end{table}

\textbf{According to this data table, what kind of association is found between "spending" and "votes"?}
\begin{enumerate}
    \item[(a)] Positive association
    \item[(b)] Negative association
    \item[(c)] Non-linear association
    \item[(d)] The variables are independent of each other
\end{enumerate}

\subsection*{Question 4}
Consider the level of satisfaction with local government services as expressed through a survey using ratings: very unsatisfied, unsatisfied, neutral, satisfied, very satisfied. This question assesses the perceived effectiveness of services such as public transportation, parks, and emergency responses.

\textbf{Which kind of variable type is this?}
\begin{enumerate}
    \item[(a)] Regular Categorical (Nominal)
    \item[(b)] Ordinal Categorical (Ordinal)
    \item[(c)] Numerical (Discrete)
    \item[(d)] Numerical (Continuous)
\end{enumerate}

\subsection*{Question 5}
In a psychological study examining stress triggers, participants were categorized by their primary work environment settings, such as 'open-plan offices', 'private offices', and 'remote work from home'. Researchers sought to determine if these settings influenced reported stress levels during work hours.

\textbf{Which is the explanatory variable and which is the response?}
\begin{enumerate}
    \item[(a)] Work environment setting is the response, and stress level is the explanatory variable.
    \item[(b)] Stress level is the response, and work environment setting is the explanatory variable.
\end{enumerate}

\subsection*{Question 6}

A company surveyed different neighborhoods to measure the level of social trust and the number of social activities. Social activities include concerts, social benefit events, sports, and other community gatherings. Social trust is defined as the level of trust residents have in their community and neighbors, measured on a scale from 0 to 100. The results are shown below:

\begin{table}[h!]
\centering
\begin{tabular}{|c|c|c|}
\hline
Neighborhood & Number of Social Activities (per month) & Social Trust Level (out of 100) \\
\hline
N1 & 1.0 & 21.48 \\
N2 & 2.9 & 48.90 \\
N3 & 4.8 & 76.20 \\
N4 & 6.7 & 96.73 \\
N5 & 8.6 & 96.87 \\
N6 & 10.5 & 98.58 \\
N7 & 12.4 & 102.14 \\
N8 & 14.3 & 85.35 \\
N9 & 16.2 & 59.21 \\
N10 & 18.1 & 37.10 \\
N11 & 20.0 & -2.32 \\
\hline
\end{tabular}
\caption{Survey Data from Various Neighborhoods}
\end{table}

\subsection*{Parts of Question 6}

\begin{enumerate}
    \item[a)] Calculate the mean, median, and standard deviation for both the number of social activities per month and the social trust levels across the neighborhoods. Explain what these measures indicate about the data distribution of each variable.
    \item[b)] Create a histogram for both the number of social activities per month and social trust levels with five bins of equal size. Describe the shape of each distribution.
    \item[c)] Plot a scatter plot between the number of social activities and social trust levels. Describe any visible patterns that you find.
    \item[d)] Based on your calculations and plots in parts (a) to (c), hypothesize about the relationship between social activities and social trust. According to these data, what is the relationship between social activities and social trust? Can you think of an intuitive explanation of what could be the underlying process that explains this relationship?
\end{enumerate}

%%%====================
\newpage

\section*{Answers Quiz 1}

\section*{Question 1}
A study exploring the effects of political corruption on tax compliance found that in countries with less
political corruption, private firms were less likely to engage in tax fraud.

\textbf{What type of relationship is described in the study?}

(a) Positive relationship

(b) Negative relationship

(c) Non-linear relationship

(d) No relationship

\textbf{Answer:} (a) Positive relationship

\textbf{Justification:} The study states that in countries with less political corruption, private firms are less likely to engage in tax fraud. This indicates that as corruption decreases, tax fraud decreases, showing a positive relationship because both variables change in the same direction.

\section*{Question 2}
A study found that as the number of community events increases from a few to a moderate number, community cohesion improves. However, when the number of events increases from moderate to high, community cohesion decreases.

\textbf{What kind of association is this?}

(a) Positive association

(b) Negative association

(c) Non-linear association

(d) The variables are independent of each other

\textbf{Answer:} (c) Non-linear association

\textbf{Justification:} The study found that community cohesion improves with an increase in community events up to a moderate number, but then decreases as the number of events becomes high. This indicates a non-linear association, where the relationship between community events and cohesion changes direction at different levels.

\section*{Question 3}
A study of one recent primary for the Republican party revealed the following data. Researchers were surprised to observe the results.

\begin{tabular}{|l|c|c|}
\hline
Candidate Names & Total Votes Won & Campaign Spending (\$) \\
\hline
Candidate A & 10,000 & 500,000 \\
Candidate B & 15,000 & 300,000 \\
Candidate C & 8,000 & 700,000 \\
Candidate D & 12,000 & 400,000 \\
\hline
\end{tabular}

\textbf{According to this data table, what kind of association is found between "spending" and "votes"?}

(a) Positive association

(b) Negative association

(c) Non-linear association

(d) The variables are independent of each other

\textbf{Answer:} (b) Negative association

\textbf{Justification:} By examining the data table:

\begin{itemize}
  \item Candidate A: 10,000 votes, \$500,000 spending
  \item Candidate B: 15,000 votes, \$300,000 spending
  \item Candidate C: 8,000 votes, \$700,000 spending
  \item Candidate D: 12,000 votes, \$400,000 spending
\end{itemize}

Sorting the candidates by spending shows that higher spending generally corresponds to fewer votes:
\begin{itemize}
  \item Candidate B: \$300,000, 15,000 votes
  \item Candidate D: \$400,000, 12,000 votes
  \item Candidate A: \$500,000, 10,000 votes
  \item Candidate C: \$700,000, 8,000 votes
\end{itemize}

This pattern suggests a negative association where higher spending is associated with fewer votes. Therefore, there is a negative relationship between the two variables.

\section*{Question 4}
Consider the level of satisfaction with local government services as expressed through a survey using ratings: very unsatisfied, unsatisfied, neutral, satisfied, very satisfied. This question assesses the perceived effectiveness of services such as public transportation, parks, and emergency responses.

\textbf{Which kind of variable type is this?}

(a) Regular Categorical (Nominal)

(b) Ordinal Categorical (Ordinal)

(c) Numerical (Discrete)

(d) Numerical (Continuous)

\textbf{Answer:} (b) Ordinal Categorical (Ordinal)

\textbf{Justification:} The levels of satisfaction (very unsatisfied, unsatisfied, neutral, satisfied, very satisfied) are ordered categories. This makes the variable ordinal since it has a meaningful order but the intervals between levels are not necessarily equal.

\section*{Question 5}
In a psychological study examining stress triggers, participants were categorized by their primary work environment settings, such as 'open-plan offices', 'private offices', and 'remote work from home'. Researchers sought to determine if these settings influenced reported stress levels during work hours.

\textbf{Which is the explanatory variable and which is the response?}

(a) Work environment setting is the response, and stress level is the explanatory variable.

(b) Stress level is the response, and work environment setting is the explanatory variable.

\textbf{Answer:} (b) Stress level is the response, and work environment setting is the explanatory variable.

\textbf{Justification:} The study examines how different work environment settings (open-plan offices, private offices, remote work from home) influence the reported stress levels of participants. Therefore, the work environment setting is the explanatory variable (independent variable), and the stress level is the response variable (dependent variable).

\section*{Answer to Question 6}

\subsection*{Part (a)}

\textbf{Mean, Median, and Standard Deviation}

1. We will calculate the mean, median, and standard deviation for both the number of social activities per month and the social trust levels across the neighborhoods.

2. The algebraic formulas are:
\[
\text{Mean} (\mu) = \frac{1}{N} \sum_{i=1}^{N} x_i
\]
\[
\text{Median: Arrange the data in ascending order and find the middle value.}
\]
\[
\text{Standard Deviation} (\sigma) = \sqrt{\frac{1}{N} \sum_{i=1}^{N} (x_i - \mu)^2}
\]

3. Using the data:
\begin{itemize}
    \item For Social Activities:
    \[
    \mu_{\text{activities}} = \frac{1}{11} (1.0 + 2.9 + 4.8 + 6.7 + 8.6 + 10.5 + 12.4 + 14.3 + 16.2 + 18.1 + 20.0)
    \]
    
    \item For Social Trust:
    \[
    \mu_{\text{trust}} = \frac{1}{11} (21.48 + 48.90 + 76.20 + 96.73 + 96.87 + 98.58 + 102.14 + 85.35 + 59.21 + 37.10 - 2.32)
    \]
    
    \item For Social Activities:
    \[
    \sigma_{\text{activities}} = \sqrt{\frac{1}{11} \left[(1.0 - \mu_{\text{activities}})^2 + (2.9 - \mu_{\text{activities}})^2 + \ldots + (20.0 - \mu_{\text{activities}})^2\right]}
    \]
    
    \item For Social Trust:
    \[
    \sigma_{\text{trust}} = \sqrt{\frac{1}{11} \left[(21.48 - \mu_{\text{trust}})^2 + (48.90 - \mu_{\text{trust}})^2 + \ldots + (-2.32 - \mu_{\text{trust}})^2\right]}
    \]
\end{itemize}

5. Final Results:
\begin{itemize}
    \item Number of Social Activities (per month):
    \begin{itemize}
        \item Mean: 10.5
        \item Median: 10.5
        \item Standard Deviation: 6.01
    \end{itemize}
    \item Social Trust Level (out of 100):
    \begin{itemize}
        \item Mean: 65.48
        \item Median: 76.2
        \item Standard Deviation: 33.80
    \end{itemize}
\end{itemize}

\textbf{Interpretation:}
\begin{itemize}
    \item Number of Social Activities:
    \begin{itemize}
        \item The mean and median being equal (10.5) indicates a symmetric distribution.
        \item The standard deviation of 6.01 suggests moderate variability around the mean.
    \end{itemize}
    \item Social Trust Level:
    \begin{itemize}
        \item The mean (65.48) and median (76.2) indicate a skewed distribution, likely left-skewed because the mean is less than the median.
        \item The high standard deviation (33.80) indicates high variability in social trust levels among the neighborhoods.
    \end{itemize}
\end{itemize}

\subsection*{Part (b)}

\textbf{Histogram}

We will create a histogram for both the number of social activities per month and the social trust levels with five bins of equal size.
We will divide the data range into five equal-sized bins for both variables. We count how many observations fall unto each bin. We use these frequency counts as the values for the y-axis.

\begin{figure}[H]
    \centering
    \includegraphics[width=\textwidth]{Materials-POL201/Fig-Q1/Hist.png}
    \caption{Histograms }
    \label{fig:fig1}
\end{figure}

\textbf{Final Results:}
\begin{itemize}
    \item Number of Social Activities:
    \begin{itemize}
        \item The histogram shows a roughly symmetric distribution.
        \item The data is concentrated around the mean with frequencies decreasing towards the extremes.
    \end{itemize}
    \item Social Trust Levels:
    \begin{itemize}
        \item The histogram shows a skewed distribution, likely left-skewed.
        \item There is a high concentration of data points at higher trust levels, with a tail extending towards lower trust levels.
    \end{itemize}
\end{itemize}

\subsection*{Part (c)}

\textbf{Scatter Plot}

\begin{itemize}
    \item We will create a scatter plot between the number of social activities and social trust levels.
    \item We will plot each pair $(x_i, y_i)$ where $x_i$ is the number of social activities and $y_i$ is the social trust level.
\end{itemize}

\begin{figure}[H]
    \centering
    \includegraphics[width=\textwidth]{Materials-POL201/Fig-Q1/Scatter_Plot.png}
    \caption{Scatter Plot}
    \label{fig:fig1}
\end{figure}

\textbf{Interpretation:}
\begin{itemize}
    \item The scatter plot shows a clear pattern where the social trust level increases with the number of social activities up to a point, and then it appears to decrease.
    \item This suggests a non-linear relationship between the number of social activities and social trust levels.
\end{itemize}

\subsection*{Part (d)}


\textbf{Hypothesis:} There is a non-linear relationship between the number of social activities and social trust levels. Initially, more social activities correlate with higher social trust, but after reaching a peak, further increases in social activities are associated with a decrease in social trust. 
\begin{itemize}
    \item Initially, increased social activities may help build community bonds and trust as more interactions occur.
    \item However, beyond a certain threshold, too many social activities could become overwhelming, leading to stress or a feeling of obligatory participation, which might reduce overall trust levels.
    \item This indicates the importance of balancing the number of social activities to maintain optimal levels of social trust within communities.
\end{itemize}

\end{document}

