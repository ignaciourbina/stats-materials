\documentclass{article}
\usepackage{amsmath}
\usepackage{amssymb}
\usepackage{geometry}
\geometry{margin=1in}

\usepackage{xcolor}  % This line includes the xcolor package
\definecolor{dblue}{rgb}{0.252, 0.378, 0.462}  % Reduced each component by about 20%

%--------------------------------------------------------

\title{Quiz 3 - Solutions}
\author{POL 201 - POL 501}
\date{}

\begin{document}
\maketitle

As a campaign organizer, you are preparing to host a political event. The event is free for attendees, but the owners of the venue will charge you a variable cost of \$70 per attendee. Additionally, there is a fixed cost for the venue rental of \$20,000 you have to pay regardless of how many people attend the event. The number of attendees is represented by a random variable $A$. Below is the distribution for the number of attendees:

\begin{table}[h!]
\centering
\begin{tabular}{c|c}
    \hline
    Attendees ($A$) & Probability ($P(A = a)$) \\
    \hline
    $A < 50$ & 0 \\
    50 & 0.10 \\
    100 & 0.20 \\
    150 & 0.30 \\
    200 & 0.25 \\
    250 & 0.15 \\
    $A > 250$ & 0 \\
    \hline
\end{tabular}
\caption{Probability Mass Function of Attendees.}
\end{table}
\begin{itemize}
    \item \textbf{Fixed Cost:} The venue rental fee is \$20,000.
\end{itemize}

\subsection*{Questions}

\subsubsection*{(a) Calculate the Expected Value of the Number of Attendees (3 points)}

Determine the expected value of the number of attendees at the event. \emph{Justify your answer}.

\begin{center}
\fbox{\parbox{0.98\textwidth}{\textcolor{dblue}{
The expected value of the number of attendees at the event is calculated by summing  the products of each number of attendees and its respective probability:
    \begin{align*}
    E[A] &= 50 \times 0.10 + 100 \times 0.20 + 150 \times 0.30 + 200 \times 0.25 + 250 \times 0.15 \\
    &= 5 + 20 + 45 + 50 + 37.5 \\
    &= 157.5 \text{ attendees}
    \end{align*} }
}}
\end{center}

\subsubsection*{(b) Define the Total Cost as a Linear Combination (1 points)}

To organize the event, the venue is charging you a cost per attendee of \$70, which will be totaled when the event finishes. Write down the total cost of the event, $C$, as a linear combination of $A$, the cost per attendee, and the venue rental fee. \emph{Justify your answer}.

\begin{center}
    \fbox{\parbox{0.98\textwidth}{\textcolor{dblue}{
    The total cost $C$ will be a sum of the variable cost (the product of the number of attendees and the cost per attendee) and the fixed venue rental fee. Thus we have:
    $$C = 70A + 20000$$}
    }}
\end{center}

\subsubsection*{(c) Compute the Probability Mass Function of the Total Cost (3 points)}

Compute the probability mass function of the total cost, $C$. In other words, for each possible value of the total cost indicate its probability, $P_C(c)$. Express this in a table. \emph{Justify your answer}.

\begin{center}
\fbox{\parbox{0.98\textwidth}{\textcolor{dblue}{
As defined above, $C = 70A + 20000$. Because each value of $A$ is mapped to only one value of $C$ we can use the probabilities for the $PMF_A(a)$. So for each value of $A$, we can calculate $C$, and use the given probabilities: \newline}

\centering
\textcolor{dblue}{
\begin{tabular}{c|c|c}
    \hline
    $A$ & $C$ & $P_C(c)$ \\
    \hline
    50 & \$23,500 & 0.10 \\
    100 & \$27,000 & 0.20 \\
    150 & \$30,500 & 0.30 \\
    200 & \$34,000 & 0.25 \\
    250 & \$37,500 & 0.15 \\
    \hline
\end{tabular}  \vspace{1em}}
}}
\end{center}


\subsubsection*{(d) Compute the Cumulative Distribution Function of the Total Cost (2 points)}

Compute the cumulative distribution function of the total cost, $C$. In other words, for each possible value of the total cost calculate $P_C(C\leq c)$. Express this in a table. \emph{Justify your answer}.

\begin{center}
    \fbox{\parbox{0.98\textwidth}{\textcolor{dblue}{
    The cumulative distribution function adds up the probabilities for all the costs up to and including each cost value: \newline}

        \centering
    \textcolor{dblue}{
    \begin{tabular}{c|l}
    \hline
    $c$ & $P_C(C\leq c)$ \\
    \hline
    \$23,500 & 0.10 \\
    \$27,000 & 0.30 = 0.10 + 0.20 \\
    \$30,500 & 0.60 = 0.10 + 0.20 + 0.30 \\
    \$34,000 & 0.85 = 0.10 + 0.20 + 0.30 + 0.25 \\
    \$37,500 & 1.00 = 0.10 + 0.20 + 0.30 + 0.25 + 0.15 \\
    \hline
    \end{tabular}  \vspace{1em}}
}}
\end{center}

\subsubsection*{(e) Compute the Expected Value of the Total Cost (3 points)}

Calculate the expected value of the total cost of the event. \emph{Justify your answer}.

\begin{center}
\fbox{\parbox{0.98\textwidth}{\textcolor{dblue}{
    The expected value of the total cost of the event, denoted as \( E(C) \), is calculated by summing the products of each possible total cost and its corresponding probability:
    \[
    E(C) = \$23,500 \times 0.10 + \$27,000 \times 0.20 + \$30,500 \times 0.30 + \$34,000 \times 0.25 + \$37,500 \times 0.15
    \]
    Calculating the above expression:
    \[
    E(C) = \$2,350 + \$5,400 + \$9,150 + \$8,500 + \$5,625 = \$31,025
    \]
    Thus, the expected total cost of the event is \$31,025. \vspace{1em} \newline
    Alternatively, we could have used the following rule for expectations of linear combinations: $E[aX+b] = aE[X] + b$. In our case, $E[C] = 70 E[A] + 20,000 =  70 \cdot 157.5 + 20,000 =  \$31,025$}
}}
\end{center}

\newpage
\subsubsection*{(f) Assess Budget Constraints (2 points)}

Your supervisor has indicated that the maximum budget for the event is \$35,000. Calculate the probability that the cost of the event exceeds this budget. \emph{Justify your answer}.

\begin{center}
\fbox{\parbox{0.98\textwidth}{\textcolor{dblue}{
    Looking at the cumulative distribution function, we see that $P_C(C\leq \$35,000) = 0.85$. We can determine this since below $C=\$37,500$ and up to $\$34,000$ a probability of $85\%$ is accumulated given the $CDF_C(c)$.  Thus, the probability that the cost of the event exceeds this budget is $$P_C(C > \$35,000) = 1 - P_C(C\leq \$35,000) = 1 - 0.85 = 0.15.$$ So, the probability that the event's cost surpasses the budget is 15\%. \newline An alternative way to think about this question is that the only value of $C$ that exceeds $\$35,000$ is $C=\$37,500$. Therefore the event ``$C > \$35,000$'' is equivalent to the event ``$C=\$37,500$''. Hence, $$P(C > \$35,000) = P(C=\$37,500) = 0.15$$}
}}
\end{center}

\subsubsection*{(g) Determine the Deviation Range (1 point)}

Suppose you communicate to your supervisor the value of the expected cost of the event. She wonders whether the typical deviation from that cost is such that \$35,000 is outside or inside of that typical deviation range.

To answer this question, calculate the standard deviation of the total cost and find the range defined by the expected value plus and minus one standard deviation ($\mu + \sigma$ and $\mu - \sigma$). Determine whether \$35,000 is inside this interval. \emph{Justify your answer}.

\begin{center}
    \fbox{\parbox{0.98\textwidth}{\textcolor{dblue}{
    The standard deviation of the total cost is calculated using the formula for standard deviation of a probability distribution:
    $$\sigma = \sqrt{\sum_{\forall i} [C_i - E(C)]^2 \cdot P_C(C_i)}$$
    We apply this formula to our values:
    \begin{align*}
    \sigma &=\sqrt{(23,500-31,025)^2 \cdot 0.10 + (27,000-31,025)^2 \cdot 0.20 + (30,500-31,025)^2 \cdot 0.30 + \cdots } \\
    & \quad \quad \overline{\cdots + (34,000-31,025)^2 \cdot 0.25 + (37,500 - 31,025)^2 \cdot 0.15} \\
    &= \$4,181.73
    \end{align*}
    Thus, the range defined by one standard deviation from the mean is \$31,025 - \$4,181.73 to \$31,025 + \$4,181.73, or \$26,843.27 to \$35,206.72. \$35,000 is \textbf{inside} this interval.}
    }}
\end{center}

\end{document}
