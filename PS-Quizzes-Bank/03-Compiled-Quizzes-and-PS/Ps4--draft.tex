\documentclass[a4paper, 11pt]{article}

\usepackage[utf8]{inputenc}
\usepackage{array}
\usepackage{booktabs}   % For improved table formatting
\usepackage{tabulary}   % For tables with adjustable column widths
\usepackage{caption}
\usepackage{amsmath}
\usepackage{amssymb}
\usepackage{tabularx}
\usepackage{xcolor}
\usepackage{soul}       % For highlighting text
\usepackage{longtable}
\usepackage{adjustbox}
\usepackage[top=1in, bottom=1in, left=1in, right=1in]{geometry}
\usepackage{lscape}
\usepackage{xurl}
\usepackage{pgfplots}   % For creating graphs and plots
\pgfplotsset{compat=1.17} % Compatibility setting for pgfplots
\usepackage{graphicx}   % For including graphics
\usepackage{float}      % Helps to place figures and tables at precise locations
\usepackage{subcaption} % For subfigures


\let\oldemptyset\emptyset
\let\emptyset\varnothing

\title{Problem Set 4}
\author{POL 501}
\date{November 22, 2024}

\begin{document}
\maketitle

\section*{Instructions}
\subsection*{Contents Covered}
\begin{itemize}
    \item \emph{From the textbook}: Chapters 7 and 8.
    \item \emph{R Lessons}: All Previous Materials.
    \item \emph{R Tutorial}: See \emph{Examining the Relationship between Age and Perceptions of Crime Safety} posted in Brightspace.
\end{itemize}
\subsection*{General Instructions}
\begin{itemize}
    \item \emph{Justify all of your answers}. Failing to do so will have score deductions.
    \item If you use formulas, always write them down and justify their appropriateness in each case.
    %\item \textbf{Unless otherwise instructed}, you can use software to compute your answers (that is, to come up with the accurate calculation of a given arithmetic operation).
    \item You \textbf{must clearly describe how you arrived at each result by stating the appropriate formula}. Correct ``numbers'' in answers, but without justification or computed with the wrong formula, will have score deductions.
    \item Your code justifies your calculations but does not replace explanations and justifications.
\end{itemize}
\subsection*{Submission Instructions}
\begin{itemize}
    \item You must submit your answers to this problem set as a PDF or Word file \textbf{generated by R markdown}.
    \item You must use the R Markdown template I uploaded in Brightspace to write your answers.
    \item \hl{You must upload your RMD file} (the code used to generate the document).
    \item \textbf{Due Date}: \textcolor{blue}{\textbf{December 3rd (Tuesday)}}.
\end{itemize}

%%%%%%%%%%%%%%%%%%%%%%%%%%%%%%%%%%%%%%%%%%%%%%%%%%%%%%%%%%

\newpage
\section*{Context}


This problem set is inspired by the recent research article \emph{``Women’s Empowerment and Child Mortality''} by Kellard et al. (2024), published in \emph{World Development}. \footnote{Kellard, N. M., Makhlouf, Y., Sarkisyan, A., \& Vinogradov, D. V. (2024). Women’s empowerment and child mortality. \emph{World Development, 183}, 106712. \url{https://doi.org/10.1016/j.worlddev.2024.106712}} The paper investigates how various dimensions of women’s empowerment—spanning civil, political, and economic rights—affect child mortality rates in different socioeconomic contexts. Using data from 134 countries over nearly seven decades, the study employs advanced regression techniques to uncover nuanced relationships, highlighting the critical role of institutions and socioeconomic conditions in translating women’s rights into child health improvements.

In this problem set, we will aim to replicate, \emph{in spirit}, some of the paper’s key findings using simpler inferential methods. Specifically, we will use hypothesis testing and basic linear regression to explore how women’s political representation and leadership are associated with child health outcomes. While the paper employs advanced panel data and instrumental variable techniques, the questions in the problem set focus on basic statistical inference methods.

This problem set uses the dataset \texttt{vdem\_wdi\_merged\_df.csv}, which combines data from the Varieties of Democracy (VDEM) and World Development Indicators (WDI) datasets. The dataset explores relationships between political representation and socioeconomic outcomes, focusing on female political leadership and its association with development metrics such as infant mortality rates.


\section*{Codebook Table}
Codebook for Variables in \texttt{vdem\_wdi\_merged\_df.csv}
\scriptsize

{ \centering
\setlength{\LTleft}{-35pt}
\setlength{\LTright}{0pt}
\begin{longtable}{|>{\raggedright\arraybackslash}p{6cm}|>{\raggedright\arraybackslash}p{9cm}|>{\raggedright\arraybackslash}p{2cm}|}
\hline
\textbf{Variable Name} & \textbf{Description} & \textbf{Data Type}  \\
\hline
\texttt{female\_legislators\_percent} & The percentage of seats in the national legislature held by female legislators in 2014, representing gender representation in parliament. & Numeric  \\
\hline
\texttt{avg\_female\_HoS\_or\_HoG} & The average proportion of years (1994–2014) in which a country had a female Head of State or Head of Government. Values range from 0 (never) to 1 (always). & Numeric  \\
\hline
\texttt{one\_or\_more\_female\_HoS\_or\_HoG\_1994to2014} &
A binary variable indicating whether a country had at least one female Head of State (HoS) or Head of Government (HoG) during 1994--2014. (\texttt{1} = Yes, \texttt{0} = No). &
Numeric \\
\hline
\texttt{infant\_mortality\_rate\_mean2015\_2018} & The average infant mortality rate (deaths under one year per 1,000 live births) for 2015–2018. Lower values indicate better infant health outcomes. & Numeric  \\
\hline
\texttt{iso3c} & The ISO 3166-1 alpha-3 code representing the country. This three-letter abbreviation uniquely identifies each country. & Text String  \\
\hline
\texttt{political\_geographic\_region} & A numeric code representing the political-geographic region to which a country belongs (e.g., 2 = Latin America, 5 = Western countries). & Numeric  \\
\hline
\texttt{year} & The reference year for which the VDEM data is anchored. In this dataset, all values are 2014. & Numeric  \\
\hline
\texttt{adolescent\_fert\_rate\_mean2015\_2018} & The average adolescent fertility rate (births per 1,000 women aged 15–19) for 2015–2018, indicating trends in adolescent health outcomes. & Numeric  \\
\hline
\texttt{life\_expect\_birth\_mean2015\_2018} & The average life expectancy at birth (in years) for 2015–2018, reflecting overall health and development. & Numeric  \\
\hline

\end{longtable}
}

\normalsize

\subsection*{Datasets Used}
\textbf{VDEM Dataset:} Provides detailed political and democratic indicators, focusing on governance and representation, covering years 1994–2014. \\
\textbf{WDI Dataset:} Captures socioeconomic indicators such as health and fertility rates, covering years 2015–2018. Both datasets provide insights into the interplay between political conditions and development outcomes.

\subsection*{Pre-Processing Steps}
\begin{itemize}
    \item Filtered VDEM data for 1994–2014 and WDI data for 2015–2018.
    \item Merged datasets to create a cross-sectional dataset anchored to 2014.
    \item Created indicators for female leadership and calculated averages for socioeconomic metrics.
\end{itemize}

\rule{\linewidth}{0.4pt}

%%%%%%%%%%%%%%%%%%%%%%%%%%%%%%%%%%%%%%%%%%%%%%%%%
\section*{Question 1: Understanding the Dataset (2 points)}

\subsection*{1a. Unit of Observation}

What is the unit of observation in this dataset?

\subsection*{1b. Variables}

Describe each variable's type (use the classification discussed in Unit I).

\subsection*{1c. Number of Observations}

How many observations are present in the dataset?


\subsection*{1d. Summary Statistics}

For all continuous and binary variables, create a table showing their mean, minimum, maximum, and standard deviation. Describe your findings.


%%%%%%%%%%%%%%%%%%%%%%%%%%%%%%%%%%%%%%%%%%%%%%%%%
\section*{Question 2: Hypothesis Testing (5 points)}

In this question, you will examine whether countries that had at least one female head of state or government between 1994 and 2014 experienced lower infant mortality rates during the 2015–2018 period compared to those that did not.

\subsection*{2a. Relevant Variables}

Identify the variables necessary to test this hypothesis. \textit{Explain your answer}.

\subsection*{2b. Hypothesis Formulation}

Write the null hypothesis ($H_0$) and the alternative hypothesis ($H_a$) in mathematical form. Clearly define any parameters or terms used in the hypotheses. Indicate whether this is a one-sided or two-sided test. \textit{Explain your answer}.

\subsection*{2c. Test Statistic}

Assume we cannot rely on the central limit theorem due to a limited sample size. In this case, which test statistic should we use to test the null hypothesis? Explain your reasoning and write the formula for the test statistic. \textit{Explain your answer}.

\subsection*{2d. Compute the Test Statistic}

Using the \texttt{dplyr} function \texttt{filter()}, and the R functions \texttt{mean()}, and \texttt{nrow()}, compute the value of the test statistic. \textit{Explain your answer}.

\subsection*{2e. Compute the P-Value and Reach a Conclusion}
\begin{itemize}
    \item Calculate the p-value of the test statistic. Explain what the p-value is capturing and how should we interpret it. (\textit{Hint}: To compute the exact probability, refer to the \emph{appropiate} function from the attached tutorial)
    \item At the 5\% significance level, can we reject the null hypothesis? Explain the conclusion of the hypothesis test.
\end{itemize}

%%%%%%%%%%%%%%%%%%%%%%%%%%%%%%%%%%%%%%%%%%%%%%%%%
\section*{Question 3: Linear Regression (4 points)}

In this question, you will explore the linear relationship between the average infant mortality rate from 2015 to 2018 (dependent variable) and the percentage of female legislators in 2014 (independent variable).

\subsection*{3a. Hypothesis Formulation}

Suppose we expect a positive relationship between the two variables. Write the corresponding hypothesis test for the population slope ($\beta_1$) in mathematical form. Indicate whether this is a one-sided or two-sided test. \textit{Explain your answer}.

\subsection*{3b. Regression Estimation}

Using the command \texttt{lm(...)}, estimate the simple linear regression model that computes the estimate for $\beta_1$. \emph{Interpret} the estimated slope coefficient. How does the dependent variable change with a one percentage point increase in female legislators? \textit{Explain your answer}.

\subsection*{3c. Test Statistic and P-Value}

From the regression output, identify the value of the test statistic and p-value for the estimated slope ($\hat{\beta}_1$). Explain how to interpret these values.

\subsection*{3d. Statistical Significance}

At the 5\% significance level, can we reject the null hypothesis? Explain the conclusion of the hypothesis test.

\end{document}
