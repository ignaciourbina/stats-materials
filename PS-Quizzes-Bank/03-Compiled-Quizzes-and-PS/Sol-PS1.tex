
\documentclass{article}
\usepackage[utf8]{inputenc}
\usepackage{amsmath}
\usepackage{amssymb}

\title{Solutions to Problem Set 1}
\author{}
\date{}

\begin{document}

\maketitle

\section*{Question 1: Variable Classification}
\begin{itemize}
    \item For ``The number of votes received by a candidate,'' the correct answer is \textbf{(c) Numerical (Discrete)}. This is because the number of votes can only be counted in whole numbers, fitting the definition of a discrete numerical variable.
    \item For ``The political party affiliation of survey respondents,'' the correct answer is \textbf{(a) Regular Categorical (Nominal)}. Political party affiliation is a label without any inherent numerical value or order, fitting the definition of a nominal categorical variable.
    \item For ``The age of a registered voter,'' the correct answer is \textbf{(d) Numerical (Continuous)}. Age can vary continuously and can be measured at a very precise scale (even fractions of a year), fitting the definition of a continuous numerical variable.
    \item For ``The type of electoral system used in different countries,'' the correct answer is \textbf{(a) Regular Categorical (Nominal)}. Electoral systems are named categories without any quantitative value or inherent order, fitting the definition of a nominal categorical variable.
    \item For "The level of agreement with the statement 'Democracy is the best form of government' measured on a Likert scale (Strongly Disagree, Disagree, Neutral, Agree, Strongly Agree)," the correct answer is \textbf{(b) Ordinal Categorical (Ordinal)}. The Likert scale provides a ranking order, but the intervals between the options are not necessarily equal, fitting the definition of an ordinal categorical variable.
    \item For "The amount of money spent by a political campaign in dollars," the correct answer is \textbf{(d) Numerical (Continuous)}. The amount of money spent can take on any value within a range and can be measured to a very precise level, fitting the definition of a continuous numerical variable
\end{itemize}

\section*{Question 2: Association of Variables}
\begin{itemize}
    \item A study indicating that ``people with lower educational achievement are less likely to vote in presidential elections'' describes a \textbf{(b) Positive relationship}. Higher levels of one variable (education) are associated with higher levels of another variable (likelihood of voting). In other words, both variables jointly move in the same direction.
    \item An analysis showing that ``Democrats and Republicans are equally likely to engage in volunteer work'' suggests \textbf{(c) No relationship}. There's no correlation shown between political affiliation and the likelihood of volunteering.
    \item Research suggesting that ``candidates who spend more money on their campaigns have a higher chance of winning the election'' indicates a \textbf{(a) Positive relationship}. Increases in campaign spending are associated with increases in the probability of winning.
    \item A survey indicating that ``older individuals are less likely to support progressive policies compared to younger individuals'' points to a \textbf{(b) Negative relationship}. An increase in age is associated with a decrease in support for progressive policies.
    \item For the statement "Data suggests that supporters of different political parties (e.g., Democrat, Republican) exhibit similar preferences for a particular type of media outlet (e.g., TV, online, newspaper)," the correct answer is \textbf{(c) No relationship}. If supporters of different political parties exhibit similar preferences for media outlets, there is no association between political party affiliation and media outlet preference.
\end{itemize}

\section*{Question 3: Identifying Variables}
\begin{itemize}
    \item For the finding that "increased access to education leads to higher employment rates," the correct choice is (b) Access to education is the explanatory variable, and employment rates are the response. Education influences employment rates. This indicates a causal relationship where changes in education levels directly affect employment rates, showing that improving education access can lead to better employment outcomes. The other direction of causation is much less likely because employment rates do not determine access to education directly, as educational policies and availability are not typically influenced by current employment statistics.
    \item Research showing that "longer daily commute times are associated with lower job satisfaction" suggests (b) Commute time is the explanatory variable, and job satisfaction is the response. The duration of the commute influences job satisfaction levels. This demonstrates a causal relationship where increased commute times negatively impact job satisfaction, implying that reducing commute times could improve employee satisfaction. The other direction of causation is much less likely because job satisfaction does not directly affect the length of commute, as commuting distance is generally a fixed variable based on where one lives and works, rather than influenced by one's level of job satisfaction.
    \item A survey indicating that "older individuals are less likely to support progressive policies" suggests (b) Age is the explanatory variable, and policy support is the response. Age influences political views and policy support. This suggests a causal relationship where age impacts the likelihood of supporting progressive policies, indicating that generational factors play a significant role in political leanings. The other direction of causation is much less likely because changing political support for a policy cannot influence the age of an individual, as age is a fixed and chronological variable that progresses independently of political views.
    \item For the statement "Data suggests that there is no clear pattern between political party affiliation (e.g., Democrat, Republican) and the preference for a particular type of media outlet (e.g., TV, online, newspaper)," the correct choice is \textbf{(b) Political party affiliation is the explanatory variable, and media preference is the response}. This suggests that the researchers initially hypothesized that political party affiliation might influence media preferences. The fact that the study found no clear pattern indicates that there is no causal relationship between these variables. The other direction of causation, where media preference influences political party affiliation, is much less likely because choosing a type of media outlet (such as TV or online news) is generally not a primary factor in determining one's political party affiliation. However, it is important to note that while media content can influence political views, the specific choice of media type (e.g., TV, online) is less likely to determine party affiliation directly.
    \item For the statement "An analysis finds that individuals who identify with a particular political ideology are not more or less likely to engage in volunteer work than those who do not," the correct choice is \textbf{(b) Political ideology is the explanatory variable, and volunteer work engagement is the response}. This indicates that the researchers thought political ideology might influence engagement in volunteer work. The finding of no relationship suggests that political ideology does not affect the likelihood of volunteering. The other direction of causation is much less likely because engaging in volunteer work does not directly influence an individual's political ideology. Political beliefs are generally formed by broader and deeper values and experiences rather than by specific actions like volunteering.
\end{itemize}


\section*{Question 4: Study Classification}
\begin{itemize}
    \item A survey examining ``the correlation between voting behavior and social media usage'' is \textbf{(a) Observational}, as it simply observes and measures characteristics without intervention.
    \item A study where participants are randomly assigned to different conditions (watch political debate or a control program) is \textbf{(b) Experimental} because it manipulates variables to investigate effects on other variables.
\end{itemize}

\section*{Question 5: Sampling Methods}
\begin{enumerate}
    \item Dividing a city into neighborhoods and randomly selecting neighborhoods to survey all residents within them is an example of \textbf{(c) Cluster Sampling}. This method involves dividing the population into clusters and then randomly selecting entire clusters for the survey.
    \item Randomly selecting individuals from each age group to ensure all age groups are represented is an example of \textbf{(b) Stratified Sampling}. This method involves dividing the population into subgroups (strata) and then taking a random sample from each stratum.
\end{enumerate}

\section*{Question 6: Experimental Design}
\begin{enumerate}
    \item \textbf{Four principles of experimental design:}
    \begin{itemize}
        \item \textbf{Control}: Controlling extraneous variables ensures that any observed effects are due to the treatment and not other factors. This principle is crucial for reducing variability and eliminating confounding variables.
        \item \textbf{Randomization}: Randomly assigning subjects to treatment groups helps ensure that each group is similar in all respects, thus eliminating selection bias.
        \item \textbf{Replication}: Repeating the experiment on multiple subjects increases the reliability of the results and ensures that findings are not due to random chance.
        \item \textbf{Blocking}: Grouping similar experimental units together and randomizing within these blocks can reduce variability and increase the precision of the experiment.
    \end{itemize}
    \item \textbf{Analysis of the hypothetical political science study:}
    \begin{itemize}
        \item \textbf{Control}: The study ensures control by selecting a city with a diverse demographic, which helps mitigate the influence of extraneous variables.
        \item \textbf{Randomization}: Participants are randomly assigned to either receive emails or not, which eliminates selection bias.
        \item \textbf{Replication}: The study involves a large and diverse group of participants, increasing the reliability of the results.
        \item \textbf{Blocking}: There is no specific mention of blocking in this study. However, the diverse demographics might implicitly act as a form of blocking.
    \end{itemize}
\end{enumerate}

\section*{Question 7: Confounding Variables}
The absence of controls for political knowledge in the study examining the relationship between educational achievement and political participation is problematic because political knowledge is a known confounding variable. Political knowledge influences both education levels and political participation. Without controlling for political knowledge, the study cannot accurately determine whether the observed correlation between education and political participation is direct or mediated by political knowledge.

\section*{Question 8: Data Interpretation}
\begin{enumerate}
    \item \textbf{Explanatory and response variables:}
    \begin{itemize}
        \item Explanatory variable: Number of visits
        \item Response variable: Increase in voter turnout (\%)
    \end{itemize}
    \item \textbf{Variable types:}
    \begin{itemize}
        \item Region: Categorical (Nominal)
        \item Number of Visits: Numerical (Discrete)
        \item Increase in Voter Turnout (\%): Numerical (Continuous)
    \end{itemize}
    \item \textbf{Sorted data table:}

\begin{center}
    \begin{tabular}{|c|c|c|}
    \hline
    Region & Number of Visits & Increase in Voter Turnout (\%) \\
    \hline
West & 2 & 0.5 \\
South & 3 & 1 \\ 
North & 5 & 2 \\ 
East & 8 & 5 \\
    \hline
    \end{tabular}
\end{center}


    \begin{verbatim}

    \end{verbatim}
    \item \textbf{Analysis of the relationship:}
    There appears to be a positive relationship between the number of candidate visits and the percentage increase in voter turnout. As the number of visits increases, the increase in voter turnout also tends to increase, suggesting a positive association.
\end{enumerate}

\section*{Question 9: Blinding}
Participants in a study are randomly assigned to receive either a real or a placebo political advertisement. Neither the participants nor the researchers who interact with them know which type of advertisement they received. This scenario describes \textbf{(b) Double Blinding}, where both the participants and the researchers are blinded to the treatment allocation.

\end{document}
