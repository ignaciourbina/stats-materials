\documentclass[12pt]{article}
\usepackage{amsmath}
\usepackage{amsfonts}
\usepackage{geometry}
\geometry{margin=1in}

\title{The Hypergeometric Distribution}
\author{}
\date{}

\begin{document}

\maketitle

\section*{Definition}

The \textbf{hypergeometric distribution} describes the probability of drawing a specific number of successes from a finite population \emph{without replacement}.

\bigskip

\noindent It applies when:
\begin{itemize}
    \item The population is of finite size $N$.
    \item The population contains exactly $K$ items classified as ``successes'' and $N-K$ as ``failures''.
    \item A sample of size $n$ is drawn \emph{without replacement}.
    \item We are interested in the probability of observing exactly $k$ successes in the sample.
\end{itemize}

\section*{Probability Mass Function (PMF)}

The hypergeometric probability of observing exactly $k$ successes in a sample of size $n$ is:

\[
P(X = k) = \frac{\dbinom{K}{k} \dbinom{N-K}{n-k}}{\dbinom{N}{n}}
\]

Where:
\begin{itemize}
    \item $N$ is the population size,
    \item $K$ is the number of successes in the population,
    \item $n$ is the sample size,
    \item $k$ is the number of observed successes in the sample,
    \item $\dbinom{a}{b}$ is a binomial coefficient (combinations).
\end{itemize}

\section*{Support (Valid Values for $k$)}

\[
\max(0, n - (N-K)) \leq k \leq \min(n, K)
\]

\section*{Example}

A deck has 52 cards, including 13 hearts. If 5 cards are drawn without replacement, what is the probability that exactly 2 are hearts?

\[
N = 52,\quad K = 13,\quad n = 5,\quad k = 2
\]

\[
P(X = 2) = \frac{\dbinom{13}{2} \dbinom{39}{3}}{\dbinom{52}{5}} = \frac{78 \cdot 9139}{2,598,960} \approx 0.3251
\]

\section*{Expectation and Variance}

\begin{align*}
\text{Expected Value:} \quad & \mathbb{E}[X] = n \cdot \frac{K}{N} \\
\text{Variance:} \quad & \text{Var}(X) = n \cdot \frac{K}{N} \cdot \left(1 - \frac{K}{N} \right) \cdot \frac{N - n}{N - 1}
\end{align*}

\section*{Comparison to Binomial Distribution}

\begin{itemize}
    \item The hypergeometric distribution is similar to the binomial, but:
    \item \textbf{Binomial:} Sampling is done \emph{with replacement} or from an infinite population.
    \item \textbf{Hypergeometric:} Sampling is done \emph{without replacement}.
    \item As $N \to \infty$, the hypergeometric distribution approaches the binomial.
\end{itemize}

\section*{Set-Theoretic Derivation of the Hypergeometric PMF}

Let $\mathcal{U}$ be a finite population with $|\mathcal{U}| = N$, partitioned into two disjoint subsets:
\[
\mathcal{U} = S \cup F, \quad \text{with } |S| = K, \quad |F| = N - K,
\]
where $S$ contains the \textit{successes} and $F$ contains the \textit{failures}.

We select a subset $A \subseteq \mathcal{U}$ of size $n$ uniformly at random.

Define the sample space:
\[
\Omega = \left\{ A \subseteq \mathcal{U} : |A| = n \right\}, \quad |\Omega| = \binom{N}{n}.
\]

Let $X(A) = |A \cap S|$ be the number of successes in the sample $A$. We are interested in computing the probability that $X = k$, i.e., that the sample contains exactly $k$ successes.

\section*{Hypergeometric Probability as a Set Ratio}

We express this probability using classical (equiprobable) set-based probability:
\[
P(X = k) = \frac{\left|\left\{ A \subseteq \mathcal{U} : |A \cap S| = k,\; |A \cap F| = n - k \right\} \right|}{\left| \left\{ A \subseteq \mathcal{U} : |A| = n \right\} \right|}.
\]

\subsection*{Denominator (Total Number of Samples)}

The denominator counts all subsets of size $n$:
\[
\left| \left\{ A \subseteq \mathcal{U} : |A| = n \right\} \right| = \binom{N}{n}.
\]

\subsection*{Numerator (Number of Favorable Samples)}

The numerator counts subsets $A$ of size $n$ that contain exactly $k$ successes and $n - k$ failures.

We define the favorable event set:
\[
\mathcal{E}_k = \left\{ A \subseteq \mathcal{U} : |A \cap S| = k,\; |A \cap F| = n - k \right\}.
\]

Each such subset $A$ is formed by choosing:
\begin{itemize}
    \item $k$ elements from $S$: $\binom{K}{k}$ choices,
    \item $n - k$ elements from $F$: $\binom{N - K}{n - k}$ choices.
\end{itemize}

Because $S$ and $F$ are disjoint, these choices are independent and jointly determine a unique subset $A$. So:
\[
|\mathcal{E}_k| = \binom{K}{k} \cdot \binom{N - K}{n - k}.
\]

\subsection*{Final Result}

Putting it all together:
\[
P(X = k) = \frac{\binom{K}{k} \cdot \binom{N - K}{n - k}}{\binom{N}{n}}.
\]

This is the \textbf{probability mass function} of the hypergeometric distribution:
\[
X \sim \text{Hypergeometric}(N, K, n).
\]

\end{document}
