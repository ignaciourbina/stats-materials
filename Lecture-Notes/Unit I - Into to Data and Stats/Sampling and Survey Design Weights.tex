\documentclass[12pt]{article}
\usepackage{amsmath, amssymb, amsthm}
\usepackage{geometry}

\geometry{margin=1in}

\title{Survey Weights in Two-Stage Sampling Designs}
\author{}
\date{}

\begin{document}

\maketitle

In general, \textbf{survey weights} reflect the inverse of an individual’s probability of being selected into the sample. This document covers survey weights in both \textbf{two-stage cluster sampling} and \textbf{two-stage stratified sampling} designs.

\section{Two-Stage Cluster Sampling}

\subsection{Stage 1: Select Clusters}
Suppose there are \( M \) total clusters in the population, and you randomly select \( m \) of them. If each cluster is equally likely to be chosen, then the probability that a specific cluster \( i \) is selected is
\[
p_i = \frac{m}{M}.
\]

\subsection{Stage 2: Select Individuals Within Each Selected Cluster}
Let cluster \( i \) contain \( N_i \) individuals, and you draw a sample of \( n_i \) individuals. If this selection is simple random sampling within cluster \( i \), then each individual in that cluster has probability
\[
q_i = \frac{n_i}{N_i}
\]
of being chosen (given that cluster \( i \) was selected in the first place).

\section{Computing the Weights}

\subsection{Overall Selection Probability}
The overall selection probability for an individual \( j \) in cluster \( i \) is given by:
\[
\text{Prob}(\text{select } j) = p_i \times q_i.
\]

\subsection{Survey Weight}
The survey weight for that individual is the inverse of the selection probability:
\[
w_{ij} = \frac{1}{p_i \times q_i}.
\]

\section{Common Special Case}
If each cluster has the same number of individuals (say \( N \) each), and you select the same sample size \( n \) per cluster, and all clusters are equally likely to be chosen, then the weight simplifies to:
\[
w_{ij} = \frac{M}{m} \times \frac{N}{n}.
\]

\section{Practical Considerations}
In practice:
\begin{itemize}
    \item If different clusters have different probabilities of selection, you must use each cluster’s actual selection probability \( p_i \).
    \item Similarly, if different numbers of individuals are sampled from each cluster, use each cluster’s actual within-cluster sampling probability \( q_i \).
    \item Many surveys refine these \textit{base weights} with additional adjustments (e.g., for non-response or post-stratification), but the fundamental step always starts with the \textbf{inverse of selection probability}.
\end{itemize}

\section{Two-Stage Stratified Sampling}

A \textbf{two-stage stratified sampling} design differs from cluster sampling in that the population is divided into mutually exclusive strata, and independent samples are drawn within each stratum.

\subsection{Stage 1: Select Primary Sampling Units (PSUs) Within Each Stratum}
Suppose the population is divided into \( H \) strata, with stratum \( h \) containing \( M_h \) primary sampling units (PSUs). If we select \( m_h \) PSUs from stratum \( h \), and each PSU has equal probability of selection, then the probability that PSU \( i \) in stratum \( h \) is selected is:
\[
p_{hi} = \frac{m_h}{M_h}.
\]

\subsection{Stage 2: Select Individuals Within Each Selected PSU}
Let PSU \( i \) in stratum \( h \) contain \( N_{hi} \) individuals. If we sample \( n_{hi} \) individuals randomly from PSU \( i \), the probability of selecting a specific individual in that PSU is:
\[
q_{hi} = \frac{n_{hi}}{N_{hi}}.
\]

\section{Computing the Weights in Stratified Sampling}

\subsection{Overall Selection Probability}
The overall selection probability for an individual \( j \) in PSU \( i \) of stratum \( h \) is:
\[
\text{Prob}(\text{select } j) = p_{hi} \times q_{hi}.
\]

\subsection{Survey Weight}
The corresponding survey weight is:
\[
w_{hij} = \frac{1}{p_{hi} \times q_{hi}}.
\]

\section{Common Special Case}
If all strata have the same number of PSUs (\( M_h = M \)), and the same number of PSUs (\( m_h = m \)) are sampled per stratum, and within each PSU the same number of individuals are sampled (\( n_{hi} = n \)), the weight simplifies to:
\[
w_{hij} = \frac{M}{m} \times \frac{N}{n}.
\]

\section{Practical Considerations in Stratified Sampling}
\begin{itemize}
    \item Strata may have different sampling fractions (\( m_h / M_h \)), so the selection probability must be calculated separately for each stratum.
    \item Within a stratum, different PSUs may have different numbers of individuals (\( N_{hi} \)), requiring different within-PSU sampling probabilities.
    \item As with cluster sampling, adjustments for non-response, post-stratification, or calibration are often applied to the base weights.
\end{itemize}

\end{document}
