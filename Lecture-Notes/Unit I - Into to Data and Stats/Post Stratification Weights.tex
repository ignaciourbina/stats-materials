\documentclass[12pt]{article}
\usepackage{amsmath, amssymb, amsthm}
\usepackage{geometry}

\geometry{margin=1in}

\title{Survey Weights: Sampling Design and Post-Stratification}
\author{}
\date{}

\begin{document}

\maketitle

\section{Introduction to Survey Weights}

Survey weights adjust for differences in selection probability and ensure that estimates derived from the sample are representative of the population. There are two main types of survey weights:

\begin{enumerate}
    \item \textbf{Sampling Design Weights:} These weights correct for unequal selection probabilities based on the survey design.
    \item \textbf{Post-Stratification Weights:} These weights adjust for discrepancies between the sample and known population characteristics, improving representativeness.
\end{enumerate}

In this lecture, we first outline how to compute weights based on the survey design and then introduce post-stratification adjustments.

\section{Sampling Design Weights}

\subsection{Weights in Simple Random Sampling}
If every individual in the population has an equal probability of being selected, the probability of selecting an individual is:
\[
p = \frac{n}{N}
\]
where:
\begin{itemize}
    \item \( N \) is the total population size,
    \item \( n \) is the sample size.
\end{itemize}
The survey weight for each sampled individual is:
\[
w = \frac{1}{p} = \frac{N}{n}.
\]

\subsection{Weights in Stratified Sampling}
In stratified sampling, the population is divided into \( H \) strata, with stratum \( h \) containing \( N_h \) individuals. A sample of size \( n_h \) is drawn from stratum \( h \). The selection probability for an individual in stratum \( h \) is:
\[
p_h = \frac{n_h}{N_h}.
\]
The weight for an individual in stratum \( h \) is:
\[
w_h = \frac{1}{p_h} = \frac{N_h}{n_h}.
\]

\subsection{Weights in Two-Stage Sampling}
For a two-stage design, where first-stage units (PSUs) are selected and then individuals within them:

\begin{itemize}
    \item \textbf{Stage 1:} Select PSU \( i \) from stratum \( h \) with probability:
    \[
    p_{hi} = \frac{m_h}{M_h}.
    \]
    \item \textbf{Stage 2:} Select individual \( j \) within PSU \( i \) with probability:
    \[
    q_{hi} = \frac{n_{hi}}{N_{hi}}.
    \]
    \item \textbf{Overall selection probability:}
    \[
    P_{hij} = p_{hi} \times q_{hi}.
    \]
    \item \textbf{Survey weight:}
    \[
    w_{hij} = \frac{1}{P_{hij}} = \frac{M_h}{m_h} \times \frac{N_{hi}}{n_{hi}}.
    \]
\end{itemize}

\section{Post-Stratification Weights}

\subsection{Purpose of Post-Stratification}
Post-stratification adjusts survey weights so that the weighted sample distribution matches known population totals for key demographic or geographic groups. This is often done when:

\begin{itemize}
    \item The sample does not perfectly reflect the population due to non-response or disproportionate selection probabilities.
    \item Population control totals (e.g., census counts) are available for certain subgroups.
\end{itemize}

\subsection{Computing Post-Stratification Weights}

Suppose we categorize individuals into \( H \) post-strata, where:
\begin{itemize}
    \item \( N_h \) is the known population count for post-stratum \( h \),
    \item \( n_h \) is the number of sampled individuals in post-stratum \( h \),
    \item \( w_h \) is the initial survey weight from the sampling design.
\end{itemize}

The post-stratification weight adjustment factor for stratum \( h \) is:
\[
r_h = \frac{N_h}{\sum_{i \in h} w_i}.
\]
Each individual's final weight is:
\[
w_{h}^{*} = w_h \times r_h.
\]
This ensures that the weighted sum of individuals in each stratum equals the known population count.

\subsection{Example: Simple Post-Stratification Adjustment}
Assume a survey sample consists of two post-strata (e.g., male and female) with the following counts:

\begin{center}
\begin{tabular}{|c|c|c|c|}
\hline
Post-Stratum & Population Size (\( N_h \)) & Sample Size (\( n_h \)) & Initial Weight (\( w_h \)) \\
\hline
Males & 600,000 & 300 & \( \frac{600,000}{300} = 2000 \) \\
Females & 400,000 & 200 & \( \frac{400,000}{200} = 2000 \) \\
\hline
\end{tabular}
\end{center}

The sum of the initial weights for males:
\[
\sum_{i \in \text{males}} w_i = 300 \times 2000 = 600,000.
\]
Since this already matches the population total, no adjustment is needed.

If, however, the sum of initial weights did not match \( N_h \), we would apply the adjustment factor:
\[
r_h = \frac{N_h}{\sum_{i \in h} w_i}.
\]

\section{Combining Design and Post-Stratification Weights}
The final weight incorporating both the sampling design and post-stratification is given by:
\[
w_i^{\text{final}} = w_i^{\text{design}} \times r_h.
\]
This ensures that:
\begin{itemize}
    \item The design weights account for the probability of selection.
    \item The post-stratification adjustment ensures representativeness.
\end{itemize}

\section{Conclusion}
Survey weights play a crucial role in making estimates representative of the population. The key steps in computing final weights are:
\begin{enumerate}
    \item Compute **design-based weights** to correct for unequal selection probabilities.
    \item Apply **post-stratification adjustments** to match known population distributions.
    \item Use the **final adjusted weights** for estimation to reduce bias and improve representativeness.
\end{enumerate}

\end{document}
