\documentclass{article}
\usepackage{amsmath, amssymb, amsthm}
\usepackage{graphicx}
\usepackage{hyperref}
\usepackage{pgfplots}

\title{Law School Admission Probability and Conditional Independence}
\author{POL 201.30}
\date{}

\begin{document}

\maketitle

\section{Introduction}

In this lecture note, we analyze the probability of being accepted to at least one school when applying to multiple institutions. We use Law School as an example, but the logic behind the calculations also applies to similar types of application processes. The key question we explore is whether our original computation of acceptance probability remains valid under different assumptions about applicant quality and school decision processes.

We consider an applicant applying to \( n \)  schools, each with an acceptance probability of \( P(A) = 0.15 \). The main objective is to determine the probability of getting accepted to at least one school:

\[
P(B) = P(\text{at least one acceptance}).
\]

\section{Computing the Probability of Acceptance to at Least One School}

If we assume that each school independently accepts the applicant with probability \( P(A) = 0.15 \), then the probability of rejection from any single school is:

\[
P(\text{rejection}) = P(A^c) = 1 - P(A) = 1 - 0.15 = 0.85.
\]

Since each school's decision is assumed independent, the probability of being rejected by all \( n \) schools is:\footnote{Note that we are using the following notation. Given independent events $A,B, C, \cdots, Z$: $P(A,B,C, \cdots , Z) = P(A \cap B \cap C \cdots \cap Z) = P(A) \cdot P(B) \cdot P(C) \cdots P(Z)$}

\begin{align*}
P(\text{no acceptances}) &= P(\text{Rejected School \#1, Rej. \#2,$\cdots$, Rej. \#\emph{n}} ) \\
 &= (0.85) (0.85) \cdots (0.85) = (0.85)^n.
\end{align*}

Then, let $S$ be the sample space. Let $a_i\in \{0,1\}$ be the application outcome from school $i$, such that $a_i=0$ if school $i$ rejected the application, and $a_i=1$ if it accepted it. Thus, an element of the sample space $\omega \in S$, is a \emph{n}-tuple such that $\omega = (a_1, a_2, \cdots, a_n)$. Using basic combinatorics, we can tell that there are $2^n$ outcomes in the sample space $S$. 

Given this large sample space, there are multiple ways in which at least one element $a_i$ in $\omega$ is equal to one.\footnote{Another way to put it is that there are multiple combinations of $a_i$ such that $\sum_{i=1}^{n}a_i \geq 1$. For instance, if you apply to three schools, $n=3$, then one way to get accepted to at least one is $\omega_1=(1,0,0)$, another is $\omega_2=(1,0,1)$, etc.} Yet, counting such outcomes one by one to compute $P(\text{at least one acceptance})$ is a very tedious task. We can do better by using the complement rule.\footnote{Recall the complement rule: $P(A) = 1 - P(A^c)$.} Let's break down the sample space into two sets: $B$ and $B^c$. We define $B$ as the collections of outcomes in which \emph{at least one school accepted the application}.\footnote{Or, using a more formal notation, $B = \{\omega |\omega \in S, \text{there exists at least one $a_i=1$ in $\omega$} \} $.} Thus, by definition, $B \cup B^c = S$, implying $P(B) + P(B^c) = 1$. Hence,

\begin{align*}
P(B) + P(B^c) = 1 \\
P(\text{at least one acceptance}) + P(\text{no acceptances}) = 1 \\
\boxed{
    P(\text{at least one acceptance}) = 1 - P(\text{no acceptances})
}
\end{align*}


Thus, the probability of receiving at least one acceptance is:

\[
P(\text{at least one acceptance}) = 1 - (0.85)^n.
\]

This formula allows us to compute the probability of success for different values of \( n \). 

\section{Examining the Assumption of Independence}

One potential concern with the above calculation is the assumption of \textbf{independent admissions decisions} across different schools. In reality, schools may not make \emph{statistically independent} decisions. To assess whether our calculation is valid, we need to evaluate the underlying probabilistic structure of the admissions process.

 \textbf{Key Question}: \emph{Is \( P(A_2 | A_1) = P(A_2)  \) a Reasonable Assumption?}

The assumption that \( P(A_2 | A_1) = P(A_2) \) implies that the decision of the second school ($i=2$) is entirely independent of the outcome of the first school ($i=1$). This is reasonable if:

\begin{itemize}
    \item Schools evaluate applicants independently.
    \item Admissions committees do not share information.
    \item The same applicant characteristics are evaluated differently by each institution.
\end{itemize}

However, in real-world settings, admissions decisions may be \textit{statistically associated} (correlated) due to common evaluation criteria, standardized test scores, and similar institutional priorities. In such cases, if an applicant is strong, receiving an acceptance from one school may indicate a higher chance of acceptance at others (i.e., \( P(A_2 | A_1) > 0.15 = P(A_2) \)), and vice versa.

\section{The Role of Applicant Quality \( Q \)}

A more refined model considers \textbf{applicant quality} \( Q \) as a latent variable that influences admissions decisions. That is, each school makes decisions based on an applicant's intrinsic quality.

For any two schools $i=1,2$, a more reasonable assumption is:

\[
P(A_2 | A_1, Q) = P(A_2 | Q).
\]

\textbf{Intuition Behind This Assumption:} \newline
This equation states that once we know the applicant’s quality \( Q \), knowing whether they were accepted to another school (i.e., \( A_1 \)) provides no further information about their probability of getting into \( A_2 \). In other words:

\begin{itemize}
    \item The applicant's quality \( Q \) fully determines their probability of acceptance.
    \item Schools make independent decisions \textbf{given} \( Q \).
    \item Any observed correlation between acceptances is due to the shared dependence on \( Q \).
\end{itemize}

This assumption is widely used in Bayesian modeling and conditional independence structures.

\section{Revisiting Our Original Calculation}
If an applicant knows their quality \( Q \) and has estimated their probability of acceptance as:

\[
P(A | Q) = 0.15,
\]

then under the assumption that each school's decision is \textbf{conditionally independent given \( Q \)}, the probability of at least one acceptance is computed in the same way as before:

\[
P(\text{at least one acceptance}) = 1 - (1 - P(A | Q))^n = 1 - (0.85)^n.
\]

Since our model assumes that individual school decisions remain independent given \( Q \), \textbf{our original computation remains valid}.

\section{Conclusion}
In the context of computing probabilities of acceptances in grad school applications, we explored the validity of the assumption that school admissions decisions are independent. While direct independence \( P(A_2 | A_1) = P(A_2) \) may not always hold due to correlated admissions criteria, the assumption of conditional independence given applicant quality \( Q \), i.e., $ P(A_2 | A_1, Q) = P(A_2| Q) $, justifies our original probability calculation.

Thus, under the assumption that an applicant has estimated their acceptance probability correctly based on \( Q \), i.e., $P(A|Q)=0.15$, the probability of at least one acceptance remains:

\[
P(\text{at least one acceptance}) = 1 - (0.85)^n.
\]

This result provides a useful guideline for applicants considering how many schools to apply to in order to maximize their chances of admission.

\end{document}
