Determining the exact number of public officials prosecuted for corruption annually in the United States is challenging due to variations in reporting and record-keeping across different jurisdictions. However, available data provides some insight:

**Total Number of Elected Officials:**
- The United States has over 500,000 elected positions across federal, state, and local governments. citeturn0search5

**Corruption Prosecutions:**
- In fiscal year 2020, there were 473 official corruption prosecutions reported, marking a 38% increase from the previous fiscal year. citeturn0search2

**Conviction Rates:**
- While specific conviction rates for these prosecutions are not detailed in the provided data, it's important to note that not all prosecutions result in convictions.

**Estimated Prosecution Rate:**
- Assuming 473 prosecutions out of over 500,000 elected officials, the annual prosecution rate for corruption charges is approximately 0.09%.

It's important to note that this estimate includes prosecutions of public officials at all levels and may encompass non-elected officials as well. Additionally, the data does not account for unreported or undetected cases of corruption.

While the percentage of officials prosecuted annually is relatively low, public corruption remains a significant concern, and efforts to detect and prosecute such offenses are ongoing. 