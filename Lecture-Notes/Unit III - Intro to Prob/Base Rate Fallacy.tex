
\documentclass{article}
\usepackage{amsmath}

\begin{document}

\title{Understanding Base Rate Neglect and Bayesian Probability}
\author{POL 201.01}
\date{}
\maketitle

\section{Introduction}

In this document, we analyze a classic example of base rate neglect in statistical reasoning using probability rules. The goal is to understand how probability rules apply to real-world decision-making and to emphasize the importance of considering prior probabilities when interpreting test results. This example serves as an exercise in applying conditional probability and Bayes' Theorem.

\section{Problem Statement}

Suppose a rare condition affects \textbf{1 in 1,000 people} (0.1\% prevalence). A test exists to detect this condition, and it has been designed to be highly accurate. Specifically:
\begin{itemize}
    \item If a person has the condition, the test correctly identifies them as positive \textbf{99\% of the time}.
    \item If a person does NOT have the condition, the test correctly identifies them as negative \textbf{99\% of the time}.
\end{itemize}

\textbf{Question:} If a person tests positive, what is the probability that they actually have the condition?

\section{Understanding the Probability Rules}

To solve this problem, we need to apply the following probability rules:

\subsection{Conditional Probability}
Conditional probability allows us to determine the probability of an event occurring given that another event has already occurred. It is defined as:
\begin{equation}
    P(A \mid B) = \frac{P(A \cap B)}{P(B)}
\end{equation}
where:
\begin{itemize}
    \item $P(A \mid B)$ is the probability of event $A$ occurring given that $B$ has occurred.
    \item $P(A \cap B)$ is the probability of both events occurring together.
    \item $P(B)$ is the probability of event $B$ occurring.
\end{itemize}

\subsection{Bayes' Theorem}
Bayes' Theorem provides a method for updating probabilities when given new information. It is expressed as:
\begin{equation}
    P(D \mid T) = \frac{P(T \mid D) P(D)}{P(T)}
\end{equation}
where:
\begin{itemize}
    \item $P(D \mid T)$ is the probability that a person has the condition (``event $D$") given that they tested positive (``event $T$").\footnote{Hence $D^c$ is the event representing ``not having the condition" and $T^c$ is the event representing ``testing negative."}
    \item $P(T \mid D)$ is the probability that the test is positive given that the person has the condition.
    \item $P(D)$ is the prior probability of having the condition.
    \item $P(T)$ is the total probability of testing positive, which we compute using the law of total probability.
\end{itemize}

\subsection{Law of Total Probability}
The probability of testing positive must consider both true positives and false positives:
\begin{equation}
    P(T) = P(T \mid D) P(D) + P(T \mid D^c) P(D^c)
\end{equation}
where:
\begin{itemize}
    \item $P(T \mid D) P(D)$ represents true positives (correctly identifying those with the condition).
    \item $P(T \mid D^c) P(D^c)$ represents false positives (incorrectly identifying healthy individuals as positive).
    \item $D^c$ represents the event of not having the condition.
\end{itemize}

\section{Step-by-Step Solution}

\textbf{Question:} If a person tests positive, what is the probability that they actually have the condition? In other words, we need to compute:
\boxed{ P(D | T)  }

\subsection{Step 1: Define the Given Probabilities}
\begin{itemize}
    \item $P(D) = 0.001$ (Prevalence of the condition: 1 in 1,000)
    \item $P(D^c) = 1 - P(D) = 0.999$ (Probability of not having the condition)
    \item $P(T \mid D) = 0.99$ (Probability of a positive test given the person has the condition, i.e., \emph{true positive})
    \item $P(T^c \mid D^c) = 0.99$ (Probability of a true negative)
    \item $1 - P(T^c \mid D^c)  = P(T \mid D^c) = 0.01$ (Probability of a \emph{false positive}, since the test correctly identifies negative cases 99\% of the time, meaning 1\% will incorrectly test positive)
\end{itemize}

\subsection{Step 2: Compute $P(T)$ (Total Probability of Testing Positive)}

to find $P(T)$, we sum the contributions of true positives and false positives:
\begin{align*}
    P(T) &= P(T \mid D) P(D) + P(T \mid D^c) P(D^c) \\
    &= (0.99 \times 0.001) + (0.01 \times 0.999) \\
    &= 0.00099 + 0.00999 \\
    &= 0.01098
\end{align*}

\subsection{Step 3: Compute $P(D \mid T)$ Using Bayes' Theorem}

to find the probability that a person actually has the condition given they tested positive, we apply Bayes' theorem:
\begin{align*}
    P(D \mid T) &= \frac{P(T \mid D) P(D)}{P(T)} \\
    &= \frac{0.99 \times 0.001}{0.01098} \\
    &= \frac{0.00099}{0.01098} \\
    &\approx 0.09 \text{ or } 9\%
\end{align*}

\section{Discussion}

The result of our calculation—showing that the probability of actually having the condition given a positive test result is only \textbf{9\%}—is often surprising to many people. According to studies in cognitive psychology, most individuals intuitively estimate this probability to be much higher, often assuming it is close to \textbf{100\%}. This cognitive bias occurs because people tend to focus on the high sensitivity (true positive rate) of the test while neglecting the low base rate of the condition (the low prevalence, i.e., a low $P(D)$). This phenomenon is known as \textbf{Base Rate Neglect}.

Base rate neglect happens because human intuition struggles with probabilistic reasoning, especially when dealing with conditional probabilities. Instead of correctly applying Bayes’ Theorem, most people rely on a heuristic approach: they see that the test is “99\% accurate” and assume that means a positive result almost certainly indicates the presence of the condition. However, this overlooks the fact that false positives can far outnumber true positives when a condition is rare.

Studies have shown that even trained professionals, including medical practitioners, often fall into this reasoning trap when interpreting diagnostic test results. This misjudgment can lead to unnecessary anxiety, additional testing, and even overtreatment of individuals who are actually healthy.

\textbf{Key Takeaway:} To correctly interpret diagnostic tests and other probabilistic information, it is essential to account for both the sensitivity of the test and the underlying prevalence of the condition. Proper application of Bayes’ Theorem helps avoid intuitive but incorrect conclusions, ensuring that decision-making is based on sound probabilistic reasoning rather than cognitive biases.

For conditions or diseases with very low prevalence, a single positive test result often does not provide sufficient evidence to confidently conclude that an individual has the condition. In such cases, it is often advisable to take the test again. If a second, independent test also returns a positive result, then the posterior probability of having the condition increases significantly. This is because the probability of two independent false positives occurring is much lower than that of a single false positive, leading to a much stronger confirmation of the diagnosis. Therefore, repeated testing is a useful strategy when dealing with low-incidence conditions and helps improve diagnostic accuracy.
\end{document}
