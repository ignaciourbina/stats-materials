

\documentclass{article}
\usepackage{amsmath}

\begin{document}

\title{Proportions as Rescaled Binomial Variables}
\author{}
\date{}
\maketitle

\section*{Key Idea}

A sample proportion is simply a rescaled binomial random variable.

\section*{Explanation}

Let $X$ be a binomial random variable:
\[
X \sim \text{Binomial}(n, p)
\]
The sample proportion is defined as:
\[
\hat{p} = \frac{X}{n}
\]

\begin{itemize}
    \item \textbf{Same randomness:} The randomness of $\hat{p}$ is entirely due to $X$.
    \item \textbf{Support:} $X$ takes values in $\{0, 1, \dots, n\}$; $\hat{p}$ takes values in $\left\{0, \frac{1}{n}, \dots, 1\right\}$.
    \item \textbf{Distribution:}
\[
    \Pr(\hat{p} = k/n) = \Pr(X = k) = \binom{n}{k} p^k (1-p)^{n-k}
\]
    \item \textbf{Mean and Variance:}
\[
    \mathbb{E}[\hat{p}] = p, \qquad \text{Var}(\hat{p}) = \frac{p(1-p)}{n}
\]
\end{itemize}

\section*{Conclusion}

Since $\hat{p}$ is obtained by scaling $X$ by $1/n$, the sample proportion is simply a rescaled binomial variable. Note:
\begin{itemize}
    \item $E((1/n)X)=(1/n) np=p$.
    \item $V((1/n)X)=(1/n)^2 V(X) = (1/n)^2 (np(1-p))=(1/n)p(1-p)$
\end{itemize}

\end{document}
