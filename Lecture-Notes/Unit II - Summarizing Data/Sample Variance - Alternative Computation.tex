\documentclass[12pt]{article}

% Preamble packages
\usepackage{amsmath, amssymb}
\usepackage{graphicx}
\usepackage{booktabs}

\title{Computing the Sample Variance}
\author{Lecture Notes}
\date{\today}

\begin{document}

% Title page


% Introduction frame
\section{Overview}
In some situations, you might have the following summary statistics:
\begin{itemize}
    \item \(\sum_{i=1}^{N} x_i^2\): the sum of the squares of the observations,
    \item \(\bar{x}\): the sample mean, and
    \item \(N\): the sample size.
\end{itemize}
We will show that you can compute the sample variance \(s^2\) using these quantities with the formula:
\[
s^2 = \frac{\sum_{i=1}^{N} x_i^2 - N\,\bar{x}^2}{N-1}.
\]


% Derivation frame
\section{Derivation of the Formula}
Start with the definition of the sample variance:
\[
s^2 = \frac{1}{N-1}\sum_{i=1}^{N} \left(x_i - \bar{x}\right)^2.
\]
\bigskip

\textbf{Step 1.} Expand the squared term:
\[
\sum_{i=1}^{N} \left(x_i - \bar{x}\right)^2
= \sum_{i=1}^{N} \left(x_i^2 - 2x_i\bar{x} + \bar{x}^2\right).
\]
\bigskip

\textbf{Step 2.} Separate the sum:
\[
\sum_{i=1}^{N} x_i^2 - 2\bar{x}\sum_{i=1}^{N} x_i + N\,\bar{x}^2.
\]
Since the sum of the observations is
\[
\sum_{i=1}^{N} x_i = N\,\bar{x},
\]
the expression simplifies to:
\[
\sum_{i=1}^{N} x_i^2 - 2N\bar{x}^2 + N\,\bar{x}^2
= \sum_{i=1}^{N} x_i^2 - N\,\bar{x}^2.
\]
\bigskip

\textbf{Step 3.} Substitute back into the variance formula:
\[
s^2 = \frac{\sum_{i=1}^{N} x_i^2 - N\,\bar{x}^2}{N-1}.
\]


% Example frame
\section{Example Calculation}
Suppose we have a sample with the following summary statistics:
\begin{itemize}
    \item \(\displaystyle \sum_{i=1}^{3} x_i^2 = 14,\)
    \item \(\bar{x} = 2,\) and
    \item \(N = 3.\)
\end{itemize}

Then, the sample variance is computed as:
\[
s^2 = \frac{14 - 3\,(2^2)}{3-1}
= \frac{14 - 12}{2}
= \frac{2}{2} = 1.
\]
Thus, the sample variance is \(1\).

% Summary frame
\section{Summary}
If you know:
\begin{itemize}
    \item \(\sum_{i=1}^{N} x_i^2\),
    \item the sample mean \(\bar{x}\), and
    \item the sample size \(N\),
\end{itemize}
you can compute the sample variance using the formula:
\[
s^2 = \frac{\sum_{i=1}^{N} x_i^2 - N\,\bar{x}^2}{N-1}.
\]
This \emph{computational formula} is especially useful when these summary statistics are readily available.


\end{document}
