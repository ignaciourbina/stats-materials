\documentclass{article}
\usepackage[letterpaper, top=0.25in, left=0.4in, right=0.4in, bottom=0.2in]{geometry}
\usepackage{amsmath, amssymb}
\usepackage{tcolorbox}
\usepackage[dvipsnames]{xcolor}
\usepackage{ifthen}
\tcbuselibrary{listings,breakable}
\usepackage{float} % Allows precise placement of floating objects
\usepackage{tikz} % Core TikZ package
\usepackage{pgfplots} % For statistical plots
\pgfplotsset{compat=1.18} % Ensure compatibility

\usepackage{comment}

\usetikzlibrary{positioning, shapes, calc, backgrounds, decorations.pathreplacing, arrows.meta, plotmarks}

% Package for changing counter properties, specifically for resetting enumerate counter per section
\usepackage{chngcntr}
% This command makes the 'enumi' counter (used by the first level of enumerate)
% reset with every new 'section' and formats its output (\theenumi)
% as <section_number>.<enumi_value>. For example, 1.1, 1.2, then 2.1, 2.2, etc.
\counterwithin{enumi}{section}


% Define a new command for a centered floating blank text box
\newcommand{\blankbox}[2][3cm]{%
    \vspace{-0.5em}
    \begin{figure}[H]
        \makebox[\linewidth]{% Ensures the box extends to full line width
            \begin{tcolorbox}[
                colback=white,
                colframe=white,  % adjust the border frame color
                width=#2, % Adjusting for A4 paper margins
                height=#1,
                boxrule=0.2mm
            ]
            \end{tcolorbox}
        }
    \end{figure}
    \vspace{-2em}
}

% Define a new boolean for showing or hiding answers
\newboolean{showanswers}
\setboolean{showanswers}{false}  % Set to true to show answers, false to hide answers

% Define the command to conditionally display answers
\newcommand{\answer}[1]{
    \ifthenelse{\boolean{showanswers}}{
        \begingroup
        \color{MidnightBlue}
        \begin{tcolorbox}[colback=white, colframe=MidnightBlue, title=Solution, fonttitle=\bfseries, breakable, fontupper=\color{MidnightBlue}]
        #1
        \end{tcolorbox}
        \endgroup
    }{}
}

%%%%%%%%%%%%%%%%%%%%%%%%%%%%%%%%%%%%%%%%%%%%%%%%%%%%%%%%%%%%%%%%%%%%%%%%%%%%%%%%%%%%%%%%%%%%%%%

\begin{document}
%\setlength{\parindent}{0pt} % remove indentation in the whole document
\pagenumbering{gobble} % Suppress page numbers
 \hspace{1em} \vspace{-1.7em} % This line might cause a small horizontal space at the very top.
\begin{center}
   \Large    POL201.01 -- Make-Up Final, Spring 2025.
\end{center}

\vspace{1em}
\noindent\textbf{First and Last Name:} \underline{\hspace{8cm}}  \quad  \textbf{Student ID:} \underline{\hspace{4.4cm}}

\vspace{0.7em}
\noindent\textbf{Instructions:}

\vspace{-0.8em}
\begin{itemize}
    \setlength{\itemsep}{-0.35em}
    \item Turn off and put away your phone and any electronic devices.
    \item You have from \textbf{12:00 PM to 2:30 PM} to complete the exam.
    \item Show all your work. Partial credit may be awarded even if final calculations are wrong.
    \item Read each question carefully and manage your time wisely.
    \item Write your answers in the \emph{space provided below each question statement}. Only answers written in the designated areas will be graded.
  %  \item Make sure to sign-in the \textbf{attendance sheet}.
    \item \textbf{Rounding:} Round all your calculations up to the fourth decimal.
    \item If you have any questions, raise your hand and ask the instructor.
\end{itemize}
\vspace{-1.1em}
\noindent\rule{\linewidth}{0.4pt} % Horizontal line to separate this section from the exam content

%%%%%%%%%%%%%%%%%%%%%%%%%%%%%%%%%%%%%%%%%%%%%%%%%%%%%%%%%%%%%%%%%%%%%%%%%%%%%%%%%%%%
\section{Probability (6 points total)}
\noindent\textbf{Instructions:} \\
Select the single best answer (A, B, C, or D) for each of the following multiple-choice questions. Circle your response clearly. You may use scratch space to perform any calculations.
\begin{enumerate}

\item \textbf{(3 points)} In a recent poll, 30\% of respondents say they watch political debates on TV, 25\% say they follow debates online, and 10\% say they do both.

If one person is selected at random, what is the probability they watch debates on TV or online?
\begin{enumerate}
    \item[(A)] $0.55$
    \item[(B)] $0.45$
    \item[(C)] $0.475$
    \item[(D)] $0.075$
\end{enumerate}  \blankbox[1.5cm]{1.0\linewidth}

\item \textbf{(3 points)}  A national survey of 700 adults asked two questions: whether they support implementing a federal carbon tax (support coded as 1, and do not support as 0) and how much they agree with the statement:
\textit{“Climate change poses a serious threat to the country’s future.”}
Responses to the agreement statement were measured on a 5-point scale (from Strongly Disagree to Strongly Agree). The results are summarized below:

\begin{center}
\begin{tabular}{lp{1.6cm}ccp{1.6cm}p{1.6cm}} \hline
 & \multicolumn{5}{c}{\textbf{Agreement to the Climate Threat Statement}} \\ \cline{2-6}
 & \textbf{Strongly Disagree} & \textbf{Disagree} & \textbf{Neutral} & \textbf{Somewhat Agree} & \textbf{Strongly Agree} \\ \hline
\textbf{Support Carbon Tax (1)}       &  30 &  50 &  70 & 120 &  80 \\
\textbf{Do Not Support (0)} &  60 &  90 & 100 &  50 &  50 \\ \hline
\end{tabular}
\end{center}

What is the probability that a respondent supports a carbon tax, \textbf{given} that they answered “Somewhat Agree” or “Strongly Agree” to the climate threat statement?

\begin{enumerate}
  \item[(A)] $0.29$
  \item[(B)] $0.5$
  \item[(C)] $0.57$
  \item[(D)] $0.67$
\end{enumerate}


\end{enumerate}

%%%%%%%%%%%%%%%%%%%%%%%%%%%%%%%%%%%%%%%%%%%%%%%%%%%%%%%%%%%%%%%%%%%%%%%%%%%%%%%%%%%%
\newpage
%%%%%%%%%%%%%%%%%%%%%%%%%%%%%%%%%%%%%%%%%%%%%%%%%%%%%%%%%%%%%%%%%%%%%%%%%%%%%%%%%%%%
\section{Descriptive Statistics (9 points total)}
\noindent\textbf{Instructions:} \\
In a public opinion study, respondents were asked to place themselves on an ideological spectrum from 1 (most liberal) to 11 (most conservative). Samples of 200 residents were collected from each of two cities in different states. The table below shows the absolute frequency of responses in each group (e.g., 22 respondents from City A placed themselves at score position of ``1," that is, ``most liberal"). Basic summary statistics are also provided.

\begin{center}
\begin{tabular}{lccccccccccc}
\hline
& \multicolumn{11}{c}{\textbf{Ideological Placement}} \\ \cline{2-12}
& \textbf{1} & \textbf{2} & \textbf{3} & \textbf{4} & \textbf{5} & \textbf{6} & \textbf{7} & \textbf{8} & \textbf{9} & \textbf{10} & \textbf{11} \\
\hline
\multicolumn{12}{l}{\textit{Absolute Frequencies}} \\
\textbf{City A} & 22 & 18 & 20 & 18 & 16 & 12 & 16 & 18 & 18 & 20 & 22 \\
\textbf{City B} & 17 & 18 & 16 & 19 & 18 & 21 & 17 & 19 & 18 & 17 & 18 \\
\hline
\end{tabular}
\end{center}

\textbf{Summary statistics of the 1 to 11 ideological placement score variable:}

\begin{itemize}
\item \textbf{City A:} Mean = 6.01; Standard Deviation = 3.34
\item \textbf{City B:} Mean = 6.03; Standard Deviation = 3.13
\end{itemize}

\begin{enumerate}
\item \textbf{(3 points)} Create a bar chart (graph) for each city's ideological scores based on the frequency table. \blankbox[6cm]{1.0\linewidth} % Describe the general shape of each distribution.

\item \textbf{(3 points)} Both cities have almost the same mean ideological score and fairly similar standard deviations. Does this mean the ideological distributions are similar? Justify your response by interpreting each distribution's shape. \blankbox[4.75cm]{1.0\linewidth}

\item \textbf{(3 points)} In plain language, what can we learn about the distribution of ideological placements for each city? What are the main takeaways? %\blankbox[6cm]{1.0\linewidth}
\end{enumerate}

%%%%%%%%%%%%%%%%%%%%%%%%%%%%%%%%%%%%%%%%%%%%%%%%%%%%%%%%%%%%%%%%%%%%%%%%%%%%%%%%%%%%
\newpage

\section{Random Variables (5 points total)}
\noindent\textbf{Instructions:} Please show all your calculations. Justify and explain all your answers. Consider the following information to answer the questions in this section: \\

Floods can vary greatly in severity, from minor water intrusion to total property loss. Suppose the following table summarizes expert estimates for the \emph{typical annual property damage} faced by uninsured homeowners in a flood-prone region. Each level reflects the amount a private homeowner might need to pay out-of-pocket for repairs or rebuilding in a given year. \textit{The table below represents four possible outcomes, each with an associated cost and probability; together, they describe a random variable for annual flood-related damages.}

\begin{center}
\begin{tabular}{|l|c|c|c|c|}
\hline
\textbf{Flood Severity} & No Flood & Moderate Flood & Heavy Flood & Catastrophic Flood \\
\hline
\textbf{Damage Cost ($C$)} & \$0 & \$5,000 & \$20,000 & \$100,000 \\
\hline
\textbf{Probability} & 0.70 & 0.20 & 0.08 & 0.02 \\
\hline
\end{tabular}
\end{center}

\begin{enumerate}
\item \textbf{(2 points)} Let $C$ be the random variable representing the yearly flood-related \emph{cost} for an uninsured property in this region. Compute the expected value $\mathbb{E}[C]$. \blankbox[6cm]{1.0\linewidth}

\item \textbf{(2 points)} In plain language, explain what $\mathbb{E}[C]$ means in this context. What should a homeowner understand from this number, even if they don’t expect a flood every year? \blankbox[5cm]{1.0\linewidth}

\item \textbf{(1 point)} Suppose a person is considering buying flood insurance for their home. They are willing to get insurance only if the \emph{amount they pay each year} for it is less than the expected yearly cost of flood damage. If they are offered insurance that costs \textbf{\$4,000 per year}, should they take it? Fully justify and explain your answer. %\blankbox[6cm]{1.0\linewidth}
\end{enumerate}

%%%%%%%%%%%%%%%%%%%%%%%%%%%%%%%%%%%%%%%%%%%%%%%%%%%%%%%%%%%%%%%%%%%%%%%%%%%%%%%%%%%%
\newpage

%% NOTE TO SELF:            %%%%%%%%%%%%%%%%%%%%%%%%%%%%%%%%%%%%%%%%%%%%%%%%%%%%%%
%%     -  Remember to match the Z test value to one of the exact values of the table.
%%            %%%%%%%%%%%%%%%%%%%%%%%%%%%%%%%%%%%%%%%%%%%%%%%%%%%%%% %%%%%%%%%%%%%%
\section{Hypothesis Testing (25 points total)}
\noindent\textbf{Instructions:} Please show all your calculations. Round your final answers to four decimal places. Justify and explain all your answers. Consider the following information to answer the questions in this section: \\

Suppose a survey of $n = 2{,}000$ U.S.\ adults asked whether they enjoy eating spicy food (recorded as ``yes") or not (recorded as ``no"). Among the $600$ respondents with \textbf{foreign-born parents}, $330$ said “yes.” Among the $1{,}400$ respondents with \textbf{U.S.-born parents}, $700$ said “yes.”

\begin{enumerate}
    \item \textbf{(10 points)} State appropriate null and alternative hypotheses to \emph{test whether there is a difference in spicy food preference between adults with foreign-born parents and those with U.S.-born parents}. Use mathematical notation and define each term used in your notation. \blankbox[3cm]{1.0\linewidth}

    \item \textbf{(10 points)} Conduct a two-proportion $z$-test at the $\alpha = 0.05$ significance level to evaluate the stated hypotheses. \\
    Clearly show: \textbf{(I)} The formulas used, along with the substituted values;
    \textbf{(II)} The resulting $z$-statistic;
    \textbf{(III)} Either the critical value or the $p$-value (\textbf{Choose only one});
    \textbf{(IV)} A clear conclusion: reject or fail to reject the null hypothesis.
    \blankbox[9cm]{1.0\linewidth}

    \item  \textbf{(5 points)}  Interpret the result in context. Is there convincing statistical evidence of a difference in preference for spicy food based on parental background? Explain in plain language. \blankbox[6cm]{1.0\linewidth}
\end{enumerate}


%%%%%%%%%%%%%%%%%%%%%%%%%%%%%%%%%%%%%%%%%%%%%%%%%%%%%%%%%%%%%%%%%%%%%%%%%%%%%%%%%%%%
\newpage
\section{Inference Concepts and Applications (30 points total)}
\noindent\textbf{Instructions:} \\
Select the single best answer (A, B, C, or D) for each of the following multiple-choice questions. Circle your response clearly. You may use scratch space to perform any calculations if and whenever required.

\begin{enumerate}
    \item \textbf{(10 points)}   A political scientist constructs a 90\% confidence interval for the proportion of voters who are against a proposed policy and finds the interval is (0.41, 0.49). Which of the following conclusions is justified based on the information provided about the interval?
    \begin{enumerate}
        \item[(A)] There is a 95\% chance that the true proportion is between 0.41 and 0.49.
        \item[(B)] The estimated sample proportion of voters who opposed the policy is 0.45, since it is in the middle of the interval.
        \item[(C)] At least 90\% of all voters are against the policy.
        \item[(D)]  We are 95\% confident that the true population proportion of voters supporting the policy is between 0.51 and 0.59.
    \end{enumerate} \blankbox[2cm]{1.0\linewidth}


    \item \textbf{(10 points)} A pollster reports that 62\% of respondents in a representative sample support Candidate Y in an upcoming election. The pollster also reports a \textbf{standard error (SE)} of 1.9 percentage points (that is, $SE=0.019$). Based on this information, which of the following is the correct statement for the corresponding 95\% confidence interval?

    \begin{enumerate}
        \item[(A)] Between 60.1\% and 63.9\% of all voters support Candidate Y.
        \item[(B)] Based on the interval, candidate Y will surely win because more than half of the voters support them.
        \item[(C)] Between 57\% and 67\% of all voters support Candidate Y.
        \item[(D)] Between 58.3\% and 65.7\% of all voters support Candidate Y.
    \end{enumerate}
    \blankbox[2cm]{1.0\linewidth}

    \item \textbf{(10 points)} A researcher wants to test whether there is a difference in average weekly cereal spending between households with children (group 1) and households without children (group 2). Which is the correct null and alternative hypothesis for this test?

    \begin{enumerate}
        \item[(A)] $H_0: \bar{x}_1 = \bar{x}_2$, $H_a: \bar{x}_1 \neq \bar{x}_2$
        \item[(B)] $H_0: \mu_1 = \mu_2$, $H_a: \mu_1 \neq \mu_2$
        \item[(C)] $H_0: \mu_1 \neq \mu_2$, $H_a: \mu_1 = \mu_2$
        \item[(D)]  $H_0: p_1 = p_2$, $H_a: p_1 \neq p_2$
    \end{enumerate}
    \blankbox[2cm]{1.0\linewidth}



\end{enumerate}


%%%%%%%%%%%%%%%%%%%%%%%%%%%%%%%%%%%%%%%%%%%%%%%%%%%%%%%%%%%%%%%%%%%%%%%%%%%%%%%%%%%%
\newpage
\section{Confidence Intervals (25 points total)}
\noindent\textbf{Instructions:} Please show all your work, including formulas used, and values substituted. Round your final answers to four decimal places.

\bigskip
\noindent A human resources analyst at a mid-sized tech company wants to estimate the average number of cups of coffee employees drink per day. The company has 120 employees. The analyst randomly samples 10 employees and records the following number of cups consumed on a typical workday:

\[
\text{Data: } \{\, 2,\ 3,\ 1,\ 4,\ 2,\ 3,\ 2,\ 5,\ 3,\ 4 \,\}
\]

\begin{enumerate}
    \item \textbf{(10 points)} Compute the sample mean $\bar{x}$ and the sample standard deviation $s$ for these 10 employees. \blankbox[6cm]{1.0\linewidth}

    \item \textbf{(10 points)} Construct a 99\% confidence interval for the \emph{population mean} number of cups of coffee consumed per day by employees at the company. Use the $t$-distribution. \\ Clearly show:
    \textbf{(I)} The formulas used, along with the substituted values;
    \textbf{(II)} Degrees of freedom ($df$) used;
    \textbf{(III)} The appropriate Critical $t$-value used to construct the interval;
    \textbf{(IV)} Your final confidence interval.
    \blankbox[8.5cm]{1.0\linewidth}

    \item \textbf{(5 points)} Provide an intuitive interpretation of the confidence interval using plain language. \blankbox[3cm]{1.0\linewidth}
\end{enumerate}


%%%%%%%%%%%%%%%%%%%%%%%%%%%%%%%%%%%%%%%%%%%%%%%%%%%%%%%%%%%%%%%%%%%%%%%%%%%%%%%%%%%%
\end{document}


    \item In a two-tailed hypothesis test, the p-value is found to be 0.08. At the $\alpha = 0.05$ level, what is the correct conclusion?
    \begin{enumerate}
        \item[(A)] Fail to reject the null hypothesis; no statistically significant evidence.
        \item[(B)] Reject the null hypothesis; the difference is statistically significant.
        \item[(C)] The null hypothesis is proven true.
        \item[(D)] The result is statistically significant because the p-value is less than 0.10.
    \end{enumerate} \blankbox[2cm]{1.0\linewidth}

\item \textbf{Consider the following statement:} \newline
    \emph{``Since any estimate is one possible result from a single sample, we shouldn’t trust it as there is no way to know how representative it is.”} \newline
    Which of the following best evaluates the truth and reasoning of this statement?
    \begin{enumerate}
    \item[(A)] The statement is \textbf{False}, because most sample estimates exactly match the population value in practice.
    \item[(B)] The statement is \textbf{False}, because although estimates come from samples, we can quantify their uncertainty using sampling distributions.
    \item[(C)] The statement is \textbf{True}, because without knowing the population, we have no grounds to assess an estimate’s accuracy.
    \item[(D)] The statement is \textbf{True}, because estimates are inherently unreliable and can’t be evaluated.
    \end{enumerate}

    \item A public health study finds that 20\% of adults smoke. What is the probability that in a random sample of two adults, \emph{neither} of them smokes, assuming independence?
\begin{enumerate}
    \item[(A)] 0.04
    \item[(B)] 0.40
    \item[(C)] 0.64
    \item[(D)] 0.80
\end{enumerate} %\blankbox[0.15cm]{1.0\linewidth}

%%%%%%%%%%%%%%%%%%%%%%%%%%%%%%%%%%%%%%%%%%%%%%%%%%%%%%%%%%%%%%%%%%%%%%%
    \item \textbf{Rounding:}
    \begin{itemize}
    \vspace{-0.6em} \setlength{\itemsep}{-0.1em}
        \item When dealing with numbers between -1 and 1, round calculations up to the fourth decimal.\footnote{For instance, suppose $SE=0.0273632$. Then, $Round(SE)=0.0274$.}
        \item When dealing with numbers outside the $(-1, -1)$ range, round calculations up to the second decimal.\footnote{For instance, suppose $Z_{crit}=2.085302$. Then, $Round(Z_{crit})=2.09$.}
    \end{itemize}
