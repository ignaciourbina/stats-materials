% ------------------------------------------------------------
%  MINIMAL PREAMBLE FOR THE PROPORTION‑INFERENCE PROBLEM SET
% ------------------------------------------------------------
\documentclass[11pt]{article}

% ---------- layout & fonts ----------
\usepackage[margin=1in]{geometry}  % nice 1‑inch margins
\usepackage{lmodern}               % Latin Modern fonts (scales better)
\usepackage[T1]{fontenc}

% ---------- math & symbols ----------
\usepackage{amsmath,amssymb}       % basic AMS math
\usepackage{bm}                    % bold math symbols if desired

% ---------- lists ----------
\usepackage{enumitem}              % custom labels in enumerate/itemize
\setlist[itemize]{itemsep=3pt}
\setlist[enumerate]{itemsep=4pt, topsep=6pt}
%  (the large problem blocks use:  \begin{enumerate}[label=\textbf{Q\,\arabic*:}, ...])

% ---------- hyperlinks (optional) ----------
\usepackage[colorlinks=true, linkcolor=blue]{hyperref}

% ---------- nice table rules if you add tables ----------
\usepackage{booktabs}

% ---------- header/footer (optional) ----------
\usepackage{fancyhdr}
\pagestyle{fancy}
\fancyhf{}
\fancyhead[L]{\small POL 201 — Random‑Variable \& Proportion Practice}
\fancyhead[R]{\small \thepage}

% ---------- useful macros ----------
\newcommand{\E}{\mathbb{E}}
\newcommand{\Var}{\operatorname{Var}}
\newcommand{\pr}{\Pr}

% ---------- ensure PDFLaTeX compiles cleanly ----------
\usepackage{microtype}             % nicer kerning & margins


\usepackage{tikz}
\usetikzlibrary{shapes.geometric, arrows.meta, positioning}
\usepackage{pgfplots}
\pgfplotsset{compat=1.18}


\title{Final Practice Problem Set}
\author{}
\date{}

\begin{document}

\maketitle

% ------------------------------------------------------------
%  Inference for One Proportion — Problems 1–5
% ------------------------------------------------------------
\begin{enumerate}[label=\textbf{Q\,\arabic*:}, left=0pt]

%-------------------------------------------------------------
\item  \emph{Can machines feel?} \\
      A June 2022 YouGov survey of $n = 2{,}434$ U.S.\ adults asked whether computers will ever experience feelings and emotions. The responses were: $21\%$ said \emph{yes}, $55\%$ said \emph{no}, and $24\%$ were \emph{not sure}.
      \begin{enumerate}[label=(\alph*)]
         \item Compute a 95\% confidence interval for the proportion of U.S.\ adults who believe computers \textbf{will} experience emotions.
         \item Does the interval support the claim that a “clear minority” of Americans hold this belief? Justify using a rhetorical threshold of 25\%.
         \item Suppose an AI company claims “public support is growing rapidly” for belief in emotional AI. Based on this snapshot, would you agree? Suggest one way to verify that trend over time.
      \end{enumerate}
      \textit{Why it’s interesting:} prompts students to think about how public opinion interacts with emerging technology, and how confidence intervals relate to rhetorical claims and trend analysis.

      \vspace{0.5em}
      \noindent\textbf{Source:} YouGov. “Do you think computers will ever experience feelings and emotions?” June 14, 2022. Retrieved from \href{https://today.yougov.com/topics/politics/survey-results/daily/2022/06/14/0cd0f/1}{today.yougov.com}

\item  \emph{Is there a gender gap in belief about AI emotions?} \\
      A June 2022 YouGov poll asked $n = 2{,}434$ U.S.\ adults whether they believe computers will ever experience feelings and emotions. Among male respondents, $26\%$ said “yes”; among female respondents, $16\%$ said “yes.” Assume the sample was evenly split between men and women.

      \begin{enumerate}[label=(\alph*)]
         \item State appropriate null and alternative hypotheses to test whether men and women differ in belief about computers experiencing emotions.
         \item Conduct a two-proportion $z$-test at the $\alpha = 0.05$ significance level.
         \item Interpret the result in context. Is there convincing evidence of a gender gap? How large is the difference, practically speaking?
      \end{enumerate}
      \textit{Why it’s interesting:} gives students practice with two-sample inference, helps interpret statistical vs. practical significance, and encourages thoughtful conclusions from public opinion data.

      \vspace{0.5em}
      \noindent\textbf{Source:} YouGov. “Do you think computers will ever experience feelings and emotions?” June 14, 2022. Retrieved from \href{https://today.yougov.com/}{today.yougov.com}


%-------------------------------------------------------------
\item  \emph{Framing the responsibility} \\
      A YouGov survey conducted on April 7, 2025, asked $n = 4{,}610$ U.S.\ adults how responsible they believe President Donald Trump is for the stock market. Of those surveyed, $43\%$ said he is \emph{very responsible}, and an additional $24\%$ said he is \emph{somewhat responsible}.
      \begin{enumerate}[label=(\alph*)]
         \item Compute a 95\% confidence interval for the proportion of U.S.\ adults who say Trump is \textbf{very responsible}.
         \item Compute a 95\% confidence interval for the proportion who say he is \textbf{at least somewhat responsible} (i.e., \emph{very} or \emph{somewhat} responsible).
         \item Some media outlets claim “a majority of Americans believe Trump is responsible for the stock market.” Based on your intervals, evaluate whether that claim is supported. How does the conclusion differ depending on which group (very vs. at least somewhat) is analyzed?
      \end{enumerate}
      \textit{Why it’s interesting:} tests how conclusions hinge on the framing and aggregation of categories; reinforces interpretation of confidence intervals against rhetorical claims.

      \vspace{0.5em}
      \noindent\textbf{Source:} YouGov. “How responsible, or not responsible, is President Donald Trump for the stock market?” April 7, 2025. Retrieved from \href{https://today.yougov.com/topics/politics/survey-results/daily/2025/04/07/6be5c/3}{today.yougov.com}
%-------------------------------------------------------------
\item \emph{A super majority support?} \\ A February 2025 national survey by YouGov of $n=4{,}334$ U.S.\ adults found that $2{,}904$ respondents
      said they would support amending the Constitution to set a maximum age (e.g.\ 75) for federal elected officials.\footnote{\textbf{Source:} YouGov. “Do you support or oppose setting a maximum age for elected officials?” Daily Question, February 21, 2025. Retrieved from \url{https://today.yougov.com/topics/politics/survey-results/daily/2025/02/21/a75c8/3}}
      \begin{enumerate}[label=(\alph*)]
         \item Compute a 95\% confidence interval for the true proportion of U.S.\ adults who favor the amendment.
         \item Does the interval include the 60\% mark commonly cited as a “super‑majority” threshold?  Interpret.
         \item Politicians claim “well over two‑thirds” of voters favour the amendment.  Test this claim at $\alpha=0.05$.
      \end{enumerate}

%-------------------------------------------------------------
\item \emph{Is there a gender gap in belief about AI emotions?} \\
      A June 2022 YouGov poll asked $n = 2{,}434$ U.S.\ adults whether they believe computers will ever experience feelings and emotions. Among male respondents, $26\%$ said “yes”; among female respondents, $16\%$ said “yes.” Assume the sample was evenly split between men and women.

      \begin{enumerate}[label=(\alph*)]
         \item State appropriate null and alternative hypotheses to test whether men and women differ in belief about computers experiencing emotions.
         \item Conduct a two-proportion $z$-test at the $\alpha = 0.05$ significance level.
         \item Interpret the result in context. Is there convincing statistical evidence of a gender gap in this belief? How large is the difference, practically speaking?
      \end{enumerate}
      \textit{Why it’s interesting:} gives students practice with two-sample inference, helps interpret statistical vs. practical significance, and encourages thoughtful conclusions from public opinion data.

      \vspace{0.5em}
      \noindent\textbf{Source:} YouGov. “Do you think computers will ever experience feelings and emotions?” June 14, 2022. Retrieved from \href{https://today.yougov.com/}{today.yougov.com}

%-------------------------------------------------------------
\item  \emph{ “No opinion” isn’t zero.}  \\
      A pollster asks whether the Supreme Court is doing an “excellent/good” job.
      Historically that approval fraction hovers near $p_0=0.45$.
      This year, in a sample of $n=850$ respondents, $354$ say “excellent/good”.
      But $170$ additional respondents answer “no opinion”.
      \begin{enumerate}[label=(\alph*)]
         \item Treating “excellent/good” versus “anything else” as a binary variable, test whether approval has dropped below 45\% ($H_0{:}p=0.45$) at $\alpha=0.01$.
         \item Discuss qualitatively how excluding the “no opinion” group (and analysing only the 680 substantive responses) would change the test statistic and $p$‑value.  Which analysis better answers the pollster’s question?
      \end{enumerate}
      \textit{Why it’s interesting:}  introduces non‑response/neutral category considerations and robustness of inferences to coding choices.

%-------------------------------------------------------------
\item  \emph{Turnout promises vs.\ reality.}  \\
      In advance of a campus referendum, the student senate claims that “at least 70\% of undergraduates intend to vote.”
      A random e‑mail poll of $n=600$ students finds that $378$ say they \emph{intend} to vote.
      On election day only $300$ of those 600 actually vote.
      \begin{enumerate}[label=(\alph*)]
         \item Build a 99\% confidence interval for the true intention proportion.
         \item Test at $\alpha=0.01$ whether intentions reach 70\%.
         \item Using the realised voting data ($300/600$), repeat the interval and test.
         \item Briefly discuss social‑desirability bias and \emph{intention–behaviour gaps}.
      \end{enumerate}
      \textit{Why it’s interesting:}  contrasts stated intention vs.\ observed behaviour using identical $n$, illustrating different population parameters and the “intention–action” gap common in political surveys.

%-------------------------------------------------------------
\item  \emph{New drug, old benchmark.}  \\
      A legacy malaria prophylaxis prevents infection in 58\% of exposed travellers.
      In a Phase III trial the new drug is given to $n=230$ volunteers; $154$ remain infection‑free.
      \begin{enumerate}[label=(\alph*)]
         \item Construct a 90\% confidence interval for the effectiveness proportion $p$.
         \item Perform a one‑sided test ($H_0{:}p\le0.58$ versus $H_A{:}p>0.58$) at $\alpha=0.10$.
         \item Explain why a one‑sided alternative is appropriate in this regulatory context and how Type I vs.\ Type II errors translate to public‑health consequences.
      \end{enumerate}
      \textit{Why it’s interesting:}  incorporates applied context of non‑inferiority/superiority trials and discussion of error costs.

%-------------------------------------------------------------
\item  \emph{Detecting bots on social media.}  \\
      A researcher builds a classifier that flags accounts as bots.
      When run on a large \emph{training} set, it has positive predictive value (precision) 0.80.
      To evaluate current bot prevalence, the classifier is applied to a random sample of $n=400$ Twitter accounts and flags $62$ of them.
      Assume the precision of 0.80 holds for today’s data.
      \begin{enumerate}[label=(\alph*)]
         \item Let $\pi$ be the true bot proportion.  Show that the expected proportion of flagged accounts equals $0.80\,\pi$.
         \item Using that relationship, derive an estimator $\widehat{\pi}$ based on the observed $62/400$ flagged.
         \item Build a 95\% CI for $\pi$ using the delta‐method approximation (treat $\widehat{\pi}=F/0.80$, where $F$ is the flagged proportion, and propagate the binomial SE of $F$).
         \item Comment on the main source of uncertainty that is *not* captured by this interval.
      \end{enumerate}
      \textit{Why it’s interesting:}  introduces measurement error in the binary outcome itself, linking precision to effective “dilution” of observed proportion and requiring a small algebraic manipulation before inference.

\end{enumerate}

% ------------------------------------------------------------
%  Inference for One Proportion — Problems 6–10
% ------------------------------------------------------------
\begin{enumerate}[label=\textbf{Q\,\arabic*:}, start=6, left=0pt]

%-------------------------------------------------------------
\item  \emph{Fact‑checking fatigue.}  \\
      A journalist asserts that fewer than one‑quarter of news‑consumers routinely fact‑check stories before sharing.
      In a scrolling‑intercept poll of $n=520$ online readers, $98$ respondents say they “usually” verify a headline before reposting it.
      \begin{enumerate}[label=(\alph*)]
          \item Carry out the test $H_0{:}p \ge 0.25$ vs.\ $H_A{:}p < 0.25$ at $\alpha=0.05$.
          \item Compute a one‑sided 95\% confidence bound for $p$ and interpret.
          \item Suppose the true $p$ equals the upper bound you just reported.  How large a sample would be needed for a margin of error no larger than $\pm 2$ percentage points?
      \end{enumerate}
      \textit{Why it’s interesting:} connects one‑sided CI to hypothesis tests and designs a follow‑up sample‑size calculation.

%-------------------------------------------------------------
\item  \emph{The disappearing landline.}  \\
      In 2015 the CDC estimated that 41\% of U.S.\ households had a working landline.
      A 2024 phone study dials a random set of $n=1{,}500$ addresses and reaches a landline in $486$ of them.
      \begin{enumerate}[label=(\alph*)]
          \item Construct a 99\% confidence interval for today’s landline proportion.
          \item Does the interval suggest a statistically significant change from 2015?  Explain.
          \item If the study had instead found \emph{600} landlines, how would the CI shift, and would the conclusion change?
      \end{enumerate}
      \textit{Why it’s interesting:} pushes students to compare historical benchmarks and explore how slight data changes alter inference.

%-------------------------------------------------------------
\item  \emph{“First‑gen” awareness campaign.}  \\
      Before orientation week, administrators believe that at most 30\% of incoming students self‑identify as first‑generation college students.
      After a social‑media campaign, a follow‑up random sample of $n=280$ first‑years finds $107$ who now self‑identify as first‑gen.
      \begin{enumerate}[label=(\alph*)]
          \item Test whether the proportion has increased beyond 30\% using $\alpha=0.05$.
          \item Compute a 90\% CI and interpret in plain language for campus administrators.
          \item Discuss two non‑sampling factors that could explain a difference between the pre‑campaign belief (30\%) and the follow‑up estimate.
      \end{enumerate}
      \textit{Why it’s interesting:} introduces program‐evaluation flavour and asks for qualitative reflection on bias and construct validity.

%-------------------------------------------------------------
\item  \emph{Solo‑driver surcharge.}  \\
      A city considers a congestion surcharge if less than half of daily commuters car‑pool or use transit.
      A stratified random survey of $n=680$ morning commuters records $356$ solo drivers.
      \begin{enumerate}[label=(\alph*)]
          \item Provide a 95\% CI for the true solo‑driver proportion and state whether the surcharge criterion (<50\%) is met.
          \item Suppose sampling weights reveal that solo drivers were \emph{under‑sampled} by a factor of 0.9.
                Re‑express the estimated proportion and qualitatively describe how the confidence interval would shift.
      \end{enumerate}
      \textit{Why it’s interesting:} highlights design weights / post‑stratification and their impact on proportion estimates.

%-------------------------------------------------------------
\item  \emph{Quality‑control and the rule of three.}  \\
      A manufacturer tests $n=120$ randomly selected microchips and finds \emph{zero} defects.
      \begin{enumerate}[label=(\alph*)]
          \item Using the “rule of three” (i.e.\ $\hat{p}_{\max}\approx 3/n$), give an approximate 95\% upper confidence bound for the true defect rate.
          \item Compare that bound with the exact Clopper–Pearson 95\% upper limit (you may quote the formula without computing).  Why are they different?
          \item Under a Six‑Sigma standard ($p<3.4$ defects per million), do these data provide convincing evidence that the chips meet the standard?
      \end{enumerate}
      \textit{Why it’s interesting:} forces students to handle the “zero success” edge‑case, introduces a quick back‑of‑the‑envelope rule, and links to industrial quality benchmarks.

\end{enumerate}

% ------------------------------------------------------------------
%  Inference for Two Proportions — Problems 11–15
% ------------------------------------------------------------------
\begin{enumerate}[label=\textbf{Q\,\arabic*:}, start=11, left=0pt]

%-------------------------------------------------------------
\item  \emph{Mail‑in vs.\ Election‑day turnout.} \\
      In a recent county election, the registrar reports that 2 ,415 of 3 ,950 mail‑in voters cast a ballot, while
      1 ,882 of 3 ,240 in‑person registrants showed up on election day.
      \begin{enumerate}[label=(\alph*)]
          \item Construct a 95\% confidence interval for the difference in turnout proportions $(p_{\text{mail}}-p_{\text{day}})$.
          \item Test at the 5\% level whether turnout is \emph{higher} among mail‑in voters.
          \item Explain in one sentence why the large‑sample normal approximation is (or is not) appropriate here.
      \end{enumerate}

%-------------------------------------------------------------
\item  \emph{Remote vs.\ on‑campus learning outcomes.} \\
      Among 180 students who took a course fully online, 139 passed the final exam;
      of 150 similar students who took the same course on campus, 123 passed.
      \begin{enumerate}[label=(\alph*)]
          \item Compute the pooled standard error and the $z$ statistic for testing equality of pass rates.
          \item At $\alpha=0.10$, do the data provide evidence that the on‑campus format yields a higher pass rate?
          \item Suppose the university wants to detect a 6‑percentage‑point difference with 80\% power at $\alpha=0.05$.  Roughly how many total students would have to be enrolled in such an experiment?
      \end{enumerate}

%-------------------------------------------------------------
\item  \emph{Vaccine acceptance across age groups.} \\
      A survey of 700 adults aged 18–34 finds 427 willing to take a new booster, while
      536 of 860 adults aged 55 + say the same.
      \begin{enumerate}[label=(\alph*)]
          \item Give a 99\% confidence interval for the difference in acceptance proportions (younger minus older).
          \item Interpret the interval in context—be explicit about direction and magnitude.
          \item Why might stratification by additional variables (e.g.\ education) change the interval?
      \end{enumerate}

%-------------------------------------------------------------
\item  \emph{Fact‑checking habits: print vs.\ digital natives.} \\
      In a media‑literacy study, $96$ of $240$ “digital natives” (grew up with smartphones) and $58$ of $210$ “print natives” (grew up pre‑internet)
      correctly identified \emph{all} factual errors in a news story.
      \begin{enumerate}[label=(\alph*)]
          \item Test whether digital natives outperform print natives at $\alpha=0.05$.
          \item Compute Cohen’s $h$ effect‑size measure for the difference in two proportions and interpret its magnitude.
      \end{enumerate}

%-------------------------------------------------------------
\item  \emph{Gender gap in climate‑action support.} \\
      A nationally representative poll asks if the respondent supports a carbon tax.
      Of 1 ,120 women, 641 say “yes”; of 1 ,030 men, 542 say “yes.”
      \begin{enumerate}[label=(\alph*)]
          \item Construct a two‑sided 95\% confidence interval for $(p_{\text{women}}-p_{\text{men}})$.
          \item Suppose a lobby group claims there is “no meaningful gender gap.”
                Translate that claim into a formal $H_0$ and report the $p$‑value associated with the observed data.
          \item Briefly discuss whether statistical significance here automatically implies practical significance for policy makers.
      \end{enumerate}

\end{enumerate}


% ------------------------------------------------------------------
%  Inference for Two Proportions — Problems 16–20
% ------------------------------------------------------------------
\begin{enumerate}[label=\textbf{Q\,\arabic*:}, start=16, left=0pt]

%-------------------------------------------------------------
\item  \emph{Urban vs.\ rural broadband access.} \\
      A technology audit randomly samples 540 households in urban ZIP codes and 470 in rural ZIP codes.
      It finds 472 urban households and 301 rural households with reliable high‑speed internet.
      \begin{enumerate}[label=(\alph*)]
          \item Compute a 90\% confidence interval for the access gap $(p_{\text{urban}}-p_{\text{rural}})$.
          \item The agency will subsidise rural rollout if the gap exceeds 12 percentage points.
                Based on your CI, should the agency act?  Explain.
      \end{enumerate}

%-------------------------------------------------------------
\item  \emph{Open‑source vs.\ proprietary software failures.} \\
      Among 1 ,800 servers running an open‑source OS, 87 crashed last quarter;
      among 900 similar servers running proprietary software, 78 crashed.
      \begin{enumerate}[label=(\alph*)]
          \item Carry out a two‑proportion $z$ test for different crash rates at $\alpha=0.01$.
          \item Explain why, despite the large $n$, the normal approximation might be questionable in the proprietary group.
          \item Suggest (no calculations needed) a method to verify robustness of your inference when expected counts are small.
      \end{enumerate}

%-------------------------------------------------------------
\item  \emph{Outreach e‑mail wording A/B test.} \\
      A nonprofit sends 12 ,500 fund‑raising e‑mails using Subject‑line A and 12 ,500 using Subject‑line B.
      Version A yields 1 ,035 donations; Version B yields 926.
      \begin{enumerate}[label=(\alph*)]
          \item Construct a 95\% CI for the lift (difference in donation proportions).
          \item Compute the \emph{relative} increase $\hat{p}_A/\hat{p}_B-1$ and
                comment on why marketers sometimes prefer that metric to an absolute difference.
      \end{enumerate}

%-------------------------------------------------------------
\item  \emph{Mask compliance before vs.\ after signage change.} \\
      Observers record mask use at a hospital entrance in two independent two‑hour windows.
      Before new signage: 212 of 356 entrants were masked.
      After signage: 247 of 332 entrants were masked.
      \begin{enumerate}[label=(\alph*)]
          \item Test at $\alpha=0.05$ whether compliance improved.
          \item Compute the 95\% CI for the change and explain whether it is practically large enough to justify the cost of the new signs.
      \end{enumerate}

%-------------------------------------------------------------
\item  \emph{Click‑through‑rate decay on mobile vs.\ desktop.} \\
      A news site shows the same headline on mobile and desktop home‑pages.
      In the first hour, 493 of 8 ,120 mobile visitors click; 552 of 6 ,050 desktop visitors click.
      In the \emph{second} hour the numbers are 308/7 ,940 (mobile) and 408/6 ,020 (desktop).
      \begin{enumerate}[label=(\alph*)]
          \item For hour 1, test $H_0{:}p_m=p_d$ vs.\ $H_A{:}p_m\neq p_d$.
          \item Repeat for hour 2.
          \item Comment on the pattern: does the platform gap widen, shrink, or stay constant?
                How would you formally test that the \emph{change‑in‑gap} is different from zero?
      \end{enumerate}

\end{enumerate}


\end{document}
