\documentclass{article}
\usepackage[letterpaper, total={7in, 10.5in}]{geometry}
\usepackage{amsmath, amssymb}
\usepackage{tcolorbox}
\usepackage[dvipsnames]{xcolor}
\usepackage{ifthen}
\tcbuselibrary{listings,breakable}
\usepackage{float}

\title{Solution Key - Midterm Practice Problem Set}
\author{}
\date{}

\begin{document}

\maketitle

\section*{Section 1: Introduction to Data}

\begin{enumerate}

\item \textbf{Statement A:} \\
\textbf{False.} Owner Age is a continuous numerical variable as it can take on a range of values. However, Business Type is a \textbf{nominal} categorical variable, not ordinal, as there is no meaningful ranking among its categories.

\item \textbf{Statement B:} \\
\textbf{False.} The study is \textbf{observational} because researchers collected self-reported data without assigning treatments or interventions. Since no random assignment was used, causation cannot be inferred.

\item \textbf{Statement C:} \\
\textbf{True.} If the sampling method ensured that every member of the population had an equal probability of being selected, then the sample is \textbf{representative}, allowing valid generalizations to the population.

\item \textbf{Statement D:} \\
\textbf{False.} Even if Loan Approval and Years in Business are strongly associated, correlation does not imply causation. Other factors, such as credit score or revenue, could be confounding variables affecting loan approval rates.

\item \textbf{Statement E:} \\
\textbf{False.} While the sample was randomly selected within a specific subgroup (businesses with financial institution interactions), the study suffers from \textbf{selection bias} since businesses without such interactions were excluded, making the sample non-representative of all businesses.

\end{enumerate} %%%%%%%%%%%%%%%%%%%%%%%%%%%%%%%%%%%%%%%%%

\section*{Section 2: Describing and Summarizing Data}

\begin{enumerate}
\item \textbf{Measures of Central Tendency:}\begin{itemize}
\item Mean: $\sum x_i / n = 54/ 18 = 3.0$.
\item Median: $(X_{(18/2)}+X_{(18/2+1)})/2=(X_{(9)}+X_{(10)})/2=(2+3)/2=2.5$
\item Mode: 1 (the value with the highest absolute frequency)
\item Interpretation: The average student in the sample spends three hours discussing politics with friends. The mean is higher than the median, but not by a significantly large amount. This suggest some degree of right-skewness in the distribution. We can also see this by looking at the mode, since the mode at 1 indicates a concentration at the lower end. Given there is some skweness, one could argue that the median should be preferred over the mean as a measure of central tendency.
\end{itemize}
\item \textbf{Measures of Spread:}
\begin{itemize}
    \item Range: $\max(x_i)-\min(x_i)=7-1=6$
    \item Interquartile Range (IQR): $IQR=Q_3-Q_1=4-1=3$. The distance between the first quartile and the third is equal to 3 units (3 hours of discussions).
        \begin{itemize}
            \item We find $Q_1$ by looking at the value in the position $0.25\cdot n=0.25\cdot 18=4.5$. We see that this falls between the observation ranked 4th and the one ranked 5th. Hence we compute $(X_{(4)}+X_{(5)})/2=(1+1)/2=1$.
            \item Similarly, we find $Q_3$ by looking at the value in the position $0.75\cdot n=0.75\cdot 18=13.5$. This falls between the observation ranked 13th and the one ranked 14th. Hence we compute $(X_{(13)}+X_{(14)})/2=(4+4)/2=4$.
        \end{itemize}
    \item Standard Deviation: $s=\sqrt{(\sum (x_i-\bar{x})^2 )/ (n-1) }= \sqrt{64/ 17} = \sqrt{3.76} = 1.94 $.
    \item Interpretation: The range of 6 suggests that there is some spread in the data, but it does not indicate whether the data is evenly distributed or concentrated in certain areas. The interquartile range (IQR) of 3 shows that the middle 50\% of the data is moderately dispersed. The standard deviation of 1.94 suggests that, on average, individual observations deviate by about 1.94 units (hours of discussions) from the mean. Overall, the spread of the data appears moderate, with no extreme variability.

\end{itemize}

\item \textbf{Skewness and Outliers:}
\begin{itemize}
    \item Skewness: The distribution is right-skewed since the mean is greater than the median.
    \item Outliers: No extreme outliers were detected based on the 1.5 * IQR rule. In other words, there are no $x_i$ such that $x_i\geq Q_3+1.5IQR = 8.5$ or $x_i\leq Q_1 - 1.5IQR = -3.5$.
\end{itemize}

\item \textbf{Impact of New Data (Adding 15):}
\begin{itemize}
    \item New Mean: The mean will increase significantly because it is sensitive to extreme values. Adding 15, a value much higher than the rest of the data, pulls the average upward.
    \item New Median: The median will either remain the same or increase slightly, depending on where the new data point falls relative to the middle of the distribution. Since the median only depends on the middle-ranked values, one extreme value does not drastically affect it.
    \item Robustness: The median is more robust than the mean because it is less affected by extreme values. The mean considers all values in the dataset, so a single large value like 15 increases it noticeably, while the median remains relatively stable.
\end{itemize}

\end{enumerate}

%%%%%%%%%%%%%%%%%%%%%%%%%%%%%%%%%%%%%%%%%%%%%%%%%%%%%%%%
\section*{Section 3: Probability Rules}


\begin{enumerate}

    \item \textbf{Question A:} \\
    \textbf{Solution:} Since Candidate A and Candidate B are mutually exclusive choices (a voter cannot support both simultaneously), we apply the \textbf{addition rule for disjoint events}:
    \[
    P(A \text{ or } B) = P(A) + P(B)
    \]
    Substituting the given probabilities:
    \[
    P(A \text{ or } B) = 0.40 + 0.35 = 0.75
    \]
    Thus, the probability that a randomly selected voter supports either Candidate A or Candidate B is 0.75.

    \item \textbf{Question B:} \\
    \textbf{Solution:} Let \( E \) be the event that a student participates in at least one extracurricular activity. The probability that a randomly selected student from each school participates is:
    \[
    P(E | A) = \frac{P(A \cap E)}{P(A)} = \frac{500}{1200} = 0.4167, \quad
    P(E | B) =  \frac{P(B \cap E)}{P(B)} = \frac{400}{800} = 0.50
    \]
    Since \( P(E | B) > P(E | A) \), a randomly selected student from School B is more likely to participate in extracurricular activities than one from School A. Thus, the provided statement is \textbf{False}.

    \item \textbf{Question C:} \\
    \textbf{Solution:} We use the formula for conditional probability:
    \[
    P(\text{Tablet} | \text{Laptop}) = \frac{P(\text{Tablet} \cap \text{Laptop})}{P(\text{Laptop})}
    \]
    Substituting the given probabilities:
    \[
    P(\text{Tablet} | \text{Laptop}) = \frac{0.48}{0.80} = 0.60
    \]
    Thus, the probability that a student owns a tablet given that they own a laptop is 0.60.

    \item \textbf{Question D:} \\
    \textbf{Solution:} Given random sampling, the probability of one voter turning out to vote is independent of another, thus, we apply the \textbf{multiplication rule for independent events}:
    \[
    P(A \cap B) = P(A) \times P(B)
    \]
    Substituting the given probabilities:
    \[
    P(\text{Both Vote}) = 0.65 \times 0.65 = 0.4225
    \]
    Thus, the probability that both randomly selected voters will turn out to vote is 0.4225.

    \item \textbf{Question E:} \\
    \textbf{Solution:} We use the general addition rule for non-mutually exclusive events:
    \[
    P(A \cup B) = P(A) + P(B) - P(A \cap B)
    \]
    Substituting the given probabilities:
    \[
    P(\text{Statistics or Political Science}) = 0.50 + 0.40 - 0.20 = 0.70
    \]
    Thus, the probability that a randomly selected student is taking either a statistics or a political science course is 0.70.

    \item \textbf{Question F:} \\
    \textbf{Solution:} Using the law of total probability, we compute:
    \[
    P(\text{Misinformation}) = P(M | SM) P(SM) + P(M | TV) P(TV) + P(M | NP) P(NP)
    \]
    Substituting the given values:
    \[
    P(\text{Misinformation}) = (0.25 \times 0.30) + (0.10 \times 0.50) + (0.05 \times 0.20)
    \]
    \[
    = 0.075 + 0.05 + 0.01 = 0.135
    \]
    Thus, the probability that a randomly selected person encounters misinformation is 0.135.

    \item \textbf{Question G:} \\
    \textbf{Solution:} Using Bayes' theorem:
    \[
    P(M | F) = \frac{P(F | M) P(M)}{P(F)}
    \]
    First, compute the total probability of being flagged:
    \[
    P(F) = P(F | M) P(M) + P(F | M^c) P( M^c)
    \]
    \[
    P(F) = (0.95 \times 0.2) + (0.15 \times 0.8) = 0.19 + 0.12 = 0.31
    \]
    Now, compute the posterior probability:
    \[
    P(M | F) = \frac{0.19}{0.31} \approx 0.613
    \]
    Thus, the probability that an article is actually misinformation given that it was flagged is 0.613.

    \item \textbf{Question H:} \\
    \textbf{Solution:} Using the conditional probability formula:
    \[
    P(X | \text{Not Undecided}) = \frac{P(X)}{P(X \cup Y)}
    \]
    Substituting the given probabilities:
    \[
    P(X | \text{Not Undecided}) = \frac{0.24}{0.39} \approx 0.615
    \]
    Thus, the probability that a voter prefers Candidate X given that they are not undecided is 0.615.

\end{enumerate}
%%%%%%%%%%%%%%%%%%%%%%%%%%%%%%%%%%%%%%%%%%%%%%%%%%%%%

\section*{Section 4: Probability - Vote Miscounting Puzzle}

\begin{enumerate}

    \item \textbf{Question A:} \\
    \textbf{Solution:} Using the \textbf{law of total probability}, we sum over the probabilities of a vote being correctly counted for each candidate:
    \[
    P(M^c) = P(M^c | A) P(A) + P(M^c | B) P(B)
    \]
    Substituting the given probabilities:
    \[
    P(M^c) = (0.90 \times 0.55) + (0.90 \times 0.45) = 0.495 + 0.405 = 0.90
    \]
    Thus, the probability that a randomly selected vote is counted correctly is 0.90.

    \item \textbf{Question B:} \\
    \textbf{Solution:} Using the \textbf{complement rule}, we find the probability of a vote being miscounted:
    \[
    P(M) = 1 - P(M^c)
    \]
    \[
    P(M) = 1 - 0.90 = 0.10
    \]
    Thus, the probability that a randomly selected vote is miscounted is 0.10.

    \item \textbf{Question C:} \\
    \textbf{Solution:} We use the \textbf{law of total probability} to determine the probability that a vote was truly cast for Candidate \( A \). Since a vote for \( A \) could have been either correctly counted as \( A \) or miscounted as \( B \), we sum these two probabilities:
    \[
    P(A^*) =  P(M^c \cap A) + P(M \cap B) = P(M^c | A) P(A) + P(M | B) P(B)
    \]
    Substituting the given probabilities:
    \[
    P(A^*) = (0.90 \times 0.55) + (0.10 \times 0.45) = 0.495 + 0.045 = 0.54
    \]

    This calculation follows from the fact that we must account for both correctly counted and miscounted votes. The first term represents the fraction of Candidate \( A \)'s votes that were accurately recorded, while the second term accounts for votes originally cast for Candidate \( A \) but incorrectly attributed to \( B \). By summing these contributions, we obtain the adjusted proportion of votes that should belong to Candidate \( A \) in the final tally.


    \item \textbf{Question D:} \\
    \textbf{Solution:} Similarly, using the \textbf{law of total probability}, we compute the probability that a vote was truly cast for Candidate \( B \):
    \[
    P(B^*) = P(M^c \cap B) + P(M \cap A) = P(M^c | B) P(B) + P(M | A) P(A)
    \]
    Substituting the given probabilities:
    \[
    P(B^*) = (0.90 \times 0.45) + (0.10 \times 0.55) = 0.405 + 0.055 = 0.46
    \]
    Thus, the probability that a randomly selected vote was truly cast for Candidate \( B \) is 0.46.

    \item \textbf{Question E:} \\
    \textbf{Solution:} Comparing \( P(A^*) \) and \( P(B^*) \), we see:
    \[
    P(A^*) = 0.54, \quad P(B^*) = 0.46
    \]
    Since \( P(A^*) > P(B^*) \), \textbf{Candidate \( A \) still wins} despite the miscounting.  In other words, even if we were to find a way to correctly count all the votes, i.e., to eliminate the miscounting, $A$ would still win.

    \item \textbf{Question F:} \\
    \textbf{Solution:} Suppose the probability of miscounting is some unknown value \( \theta \), where \( P(M | A) = P(M | B) = \theta \). We express \( P(A^*) \) and \( P(B^*) \) in terms of \( \theta \):
    \[
    P(A^*) = (1 - \theta) P(A) + \theta P(B) = (1 - \theta) (0.55) + \theta (0.45)
    \]
    \[
    P(B^*) = (1 - \theta) P(B) + \theta P(A) = (1 - \theta) (0.45) + \theta (0.55)
    \]
    Setting \( P(A^*) = P(B^*) \) to solve for \( \theta \):
    \[
    (1 - \theta) (0.55) + \theta (0.45) = (1 - \theta) (0.45) + \theta (0.55)
    \]
    \[
    0.55 - 0.55\theta + 0.45\theta = 0.45 - 0.45\theta + 0.55\theta
    \]
    \[
    0.55 - 0.10\theta = 0.45 + 0.10\theta
    \]
    \[
    0.10 = 0.20\theta
    \]
    \[
    \theta = 0.50
    \]
    This means that if the probability of miscounting exceeds 50\%, the expected winner could change. It shows that elections with high miscounting rates can significantly impact results.

\end{enumerate}
%%%%%%%%%%%%%%%%%%%%%%%%

\section*{Section 5: Linear Combination of Random Variables}

\begin{enumerate}

    \item[(a)] \textbf{Expected total commission earnings:}
    Using the linearity of expectation:
    \[
    E[Z] = E[0.05X + 0.10Y] = 0.05E[X] + 0.10E[Y]
    \]
    Substituting the given expectations:
    \[
    E[Z] = 0.05(5000) + 0.10(3000) = 250 + 300 = 550
    \]
    Thus, the expected total commission earnings are \$550.

    \item[(b)] \textbf{Variance of total commission earnings:}
    Since \( X \) and \( Y \) are independent, the variance rule states:
    \[
    V(Z) = V(0.05X + 0.10Y) = (0.05^2)V(X) + (0.10^2)V(Y)
    \]
    Substituting the given variances:
    \[
    V(Z) = (0.05^2)(1,000,000) + (0.10^2)(400,000)
    \]
    \[
    = (0.0025 \times 1,000,000) + (0.01 \times 400,000)
    \]
    \[
    = 2500 + 4000 = 6500
    \]
    Thus, the variance of total commission earnings is 6500.

    \item[(c)] \textbf{Standard deviation of \( Z \):}
    The standard deviation is given by:
    \[
    \sigma_Z = \sqrt{V(Z)}
    \]
    Substituting the computed variance:
    \[
    \sigma_Z = \sqrt{6500} \approx 80.62
    \]
    Thus, the standard deviation of total commission earnings is \$80.62.

    \item[(d)] \textbf{Z-score of 630.62:}
    The Z-score is calculated as:
    \[
    Z_{\text{score}}(630.62) = \frac{630.62 - E[Z]}{\sigma_Z}
    \]
    Substituting the computed values:
    \[
    Z_{\text{score}} = \frac{630.62 - 550}{80.62} = \frac{80.62}{80.62} = 1.00
    \]
    Thus, 630.62 is exactly one standard deviation above the mean.

    \item[(e)] \textbf{Probability of earning more than \$630.62:}
    Since \( Z_{\text{score}}(630.62) = 1.00 \), we use the standard normal table:
    \[
    P(Z \leq 1.00) = 0.8413
    \]
    Applying the complement rule:
    \[
    P(Z > 1.00) = 1 - P(Z \leq 1.00) = 1 - 0.8413 = 0.1587
    \]
    Thus, the probability that Alex earns more than \$630.62 in commissions is 0.1587 (or 15.87\%).

\end{enumerate}

\section*{Section 6: Random Variables II}

\begin{enumerate}

    \item[(a)] \textbf{Expected number of participants:}
    Using the definition of expected value for a discrete random variable:
    \[
    E[P] = \sum p_i \cdot \Pr(p_i)
    \]
    Substituting the given values:
    \[
    E[P] = (50000 \times 0.08) + (75000 \times 0.22) + (100000 \times 0.31) + (125000 \times 0.24) + (150000 \times 0.15)
    \]
    \[
    = 4000 + 16500 + 31000 + 30000 + 22500 = 104000
    \]
    Thus, the expected number of participants is 104,000.

    \item[(b)] \textbf{Total Monthly Cost as a Linear Combination:}
    The total monthly cost \( C \) includes a fixed administrative cost and a variable cost per participant:
    \[
    C = 6,000,000 + 200P
    \]
    Since \( P \) is a random variable, this expresses \( C \) as a linear function of \( P \).

    \item[(c)] \textbf{Probability Mass Function (PMF) of Total Monthly Cost:}
    Using the computed values, the total cost outcomes and their probabilities are:

\begin{table}[H]
    \centering
    \begin{tabular}{c|c|c}
        \hline
        \textbf{Participants (P)} & \textbf{Total Cost (\$)} & \textbf{Probability} \\
        \hline
        50,000 & 16,000,000 & 0.08 \\
        75,000 & 21,000,000 & 0.22 \\
        100,000 & 26,000,000 & 0.31 \\
        125,000 & 31,000,000 & 0.24 \\
        150,000 & 36,000,000 & 0.15 \\
        \hline
    \end{tabular}
    \caption{Probability Mass Function (PMF) of Total Monthly Cost, Computed from Participants}
    \label{tab:pmf_total_cost}
\end{table}


    \item[(d)] \textbf{Expected Value of the Total Monthly Cost:}
    Using linearity of expectation:
    \[
    E[C] = 6,000,000 + 200E[P]
    \]
    Substituting \( E[P] = 104,000 \):
    \[
    E[C] = 6,000,000 + (200 \times 104,000) = 6,000,000 + 20,800,000 = 26,800,000
    \]
    Thus, the expected total monthly cost is \$26,800,000.

    \item[(e)] \textbf{Cumulative Distribution Function (CDF) of Total Monthly Cost:}
    The cumulative probabilities for each cost level are:

    \begin{table}[H]
    \centering
    \begin{tabular}{c|c}
        \hline
        \textbf{Total Cost (\$)} & \textbf{Cumulative Probability} \\
        \hline
        16,000,000 & 0.08 \\
        21,000,000 & 0.30 \\
        26,000,000 & 0.61 \\
        31,000,000 & 0.85 \\
        36,000,000 & 1.00 \\
        \hline
    \end{tabular}
    \caption{Cumulative Distribution Function of Total Monthly Cost}
    \end{table}

    \item[(f)] \textbf{Assessing Budget Constraints:}
    The probability that the program’s cost exceeds \$30,000,000 is:
    \[
    P(C > 30,000,000) = P(C = 31,000,000) + P(C = 36,000,000)
    \]
    \[
    = 0.24 + 0.15 = 0.39
    \]
    Thus, the probability that the cost of the program exceeds \$30,000,000 is 39\%.

\end{enumerate}


\end{document}
