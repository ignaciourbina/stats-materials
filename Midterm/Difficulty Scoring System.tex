\documentclass{article}
\usepackage{amsmath}

\title{Difficulty Scoring System for Assessment Instruments}
\author{}
\date{}

\begin{document}

\maketitle

\section{Introduction}
This document presents a structured methodology for assessing the relative difficulty of a test or problem set. The system assigns a difficulty score to each question based on three dimensions: 

\begin{itemize}
    \item \textbf{Cognitive Load (CL)}: Measures the mental effort required to answer the question.
    \item \textbf{Mathematical Complexity (MC)}: Assesses the computational difficulty of solving the problem.
    \item \textbf{Interpretative Depth (ID)}: Evaluates how much reasoning and justification are required.
\end{itemize}

Each component is scored from 0 to 40, leading to a maximum question difficulty score of 120.

\section{Scoring System}
Each question receives a total difficulty score based on the formula:
\[
D_i = CL_i + MC_i + ID_i
\]
where \(D_i\) is the total difficulty score for question \(i\), and \(CL_i\), \(MC_i\), and \(ID_i\) are the component scores.

The overall difficulty of a test containing \( N \) questions is then:
\[
D_{\text{total}} = \sum_{i=1}^{N} D_i
\]

The **maximum theoretical difficulty** of the instrument is calculated assuming that every question attains the highest possible difficulty score:
\[
D_{\text{max}} = 120 \times N
\]

\section{Computing Relative Difficulty}
To determine the relative difficulty of the instrument, we use the ratio:
\[
D_{\text{relative}} = \frac{D_{\text{total}}}{D_{\text{max}}} \times 100
\]

This percentage represents how difficult the actual instrument is relative to the hardest possible test.

\subsection{Steps to Compute Relative Difficulty}
\begin{enumerate}
    \item Assign scores for \( CL \), \( MC \), and \( ID \) for each question.
    \item Compute the total difficulty score for each question: \( D_i = CL_i + MC_i + ID_i \).
    \item Sum the scores across all questions to get \( D_{\text{total}} \).
    \item Compute the maximum possible difficulty: \( D_{\text{max}} = 120 \times N \).
    \item Compute relative difficulty using:
    \[
    D_{\text{relative}} = \frac{D_{\text{total}}}{D_{\text{max}}} \times 100
    \]
    \item Interpret the result:
    \begin{itemize}
        \item \( D_{\text{relative}} < 40\% \) indicates an easy test.
        \item \( 40\% \leq D_{\text{relative}} < 70\% \) indicates a moderate test.
        \item \( D_{\text{relative}} \geq 70\% \) indicates a hard test.
    \end{itemize}
\end{enumerate}

\section{Conclusion}
This system provides a structured, quantitative approach to measuring test difficulty, ensuring consistency across different assessments. It allows educators to design instruments that are appropriately challenging and balanced.

\end{document}
