\documentclass[handout]{beamer}
\input{Lecture-Slides/preamble.txt}


% Title Information

\title{Introduction to Statistical Methods in Political Science}
\subtitle{Lecture 11: Large-Sample Inference for Means}
\author{Ignacio Urbina \texorpdfstring{\\ \vspace{0.3em}}{ } \scriptsize \textcolor{gray}{Ph.D. Candidate in Political Science}}
\date{}

%--------------------------------------------------------------------------
\begin{document}
\frame{\titlepage}

\section{Confidence Intervals for Means}

% Transition slide: Confidence Intervals
\transitionslide{Confidence Intervals for Means}


\begin{frame}
\frametitle{Motivation: From Proportions to Means}
Last week, we focused on inference for \textbf{population proportions ($p$)}.
\begin{itemize}
\item Confidence intervals and hypothesis tests for $p$ using large-sample Z-procedures.
\end{itemize}
Now, we shift to \textbf{quantitative (continuous) data} where the parameter is the \textbf{population mean ($\mu$)}.
\begin{itemize}
\item Examples: Average height, mean income, time.
\end{itemize}
\textbf{Goal:} Develop inference methods for $\mu$ using CLT and the sampling distribution of $\bar{x}$.
\end{frame}

\subsection{Confidence Interval: One Mean}

\begin{frame}
\frametitle{Foundation: The Central Limit Theorem (Recap)}
For large samples ($n \ge 30$), the sampling distribution of $\bar{x}$ is approximately normal:
\[ \bar{x} \sim N\left(\mu, \frac{\sigma^2}{n}\right) \]
\begin{itemize}
\item Approximate normality holds regardless of the original population distribution.
\end{itemize}
\end{frame}

\begin{frame}
\frametitle{Handling Unknown $\sigma$ \& The Z-Statistic}
When $\sigma$ is unknown, we estimate it using the sample standard deviation $s$ (plug-in principle).
The resulting test statistic:
\[ Z = \frac{\bar{x} - \mu}{s/\sqrt{n}} \]
\end{frame}

\begin{frame}
\frametitle{Large-Sample CI for $\mu$: Formula \& Structure}
Confidence Interval formula:
\[ \bar{x} \pm z_{1-\frac{\alpha}{2}} \left(\frac{s}{\sqrt{n}}\right) \]
Structure:
\begin{itemize}
\item Point Estimate: $\bar{x}$
\item Margin of Error: $z_{1-\frac{\alpha}{2}} \times \dfrac{s}{\sqrt{n}}$
\end{itemize}
\end{frame}

\begin{frame}
\frametitle{Margin of Error for $\mu$ CI}
Depends on:
\begin{enumerate}
\item Critical Value $z_{1-\frac{\alpha}{2}}$
\item Standard Error $SE = \dfrac{s}{\sqrt{n}}$
\end{enumerate}
Thus, the Margin of Error (ME) is:
\[ ME = z_{1-\frac{\alpha}{2}} \times SE \]
\end{frame}

\begin{frame}
\frametitle{Interpretation of a Confidence Interval for $\mu$}
Example: 95\% CI = (5.2, 6.8)
Correct Interpretation:
\begin{itemize}
\item \textbf{"We are 95\% confident that $\mu$ lies within (5.2, 6.8)."}
\end{itemize}
\end{frame}

\begin{frame}
\frametitle{Example: CI for Average Commute Time (Problem Statement)}
\textbf{Problem:} What is the average daily commute time for workers in a city? \newline
\textbf{Data:} $n=100$ workers, sample mean commute time $\bar{x}=32.5$ minutes, sample standard deviation $s=7.0$ minutes. \newline
\textbf{Task:} Construct a 95\% confidence interval for the true mean commute time.
\end{frame}

\begin{frame}
\frametitle{Example: CI for Average Commute Time (Calculation)}
Given:
\begin{itemize}
\item $n = 100$, $\bar{x} = 32.5$, $s = 7.0$
\item 95\% confidence level ($\alpha=0.05$) $\Rightarrow z_{1-\frac{\alpha}{2}} = z_{0.975}= 1.96$
\end{itemize}
Standard error:
\[ SE = \frac{7.0}{\sqrt{100}} = 0.7 \]
Margin of error:
\[ ME = 1.96 \times 0.7 = 1.372 \]
Confidence Interval:
\[ (32.5 - 1.372,\ 32.5 + 1.372) = (31.128,\ 33.872) \]
\end{frame}

\subsection{Confidence Interval: Two Means}

\begin{frame}
\frametitle{Foundation: Distribution of $\bar{x}_1 - \bar{x}_2$ (Recap)}
For large independent samples:
\[ \bar{x}_1 - \bar{x}_2 \sim N\left(\mu_1 - \mu_2, \frac{\sigma_1^2}{n_1} + \frac{\sigma_2^2}{n_2}\right) \]
\end{frame}

\begin{frame}
\frametitle{Handling Unknown Variances \& The Z-Statistic (Two Means)}
Estimate population variances with sample variances:
\[ s_1^2 \quad \text{and} \quad s_2^2 \]
Standard error for $\bar{x}_1 - \bar{x}_2$:
\[ SE = \sqrt{\frac{s_1^2}{n_1} + \frac{s_2^2}{n_2}} \]
Use $z$-statistic for large samples.
\end{frame}

\begin{frame}
\frametitle{Large-Sample CI for $\mu_1 - \mu_2$: Formula \& Structure}
Confidence Interval:
\[ (\bar{x}_1 - \bar{x}_2) \pm z_{1-\frac{\alpha}{2}} \times SE \]
\[ SE = \sqrt{\frac{s_1^2}{n_1} + \frac{s_2^2}{n_2}} \]
\end{frame}

\begin{frame}
\frametitle{Interpretation of CI for $\mu_1 - \mu_2$}
\begin{itemize}
\item Positive Interval: $\mu_1 > \mu_2$
\item Interval Contains 0: No significant difference
\item Negative Interval: $\mu_1 < \mu_2$
\end{itemize}
\end{frame}

\begin{frame}
\frametitle{Example: CI for Comparing Study Methods (Problem Statement)}
\textbf{Problem:} Do two different study methods lead to different average exam scores among university students? \newline
\textbf{Data:} Method 1 ($n_1=50$, $\bar{x}_1=85.2$, $s_1=8.5$); Method 2 ($n_2=60$, $\bar{x}_2=81.5$, $s_2=9.1$). \newline
\textbf{Task:} Construct a 99\% confidence interval for $\mu_1 - \mu_2$.
\end{frame}

\begin{frame}
\frametitle{Example: CI for Comparing Study Methods (Calculation)}
Given:
\begin{itemize}
\item $n_1=50$, $\bar{x}_1=85.2$, $s_1=8.5$
\item $n_2=60$, $\bar{x}_2=81.5$, $s_2=9.1$
\item 99\% confidence level ($\alpha=0.01$) $\Rightarrow z_{1-\frac{\alpha}{2}} = z_{0.995} = 2.576$
\end{itemize}
Standard error:
\[ SE = \sqrt{\frac{8.5^2}{50} + \frac{9.1^2}{60}} \approx 1.68 \]
Margin of error:
\[ ME = 2.576 \times 1.68 \approx 4.33 \]
Confidence Interval:
\[ (85.2 - 81.5) \pm 4.33 = 3.7 \pm 4.33 \]
\[ (-0.63,\ 8.03) \]
\end{frame}

\section{Hypothesis Testing for Means}

% Transition slide: Hypothesis Testing
\transitionslide{Hypothesis Testing for Means}


\begin{frame}
\frametitle{Motivation: Hypothesis Tests for Means}
Apply hypothesis testing to population \textbf{means ($\mu$)}:
\begin{itemize}
\item Steps: Hypotheses, $\alpha$, Test Statistic, Decision, Conclusion
\end{itemize}
\end{frame}

\subsection{Hypothesis Test: One Mean (Two-Sided)}

\begin{frame}
\frametitle{Example: Average Hours of Sleep (Problem Statement)}
\textbf{Problem:} Do college students sleep an average of 7 hours per night? \newline
\textbf{Data:} $n=64$, $\bar{x}=6.8$ hours, $s=0.6$ hours. \newline
\textbf{Task:} Test $H_0: \mu = 7$ against $H_a: \mu \neq 7$ at $\alpha=0.05$.
\end{frame}

\begin{frame}
\frametitle{Example: Average Hours of Sleep (Calculation)}
Calculate standard error:
\[ SE = \frac{0.6}{\sqrt{64}} = 0.075 \]
Test statistic:
\[ Z = \frac{6.8-7.0}{0.075} = -2.67 \]
Critical value for a test with $\alpha=0.05$: $\pm1.96$.\newline 
Since $|-2.67| > 1.96$, reject $H_0$.
\end{frame}

\subsection{Hypothesis Test: Two Means (One-Sided)}

\begin{frame}
\frametitle{Example: New Drug Effectiveness (Problem Statement)}
\textbf{Problem:} Is a new drug more effective at reducing blood pressure than the current standard drug? \newline
\textbf{Data:} Drug 1 ($n_1=120$, $\bar{x}_1=15.5$, $s_1=5.0$); Drug 2 ($n_2=100$, $\bar{x}_2=13.8$, $s_2=4.5$). \newline
\textbf{Task:} Test $H_0: \mu_1 - \mu_2 = 0$ vs. $H_a: \mu_1 - \mu_2 > 0$ at $\alpha=0.01$.
\end{frame}

\begin{frame}
\frametitle{Example: New Drug Effectiveness (Calculation)}
Calculate standard error:
\[ SE = \sqrt{\frac{5.0^2}{120} + \frac{4.5^2}{100}} \approx 0.641 \]
Test statistic:
\[ Z = \frac{15.5-13.8}{0.641} \approx 2.65 \]
Critical value: $z_{0.99} \approx 2.33$.
Since $2.65 > 2.33$, reject $H_0$.
\end{frame}

\subsection{Hypothesis Testing: Nonzero Null Values}

\begin{frame}
\frametitle{Hypothesis Tests with Nonzero Null Values}
Previously, we tested $H_0: \mu = 0$ or $H_0: \mu_1 - \mu_2 = 0$.\newline
Now, we allow the null hypothesis to state a nonzero value:
\begin{itemize}
\item $H_0: \mu = \mu_0$ where $\mu_0 \neq 0$
\item $H_0: \mu_1 - \mu_2 = \Delta_0$ where $\Delta_0 \neq 0$
\end{itemize}
\textbf{Interpretation:} We are testing whether the population mean or the difference between two means equals a specific number.
\end{frame}

\begin{frame}
\frametitle{Adjusted Z-Statistic Formulas}
\textbf{One Mean:}
\[
Z = \frac{\bar{x} - \mu_0}{s / \sqrt{n}}
\]
where $\mu_0$ is the null hypothesized mean (not necessarily 0).\newline

\textbf{Two Means:}
\[
Z = \frac{(\bar{x}_1 - \bar{x}_2) - \Delta_0}{\sqrt{\frac{s_1^2}{n_1} + \frac{s_2^2}{n_2}}}
\]
where $\Delta_0$ is the hypothesized difference (often 0, but not always).\newline

\textbf{Note:} Subtract $\mu_0$ or $\Delta_0$ when forming the numerator.
\end{frame}

\subsection{Hypothesis Test: One Mean (Nonzero Null Value)}

\begin{frame}
\frametitle{Example: Average Calories Consumed (Problem Statement)}
\textbf{Problem:} Do individuals consume an average of 2500 calories per day? \newline
\textbf{Data:} $n=49$, $\bar{x}=2550$ calories, $s=150$ calories. \newline
\textbf{Task:} Test $H_0: \mu = 2500$ against $H_a: \mu \neq 2500$ at $\alpha=0.05$.
\end{frame}

\begin{frame}
\frametitle{Example: Average Calories Consumed (Calculation)}
Calculate standard error:
\[
SE = \frac{150}{\sqrt{49}} = 21.43
\]
Test statistic:
\[
Z = \frac{2550-2500}{21.43} \approx 2.33
\]
Critical value: $\pm1.96$. \newline
Since $2.33 > 1.96$, reject $H_0$.
\end{frame}

\subsection{Hypothesis Testing: Two Means (Nonzero Difference)}

\begin{frame}
\frametitle{Hypothesis Tests for Two Means: Nonzero Difference}
In many situations, we want to test whether the difference between two population means equals a value other than zero.

Examples:
\begin{itemize}
\item Testing if a new product improves scores by 5 points compared to an old product.
\item Checking if two treatments differ by a clinically significant margin (e.g., 2 mmHg in blood pressure).
\end{itemize}
\textbf{Null hypothesis:} $H_0: \mu_1 - \mu_2 = \Delta_0$ where $\Delta_0 \neq 0$.
\end{frame}

\begin{frame}
\frametitle{Adjusted Z-Statistic: Two Means}
For large independent samples, the test statistic becomes:
\[
Z = \frac{(\bar{x}_1 - \bar{x}_2) - \Delta_0}{\sqrt{\frac{s_1^2}{n_1} + \frac{s_2^2}{n_2}}}
\]
where:
\begin{itemize}
\item $\Delta_0$ is the hypothesized difference between means.
\item $SE = \sqrt{\dfrac{s_1^2}{n_1} + \dfrac{s_2^2}{n_2}}$ is the standard error.
\end{itemize}
\textbf{Key change:} Subtract $\Delta_0$ from the observed difference $(\bar{x}_1 - \bar{x}_2)$ in the numerator.
\end{frame}

\begin{frame}
\frametitle{Example: Comparing Manufacturing Processes (Problem Statement)}
\textbf{Problem:} Is there a difference of 5 units in mean output between two manufacturing processes? \newline
\textbf{Data:} Process 1 ($n_1=40$, $\bar{x}_1=105$, $s_1=10$); Process 2 ($n_2=50$, $\bar{x}_2=98$, $s_2=12$). \newline
\textbf{Task:} Test $H_0: \mu_1 - \mu_2 = 5$ against $H_a: \mu_1 - \mu_2 \neq 5$ at $\alpha=0.05$.
\end{frame}

\begin{frame}
\frametitle{Example: Comparing Manufacturing Processes (Calculation)}
Calculate standard error:
\[
SE = \sqrt{\frac{10^2}{40} + \frac{12^2}{50}} \approx 2.32
\]
Test statistic:
\[
Z = \frac{(105-98) - 5}{2.32} \approx 0.43
\]
Critical value: $\pm1.96$. \newline
Since $0.43 < 1.96$, fail to reject $H_0$.
\end{frame}

%...................................................................
\section{Large Sample Inference with Paired Samples}

% Transition slide: 
\transitionslide{Large Sample Inference with Paired Samples}


\begin{frame}
\frametitle{Paired Samples: When and Why?}
\textbf{Paired data} arise when observations are naturally matched:
\begin{itemize}
    \item Before-and-after measurements (e.g., pre- and post-treatment)
    \item Measurements on the same individual under two conditions. Hence, before and after data \emph{are not independent}.
\end{itemize}
\textbf{Key idea:} Reduce the two measurements to a single difference score:
\[ d_i = x_{i,1} - x_{i,2} \]
Then apply inference methods to the sample of differences. Key assumption: \emph{Independence across individuals}.
\end{frame}

\begin{frame}
\frametitle{Paired Samples: CI and Hypothesis Test}
\small 
\begin{columns}[t]
    % Left column: Confidence Interval
    \begin{column}{0.48\textwidth}
    \textbf{Confidence Interval for $\mu_d$}

    \vspace{0.5em}
    For large samples of paired differences ($n \ge 30$), apply CLT:
    \[
    \bar{d} \sim N\left(\mu_d, \frac{s_d^2}{n}\right)
    \]

    \vspace{0.3em}
    CI formula:
    \[
    \bar{d} \pm z_{1-\frac{\alpha}{2}} \times \frac{s_d}{\sqrt{n}} 
    \]

    \vspace{0.3em}
    \begin{itemize}
        \item $\bar{d}$: Mean of differences
        \item $s_d$: Std. dev. of differences
        \item $z_{1-\frac{\alpha}{2}}$: Critical value
    \end{itemize}
    \end{column}

    % Right column: Hypothesis Test
    \begin{column}{0.48\textwidth}
    \textbf{Hypothesis Test for $\mu_d$}

    \vspace{0.5em}
    Test the null hypothesis:
    \[
    H_0: \mu_d = \mu_{d,0} \quad \text{vs.} \quad H_a: \mu_d \neq \mu_{d,0}
    \]

    \vspace{0.3em}
    Z-statistic:
    \[
    Z = \frac{\bar{d} - \mu_{d,0}}{s_d / \sqrt{n}}
    \]

    \vspace{0.3em}
    \begin{itemize}
        \item Compare $Z$ to critical value or use $p$-value
        \item Reject $H_0$ if evidence is strong
    \end{itemize}
    \end{column}
\end{columns}

\end{frame}


\begin{frame}
\frametitle{Example: Effect of a Sleep Intervention}
\small 
\textbf{Problem:} Does a cognitive-behavioral sleep intervention improve sleep duration among adults with mild insomnia? 

\textbf{Study Design:} A sample of $n = 200$ individuals participated in a 4-week cognitive-behavioral therapy (CBT) program targeting sleep hygiene, routines, and stress reduction. Each participant recorded their average nightly sleep duration \textit{before and after} the program. \newline

\textbf{Data:} Mean difference $\bar{d} = 0.75$ hours, $s_d = 1.2$ hours, $n=200$. \newline
\textbf{Task:} Construct a 95\% confidence interval for the mean increase in sleep duration, $\mu_d$. \pause
\[
SE = \frac{1.2}{\sqrt{200}} \approx 0.085,\quad z_{0.975} = 1.96
\]
\[
ME = 1.96 \times 0.085 \approx 0.17,\quad CI = (0.75 \pm 0.17) = (0.58,\ 0.92)
\]
\textbf{Conclusion:} We are 95\% confident the CBT program increased sleep duration by 0.58 to 0.92 hours per night.
\end{frame}




%...................................................................
\section{Summary}
% Transition slide: Summary

\transitionslide{Summary: Key Formulas and Concepts}


\begin{frame}
\frametitle{Summary Table}

\begin{table}[H]
\centering
\renewcommand{\arraystretch}{3.5}
\resizebox{.95\textwidth}{!}{%
\begin{tabular}{|p{0.2\textwidth}|c|c|p{0.26\textwidth}|}
\hline
\textbf{Scenario} & \textbf{Confidence Interval (CI)} & \textbf{Hypothesis Test (Z-Statistic)} & \textbf{Notes} \\
\hline
\textbf{One Mean ($\mu$)} & $\bar{x} \pm z_{1-\frac{\alpha}{2}} \dfrac{s}{\sqrt{n}}$ & $Z = \dfrac{\bar{x} - \mu_0}{s/\sqrt{n}}$ & $\mu_0$ can be 0 or nonzero \\
\hline
\textbf{Two Means ($\mu_1 - \mu_2$)} & $(\bar{x}_1-\bar{x}_2) \pm z_{1-\frac{\alpha}{2}} \sqrt{\dfrac{s_1^2}{n_1}+\dfrac{s_2^2}{n_2}}$ & $Z = \dfrac{(\bar{x}_1-\bar{x}_2) - \Delta_0}{\sqrt{\dfrac{s_1^2}{n_1}+\dfrac{s_2^2}{n_2}}}$ & $\Delta_0$ can be 0 or nonzero \\
\hline
\textbf{Paired Samples ($\mu_d$)} & $\bar{d} \pm z_{1-\frac{\alpha}{2}} \dfrac{s_d}{\sqrt{n}}$ & $Z = \dfrac{\bar{d} - \mu_{d,0}}{s_d/\sqrt{n}}$ & Analyze differences $d_i = x_{i,1} - x_{i,2}$. \\
\hline
\end{tabular}
}
\end{table}
\scriptsize 
\textbf{Important:}
\begin{itemize}
\item Subtract $\mu_0$, $\Delta_0$, or $\mu_{d,0}$ when performing a hypothesis test.
\item $\mu_d=0$ equals no difference for the average individual difference.
\item For confidence intervals, center at the sample statistic.
\item $z_{1-\frac{\alpha}{2}}$ depends on the confidence level (e.g., 1.96 for 95\%).
\end{itemize}

\end{frame}



\end{document}


