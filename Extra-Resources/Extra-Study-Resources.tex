\documentclass[12pt]{article}
\usepackage{geometry}
\geometry{margin=0.8in}
\usepackage{setspace}
\usepackage{enumitem}
\usepackage{parskip}
\usepackage{hyperref}
\usepackage{newtxtext,newtxmath}
\usepackage{enumitem}
\usepackage{titlesec}

% Times New Roman font
\usepackage{newtxtext,newtxmath}

\titleformat{\section}{\normalfont\Large\bfseries}{\thesection}{1em}{}
\titleformat{\subsection}{\normalfont\large\bfseries}{\thesubsection}{1em}{}


\title{Extra Resources to Support Your Study}
\date{}

\renewcommand{\normalsize}{\fontsize{13pt}{15pt}\selectfont}


\begin{document}

\maketitle

\section*{1. Introduction: When and How to Use Additional Resources}

Additional resources can be incredibly helpful, but they are most effective when used intentionally and strategically. If you’re struggling to understand a concept from lecture or the textbook, or if you need reinforcement through alternative explanations, curated online resources can provide clarity.

However, the key is to use them as supplements—not replacements—for core materials and active practice. Watching videos or reading alternative explanations is most valuable \emph{after} you’ve made a genuine effort to engage with the material on your own.

Ask yourself:
\begin{itemize}
    \item Do I understand this concept well enough to explain it in my own words?
    \item Can I solve representative problems without looking at the solution?
    \item Have I truly understood the foundational concepts that this topic builds on?
\end{itemize}

If not, revisit prior units and reinforce core concepts first. Then, use these resources to deepen and solidify your understanding:
\begin{itemize}
    \item \textbf{Not as a substitute for retrieval practice}, but as a supplement after you've tested your understanding. For example, after struggling through a practice problem or trying to recall a definition, watch a video to fill in gaps and reinforce.

    \item \textbf{To deepen conceptual understanding}, not to memorize procedures. If you find yourself unsure about why a formula works or when to apply a method, a well-explained video (like StatsQuest or JB Statistics) can provide the “why,” which is key to flexible thinking.

    \item \textbf{To build connections across topics}. If a new concept feels disconnected, look for resources that place it in context. Khan Academy and JB Statistics often link new material to foundational ideas—use this to reinforce the “chain” of concepts you’ve built.

    \item \textbf{To clarify definitions and terminology}. Use resources like the Oxford Dictionary of Statistics when you're unsure about precise terms. Definitions are not filler—they are the language of the problems and will show up repeatedly in questions and explanations.

    \item \textbf{Only after struggling productively}. Resist the urge to watch a solution video as soon as you’re confused. First, try to work through the difficulty yourself. Looking up explanations too early short-circuits your learning and gives you the illusion of understanding.

    \item \textbf{In small, spaced doses}, not marathon sessions. Watching five videos back-to-back feels busy but is less effective than engaging deeply with one, followed by retrieval practice or problem solving.

    \item \textbf{To reinforce—not replace—problem solving}. Videos and tutorials are helpful, but only if followed by hands-on work. Practice is where actual learning takes place. Don’t let clarity from a video become a passive illusion—test whether you can do it yourself.
\end{itemize}

\textit{In short: use these resources strategically, actively, and always in partnership with the kinds of effective study habits that actually lead to learning.}


\section*{2. Recommended Additional Resources}

Here are trusted and student-friendly resources to support your understanding of statistics, especially for introductory social science contexts:

\begin{itemize}
    \item \textbf{Khan Academy – AP Statistics:}\\
    Excellent for visual learners and quick topic reviews. Content is aligned with core statistical concepts and includes video walkthroughs and exercises.\\
    \href{https://www.khanacademy.org/math/ap-statistics}{\texttt{https://www.khanacademy.org/math/ap-statistics}}

    \item \textbf{StatQuest with Josh Starmer:}\\
    Highly engaging and clear explanations of statistical methods, distributions, and inference concepts. Particularly good for building intuition.\\
    \href{https://statquest.org/video-index/}{\texttt{https://statquest.org/video-index/}}

    \item \textbf{Jeremy Balka’s JB Statistics (YouTube and Website):}\\
    Well-structured videos with an academic tone that covers essential material with clarity and rigor. Great for review and deeper dives. There is also a free textbook available. \\
    \href{https://www.jbstatistics.com/}{\texttt{https://www.jbstatistics.com/}}

    \item \textbf{Oxford Dictionary of Statistics:}\footnote{Log-in to your SBU Google account to access the material.}\\
    Useful for looking up precise definitions of statistical terms and concepts. Ideal when you encounter definitions you feel do not fully grasp. It has a search tool you can use to look up definitions, concepts, and terms.  \\
    \href{https://www-oxfordreference-com.proxy.library.stonybrook.edu/display/10.1093/acref/9780199679188.001.0001/acref-9780199679188}{\texttt{Oxford Reference Online (SBU Library Access Required)}}

    \item \textbf{Against All Odds: Inside Statistics} \\
    A high-quality video series that integrates real-world applications of statistics with clear, engaging explanations. Each 6–14 minute episode introduces a concept through authentic contexts—from analyzing lightning patterns to investigating disease resistance and environmental risk. This resource emphasizes \emph{doing} statistics, not just learning about it.
        Best used to:
        \begin{itemize}
            \item Build intuition by seeing how statistics applies across disciplines (science, medicine, policy, journalism, etc.).
            \item Reinforce conceptual understanding with memorable stories and visuals.
            \item Reflect on the logic of inference in messy, real-world data—beyond clean textbook examples.
        \end{itemize}

\href{https://www.learner.org/series/against-all-odds-inside-statistics/}{\texttt{https://www.learner.org/series/against-all-odds-inside-statistics/}}

\end{itemize}

\section*{3. Practice: Applying What You’ve Learned}

Ultimately, real learning happens when you actively apply concepts. Videos and summaries may feel helpful, but they won’t replace the value of working through problems on your own.

\textbf{It is strongly recommend that you:}
\begin{itemize}
    \item Work through the \textbf{odd-numbered problems} at the end of each subunit and chapter of the course textbook. All the odd numbered problems include fully developed answers at the end of the book, allowing you to check your work. Download the textbook if you haven't done so already:   \href{https://www.openintro.org/book/os/}{\texttt{https://www.openintro.org/book/os/}}.
    \item Select a \textbf{diverse set of problems}—not just the ones that look familiar. Include word problems, conceptual reasoning, and calculation-based questions.
    \item Attempt problems \textbf{independently}. Struggle productively before seeking the solution. This is where true learning occurs.
    \item Ensure you have a solid understanding of prior units before moving on. \textbf{Do not skip foundational concepts}, even if they feel tedious—many future topics build directly on them.
    \item Track your progress: if you can develop complete and correct solutions to several different problems within a subunit, that’s a strong sign you understand the material.
\end{itemize}


\end{document}
